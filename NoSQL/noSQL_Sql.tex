\textbf{\textcolor{darkblue}{ SQL vs. NoSQL}}~

\section*{SQL vs. NoSQL}
\subsection*{Frage 1}
Was sind die Vor- und Nachteile von 2 bzw. 3 Tier-Architekturen? Wo können Skalierungsprobleme auftreten und wie lassen sie sich lösen?
\subsection*{Antwort}
\begin{itemize}
	\item 
\end{itemize}

\subsection*{Frage 2}
Was sind die Stärken und Schwaächen von RDBMS?
\subsection*{Antwort}
\begin{itemize}
	\item 
\end{itemize}

\subsection*{Frage 3}
Welche Aspekte werden durch NoSQL-Datenbanken adressiert?
\subsection*{Antwort}
\begin{itemize}
	\item 
\end{itemize}

\subsection*{Frage 4}
Was ist der Unterschied zwischen Scale-Up und Scale-Out?
\subsection*{Antwort}
\begin{itemize}
	\item vertikale Skalierbarkeit Scale UP
	\item horizontale Skalierbarkeit Scale Out
\end{itemize}

\subsection*{Frage 5}
Erläutern Sie die Unterschiede zwischen Shared Memory, Shared Disk und Shared Nothing Systemen? Was sind die Vor- und Nachteile?
\subsection*{Antwort}
logisch

\subsection*{Frage 6}
Was bedeutet "schemafreie Datenmodellierung"?
\subsection*{Antwort}
kaum restriktionen , einfache key-value stores oder dokument stores

\subsection*{Frage 7}
Was füer Arten von NoSQL-DBS gibt es? Charakterisieren Sie diese kurz und nennen Sie jeweils Beispiele für konkrete Systeme!
\subsection*{Antwort}
Die Mischung macht's: Kombination aus verschiedenen Technologien, um Optimum zu erreichen, 
Sehr Komplexe Infrastruktur!

\subsection*{Frage 8}
Was bedeutet Polyglot Persistence? Bewerten Sie den Ansatz!
\subsection*{Antwort}
\begin{itemize}
	\item 
\end{itemize}

\subsection*{Frage 9}
Was versteht man unter Big Data? Wofür stehen die 3 bzw. 4 V's? Welche der V's adressieren die meisten NoSQL-Datenbanken?
\subsection*{Antwort}
\begin{itemize}
	\item Volume ( Quantität der Daten)
	\item Variety (Art der strukturierung der Daten)
	\item Veracity ( unglaubwüdige Daten)
	\item Velocity ( daten werden schnell, exponential erzeugt)
	
\end{itemize}
