\textbf{\textcolor{darkblue}{ Verteilte Datenverarbeitung}}~

\section*{Verteilte Datenverarbeitung}
\subsection*{Frage 1}
Was ist das 2 PC –Protokoll? Wie funktioniert es und wofür wird es benötigt?
\subsection*{Antwort}
\begin{itemize}
	\item  nur wenn mehrere die schreibvorgänge erfolgreich abgeschlossen haben, war es erfolgreich
\end{itemize}

\subsection*{Frage 2}
Welchen Einfluss hat das 2PC Protokoll auf die Verfügbarkeit von verteilten DBMS
\subsection*{Antwort}
\begin{itemize}
	\item 
\end{itemize}

\subsection*{Frage 3}
Was versteht man unter Sharding und Auto-Sharding?
\subsection*{Antwort}
\begin{itemize}
	\item 
\end{itemize}
\subsection*{Frage 4}
Erläutern Sie das Konzept der Replikation. Welche Arten gibt es? Welche Probleme kann man damit lösen und welche bekommt man dafür
\subsection*{Antwort}
\begin{itemize}
	\item 
\end{itemize}
\subsection*{Frage 5}
Wie kann man Konsistenz bei replizierten Daten sicherstellen? Und bei partitionierten Daten?
\subsection*{Antwort}
\begin{itemize}
	\item 
\end{itemize}
\subsection*{Frage 6}
Was besagt das CAP-Theorem? Geben Sie eine anschauliche Begründung!
\subsection*{Antwort}
\begin{itemize}
	\item Konsistenz
	\item verfügbarkeit
	\item ausfallsicher
\end{itemize}
\subsection*{Frage 7}
Was sind jeweils die Vorteile von CP und AP-Systemen und was die Einschränkungen?
\subsection*{Antwort}
\begin{itemize}
	\item cp = konsisteznz
	\item ap = verfügbar und ausfallsicher
\end{itemize}
\subsection*{Frage 8}
Stellen Sie die Konsistenzmodelle ACID und BASE gegenüber!
\subsection*{Antwort}
\begin{itemize}
	\item ACID schwerpunkt auf konsistenz ( 2 Phasen commit)
	\item BASE schwerpunkt verfügbarkeit,  späte konflikt erkennung
\end{itemize}