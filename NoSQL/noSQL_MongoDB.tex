\textbf{\textcolor{darkblue}{ MongoDB}}~

\section*{MongoDB}
\subsection*{Frage 1}
Was sind die wesentlichen Vor- und Nachteile dokumentenorientierter zur relationalen Datenbanken?
\subsection*{Antwort}
Vorteile
\begin{itemize}
	\item Zugriffe und Abfragen direkt auf Dokumente Möglich
	\item keine Schemarestriktion
	\item hohe Flexibilität mit großen Datenmengen
	\item Geschwindigkeitsvorteil gegenüber RDBMS
	\item bessere horizontale Skalierung
\end{itemize}
Nachteile
\begin{itemize}
	\item geringe Verbreitung
 	\item damit auch geringe Unterstützung von Schnittstellen und Tools
 	\item keine Dokumentenübergreifenden Operationen 
 	\item darstellen komplexer Beziehungen eher schwierig
 	\item durch Schemafreiheit müssen eventuelle Wertprüfungen nicht von DB  
 	  übernommen müssen selbst programmiert werden
\end{itemize}
\subsection*{Frage 2}
Für welche Anwendungsbereiche bietet sich die Nutzung von MongoDB an?
\subsection*{Antwort}
\begin{itemize}
	\item Anwendungen ohne festes Datenbankschema
	\item Anwendungen in denen Daten in nicht tabellarischer Form gespeichert werden
	\item Webanwendungen, da JSON auch zur Serialisierung verwendet wird
	\item Anwendungen mit sehr großen Datenmengen
	\item Anwendungen bei denen viele Nutzer versch. Daten erzeugen und diese schnell gespeichert / verarbeitet werden müssen
\end{itemize}

\subsection*{Frage 3}
Erläutern Sie den grundlegenden Aufbau von MongoDB?
\subsection*{Antwort}
MongoDb speichert als Grundeinheit JSON Objekte und besteht aus den folgenden Bestandteilen:

Datenbanken
\begin{itemize}
	\item MongoDB-Prozess kann mehrere Datenbanken verwalten
	\item Datenbank besteht aus beliebig vielen Collections
	\item DatenbankName.CollectionName stellt Namespace dar
\end{itemize}

Collections
\begin{itemize}
	\item speichert Dokumente
	\item Datenbank besteht aus beliebig vielen Collections
	\item vergleichbar mit Tabelle einer relationalen DB
	\item Dokumente können völlig verschieden sein (Schemalos)
\end{itemize}

Capped Collections
\begin{itemize}
	\item größenbeschränkte Dokumentensammlung
	\item Größe und Anzahl an Dokumenten kann beim Anlegen festgelegt werden
	\item Verhält sich wie digitaler Ringspeicher nach Erreichen der Größe
	\item kann tailable Cursors verwenden
\end{itemize}

System Collections
\begin{itemize}
	\item automatisch von MongoDb erzeugt
	\item Verwalten Indizes, Namespaces, JavaScript und Benutzer
\end{itemize}

\subsection*{Frage 4}
Wie kommuniziert eine Anwendung mit Shards im Sharding Cluster und warum findet die Kommunikation so statt?
\subsection*{Antwort}
Eine Anwendung kommuniziert über Router (Schnittstellen) mit den Shards. Dabei werden in sogenannten Config Servers Metadaten und Konfigurationseinstellungen gespeichert. 
Die Kommunikation soll/muss so stattfinden, da die Anwendung sonst nicht weiß, wie Datenbankoperationen ausgeführt werden. Zum Beispiel: Wie interpretiere ich Datenbankoperationen? Auf welchem Shard befindet sich welcher Datensatz? (s. Folie 41/42 für Veranschaulichung)
\subsection*{Frage 5}
Wie funktioniert Replikation in MongoDB? Welches Konsistenzmodell kommt dabei zum Einsatz?
\subsection*{Antwort}
MongoDB implementiert eine Master-Slave Replikation. Es gibt nur einen Master (Primary) und nur auf dem Master können Write-Operationen ausgeführt werden. Auf den Slaves (Secondaries) kann nur gelesen werden. Bei einem Write im Master werden die Daten asynchron auf die Slaves aktualisiert. (s. Folie 36 für Veranschaulichung)

\subsection*{Frage 6}
Wann wird bei Replica-Sets ein neuer Primary Server gewählt?
\subsection*{Antwort}
Ein Primary Server wird dann neu gewählt,  wenn…
\begin{itemize}
	\item der Primary Server ausfällt.
	\item ein neues Replica-Set (Secondary) zum System hinzufügt wird.
	\item ein Replica-Set die Verbindung zum Primary verliert.
	\item ein Primary zurücktritt und zu einem Replica-Set wird.
\end{itemize}
Eine Wahl beginnt, indem sich ein Replica-Set als neuen Primary vorschlägt. Jedes andere Replica-Set hat dann genau eine Stimme, um dafür oder dagegen zu stimmen.
Die Wichtigsten Kriterien damit ein Replica-Set zum Primary gewählt wird sind folgende:
\begin{itemize}
	\item Antwortzeit von anderen Replica-Sets, nach dem man sie angepingt hat (Replica-Set mit der schnellsten Antwortzeit bekommt die Stimme)
	\item Priorität (muss manuell festgelegt werden, Replica-Set mit der höchsten Priorität leitet als erstes eine Wahl für sich selbst ein)
	\item Aktuellster Stand (letzte Datenbankoperation ist ausgeführt worden)
	\item Aufrechte Verbindung zu mindestens der Hälfte aller Replica-Sets
\end{itemize}

\subsection*{Frage 7}
Wie funktioniert prinzipiell das Map-Reduce-Verfahren?
\subsection*{Antwort}
Mit Hilfe von Map-Reduce soll erreicht werden, dass die Gesamtlast auf verschiedene Rechner verteilt wird, um so die Performance zu stärken. Vergleichbar mit einer großen Pizza: Je mehr Leute von der Pizza essen, desto schneller ist diese vertilgt. 

