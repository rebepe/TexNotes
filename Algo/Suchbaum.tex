\newcommand{\red}{\textcolor{red}{T1}}
\newcommand{\blue}{\textcolor{blue}{T4}}
\newcommand{\bluedrei}{\textcolor{blue}{T3}}
\newcommand{\tg}[1]{\textcolor{green}{#1}}
\newcommand{\tdg}[1]{\textcolor{darkgreen}{#1}}
\newcommand{\cen}{\textcolor{orange}{\textbf{c}}}
\newcommand{\reda}{\textcolor{red}{a}}
\newcommand{\blueb}{\textcolor{blue}{b}}


\minpurp{Rotationen}
\begin{tikzpicture}[level distance=1.75em, sibling distance = 2em]
\node{ \blueb }
child{ node{\reda}
child{ node{\red}}
child{ node{\tg{T2}}}}
child{ node{\bluedrei}
}
;
\end{tikzpicture}
\begin{minipage}[b]{0.1\textwidth} \centering 
Rechts um \textcolor{red}{a} \\
$\Longrightarrow$ \\
Links um \textcolor{blue}{b} \\
$\Longleftarrow$
\end{minipage}
\begin{tikzpicture}[level distance=1.75em, sibling distance = 2em]
\node{\reda} 
child{ node{\red}}
child{node{\blueb}
child{ node{\tg{T2}}}
child{node{\bluedrei}}
};
\end{tikzpicture}




\minpurp{Doppelrotation}
\begin{tikzpicture}[level distance=1.75em, sibling distance = 2em]
\node{\blueb} 
child{ node{\reda}
child{ node{\red}
}
child{ node{\cen}
child{ node{\tg{T2}}}
child{ node{\tdg{T3}}}
}
}
child{ node{\blue}}
;
\end{tikzpicture} =>
\begin{tikzpicture}[sibling distance=6em, level distance=2em ,level 2/.style={sibling distance=3em} ]
\node{\cen}
  child{ node{\reda}{
    child{ node{\red}}
    child{ node{\tg{T2}}}
  }}
  child{ node{\blueb}
    child{ node{\tdg{T3}}}
    child{ node{\blue}}
  }
;

\end{tikzpicture} <=
\begin{tikzpicture}[level distance=1.75em, sibling distance = 2em]
\node{\reda} 
child{ node{\red}}
child{ node{\blueb}
child{ node{\cen}
child{ node{\tg{T2}}}
child{ node{\tdg{T3}}}}
child{ node{\blue}
}
}

;
\end{tikzpicture} 

\minpurp{Bäume}
\minmeth{Preorder:} \fbox{Knoten}-linkerBaum-rechterBaum \textcolor{lightgray}{// Wurzeln sind links}

\minmeth{Inorder:} linkerBaum-\fbox{Knoten}-rechterBaum

\minmeth{Postorder:} linkerBaum-rechterBaum-\fbox{Knoten}\textcolor{lightgray}{// Wurzeln sind rechts}

\minpurp{Binärer Suchbaum}
Knoten haben 2 Kinder, kleiner im Linken, größer im rechten Teilbaum

\minmeth{Suche}
Starte bei Wurzel,rekursiv:  wenn gesucht größer: rechts, wenn kleiner links
 
\minmeth{Einfügen}
Suche im Baum; füge ein (achte auf links/rechts)
 
\minmeth{Löschen} 
Suche den Knoten; wenn:

\halfpage{
\begin{itemize}
%\item Blatt: Lösche Knoten, Referenz im Vorgänger auf null
\item 1 Nachfolger: Referenz im Vorgänger auf nachfolger, lösche Knoten
\item 2 Nachfolger: Tausche mit größtem Element im linken Teilbaum (symmetrischer Vorgänger), lösche Element
\end{itemize}
}

symmetrischer Vorgänger: einmal nach links, dann rechts solange möglich;

\minpurp{AVL-Baum}
Bedingung: |Balancefaktor| < 1 \\
Balancefaktor = $\text{Höhe}_{rechter Teilbaum} - \text{Höhe}_{linker Teilbaum} $\\
\textcolor{lightgray}{$ h < 1.45 \log_2(n+2)-1.33$ }=> höhe max. 45 \% schlechter als best-case

\minmeth{Einfügen/Löschen} wie bei binär, anschließend Ausgleichen

\submeth{Ausgleich nach einfügen}
(Umgekehrt für +2)\\
Suche balancefaktor -2 am weitesten unten im Pfad zum eingefügten element\\
betrachte linken Nachfolger (b):\\
bei -1: Rechtsrotation um linken Nachfolger (a)\\
bei +1: Doppelrotation

Maximal einmal ausgleichen nötig




\submeth{Ausgleich nach Löschen:}
(Umgekehrt für +2)\\
Suche balancefaktor -2 am weitesten unten im Pfad zum gelöschten element;\\
betrachte linken Nachfolger (a):\\
bei -1,0: Rechtsrotation um linken Nachfolger (a)\\
bei +1: Doppelrotation

wenn linker Nachfolger +1 oder -1 war, dann für höhere knoten vllt. weiter ausgleichen.




%\minpurp{probabilistische Suchbäume}
%Zufälliges einfügen: Durchschnittshöhe = $2\frac{n+1}{n}H_n-3 \approx 2\ln(n)$ =39\% mehr als der best Case

\minpurp{Treap}
jedem element wird zusätzlich eine zufällige Priorität zugewiesen\\
Heapbedingung: $Prio_{Vater} < Prio_{Kind}$~~~~
=> Treap ist eindeutig bestimmt

Erwartungswert Pfadlänge: $2\frac{n+1}{n}H_n-3$; ~~~Erwartungswert Rotationen: <2 

\minmeth{Einfügen}
Analog Binärer Baum; danach: Rotation(nach oben) bis heapbedingung erfüllt

\minmeth{Löschen}
Suche knoten, rotiere mit kleinerem Nachfolger (Priorität) bis Blatt; lösche


\renewcommand{\min}{\lceil\frac{d}{2}\rceil}
\newcommand{\mincontent}{\lfloor\frac{d-1}{2}\rfloor}
\minpurp{B-Bäume}
Entwickelt für Festplatten, minimieren zugriffe in datenbanken

Ordung d => Knoten hat $\min$ bis d Nachfolger, zwischen $\mincontent$ und d-1 Elemente,\\
Wurzel hat mind. 2 Nachfolger oder ist Blatt \\
alle Blätter sind immer auf einer Ebene => immer vollst. ausgeglichen 

Baum mit höhe h hat mindestens $1+2\dfrac{\min^h - 1}{\min -1}$ und maximal $ \dfrac{d^{h+1}-1}{d-1}$ Knoten

höhe ist $\mathbf{O(\log_2(n))}$, \textcolor{gray}{genauer: zwischen $\log_d(n+1)-1$ und $\log_{\lfloor(d-1)/2\rfloor+1}\left(\dfrac{n+1}{2}\right)$}

Aufbau eines Knotens/Seite:
\begin{tabular}{|c|c|c|c|c|}
\hline
\textit{Adresse} & Element & \textit{Adresse} & $\dots$ & \textit{Adresse}\\
\hline
\end{tabular}

Es gilt binärbaum Bedingung für jedes Element mit seiner rechte/linke Adresse

Bsp: d = 4
\begin{tikzpicture}
\tikzstyle{bplus}=[rectangle split, rectangle split horizontal,rectangle split ignore empty parts,draw]
\tikzstyle{every node}=[bplus]
\tikzstyle{level 0}=[sibling distance=60mm,level distance=2em]
\tikzstyle{level 1}=[sibling distance=60mm,level distance=3em]
\tikzstyle{level 2}=[sibling distance=20mm,level distance=3em]
\node {15 \nodepart{two} - \nodepart{three} - } [->]
  child {node {3 \nodepart{two} 7 \nodepart{three} - }
    child {node {1 \nodepart{two} 2 \nodepart{three} -}}
    child {node {4 \nodepart{two} 6 \nodepart{three} -}}
    child {node {8 \nodepart{two} 9 \nodepart{three} -}    }
  } 
  child {node {21 \nodepart{two} 28 \nodepart{three} -}
   child {node {17 \nodepart{two} 20 \nodepart{three} -}}
    child {node {22 \nodepart{two} 25 \nodepart{three} -}}
    child {node {29 \nodepart{two} 30 \nodepart{three} -} }   
    }
;\end{tikzpicture}


\minmeth{Einfügen} Suche; füge ein (sortierung beachten) \\
Wenn  das Blatt übervoll ist: (ggf. rekursiv)

\halfpage{
\begin{enumerate}
\item  Suche das mittlere Element $M_{itte}$ des übervollen Blattes, die elemente rechts davon werden neues Blatt;
\item  verschiebe die $M_{itte}$ in den Vaterknoten, der rechte Verweis zeigt auf das neue Blatt; 
\end{enumerate}   }

\minmeth{Löschen} 

\halfpage{
\begin{enumerate}
\item Element nicht in einem Blatt => tausche es mit dem Nachfolger in Sortierreihenfolge (ist in einem Blatt);  lösche;
\item ist Seite danach zu Klein ($<\mincontent$) versuche Ausgleich mit direkten Nachbarblatt:
dazwischenliegendes Element ($M_{itte})$ kommt vom Vaterknoten in den zu kleinen Knoten, der Nachfolger/Vorgänger aus dem anderen in den Vaterknoten
\item Ausgleich nicht möglich: Füge 2 benachbarte Knoten + $M_{itte}$ aus Vaterknoten zu einem Zusammen. Wiederhole ggf. rekursiv.
\end{enumerate}}

\minmeth{B*-Baum} 
Beim Einfügen in volle Seite versuche ausgleich mit direkten Nachbarn => bessere Speicherausnutzung

%\minmeth{B+-Baum}
%Bei Datenbank-Indexen: nur die Suchschlüssel im B-Baum, Blätter zeigen auf die Datenblöcke, diese sind doppelt verkettet. Der Baum selbst enthält nur Schlüsselwerte