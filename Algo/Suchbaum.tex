

\minpurp{Bäume}
Preorder: \fbox{Knoten}-linkesKind-rechtesKind

Inorder: linkesKind-\fbox{Knoten}-rechtesKind

Postorder: linkesKind-rechtesKind-\fbox{Knoten}

\minpurp{Binärer Suchbaum}
jeder Knoten hat 2 Kinder, alle im linken Teilbaum sind Kleiner, alle im rechten größer

\minmeth{Suche}
Starte in der Wurzel, wenn item größer gehe nach rechts, kleiner links, = gefunden; wiederhole bis gefunden oder Blatt 
 
\minmeth{Einfügen}
 Suche im Baum; füge ein (achte auf links/rechts)
 
\minmeth{Löschen} 
Suche den Knoten; wenn:

\halfpage{
\begin{itemize}
\item nicht vorhanden: nichts tun
\item Blatt: Lösche Knoten, Referenz im Vorgänger auf null
\item 1 Nachfolger: Referenz im Vorgänger auf nachfolger, lösche Knoten
\item 2 Nachfolger: Tausche mit größtem Element im linken Teilbaum (symmetrischer Vorgänger), lösche Element\\
\end{itemize}
}
symmetrischer Vorgänger: einmal nach links, dann rechts solange möglich;

\minpurp{AVL-Baum}
Bedingung: |Balancefaktor|= 1 \\
Balancefaktor = $\text{Höhe}_{rechter Teilbaum} - \text{Höhe}_{linker Teilbaum} $\\
$ \Rightarrow h < 1.45 \log_2(n+2)-1.33$ => höhe max. 45 \% schlechter als best-case

\minmeth{Einfügen/Löschen} wie bei binär + Ausgleichen


\minmeth{Ausgleich nach einfügen:}
(Umgekehrt für +2)\\
Suche balancefaktor -2 am weitesten unten im Pfad zum eingefügten element\\
betrachte linken Nachfolger (b):\\
bei -1: Rechtsrotation um linken Nachfolger (a)\\
bei +1: Doppelrotation

Genau einmal Ausgleichen benötigt




\minmeth{Ausgleich nach Löschen:}
(Umgekehrt für +2)\\
Suche balancefaktor -2 am weitesten unten im Pfad zum gelöschten element;\\
betrachte linken Nachfolger (a):\\
bei -1,0: Rechtsrotation um linken Nachfolger (a)\\
bei +1: Doppelrotation

wenn linker Nachfolger +1 oder -1 war, dann für höhere knoten vllt. weiter ausgleichen.


\newcommand{\red}{\textcolor{red}{T1}}
\newcommand{\blue}{\textcolor{blue}{T4}}
\newcommand{\bluedrei}{\textcolor{blue}{T3}}
\newcommand{\tg}[1]{\textcolor{green}{#1}}
\newcommand{\tdg}[1]{\textcolor{darkgreen}{#1}}
\newcommand{\cen}{\textcolor{orange}{\textbf{c}}}
\newcommand{\reda}{\textcolor{red}{a}}
\newcommand{\blueb}{\textcolor{blue}{b}}


\minpurp{Rotationen}
\begin{tikzpicture}[level distance=1.75em, sibling distance = 2em]
\node{ \blueb }
child{ node{\reda}
child{ node{\red}}
child{ node{\tg{T2}}}}
child{ node{\bluedrei}
}
;
\end{tikzpicture}
\begin{minipage}[b]{0.1\textwidth} \centering 
Rechts um \textcolor{red}{a} \\
$\Longrightarrow$ \\
Links um \textcolor{blue}{b} \\
$\Longleftarrow$
\end{minipage}
\begin{tikzpicture}[level distance=1.75em, sibling distance = 2em]
\node{\reda} 
child{ node{\red}}
child{node{\blueb}
child{ node{\tg{T2}}}
child{node{\bluedrei}}
};
\end{tikzpicture}




\minpurp{Doppelrotation}
\begin{tikzpicture}[level distance=1.75em, sibling distance = 2em]
\node{\blueb} 
child{ node{\reda}
child{ node{\red}
}
child{ node{\cen}
child{ node{\tg{T2}}}
child{ node{\tdg{T3}}}
}
}
child{ node{\blue}}
;
\end{tikzpicture} =>
\begin{tikzpicture}[sibling distance=6em, level distance=2em ,level 2/.style={sibling distance=3em} ]
\node{\cen}
  child{ node{\reda}{
    child{ node{\red}}
    child{ node{\tg{T2}}}
  }}
  child{ node{\blueb}
    child{ node{\tdg{T3}}}
    child{ node{\blue}}
  }
;

\end{tikzpicture} <=
\begin{tikzpicture}[level distance=1.75em, sibling distance = 2em]
\node{\reda} 
child{ node{\red}}
child{ node{\blueb}
child{ node{\cen}
child{ node{\tg{T2}}}
child{ node{\tdg{T3}}}}
child{ node{\blue}
}
}

;
\end{tikzpicture} 

%\minpurp{probabilistische Suchbäume}
%Zufälliges einfügen: Durchschnittshöhe = $2\frac{n+1}{n}H_n-3 \approx 2\ln(n)$ =39\% mehr als der best Case

\minpurp{Treap}
jedem element wird zusätzlich eine zufällige Priorität zugewiesen\\
Heapbedingung: das Elternelement hat kleinere Priorität als die Kinder

=> Treap ist eindeutig bestimmt

Erwartungswert Pfadlänge: $2\frac{n+1}{n}H_n-3$\\
Erwartungswert Rotationen: <2

\minmeth{Suchen \& Einfügen}
Analog zu Binärer Baum; anschließend: Rotation des neuen Knotens nach oben, bis heapbedingung erfüllt ist;
\minmeth{Löschen}
Rotation des zu löschenden Knoten mit dem Nachfolger kleinerer Priorität, bis der Knoten ein Blatt ist, dann löschen.


\renewcommand{\min}{\lceil\frac{d}{2}\rceil}
\newcommand{\mincontent}{\lfloor\frac{d-1}{2}\rfloor}
\minpurp{B-Bäume}
für festplatten, minimieren zugriffe in datenbanken

Ordung d: Knoten haben zwischen $\min$ und d Nachfolger, $leq$ $\mincontent -1$ und d-1 Elemente,\\
Wurzel hat mind. 2 Nachfolger oder ist Blatt \\
alle Blätter sind immer auf einer Ebene => immer vollst. ausgeglichen 

Baum mit höhe h hat mindestens $1+2\dfrac{\min^h - 1}{\min -1}$ und maximal $ \dfrac{d^{h+1}-1}{d-1}$ Knoten

höhe ist $O(\log_2(n)$, genauer: zwischen $\log_d(n+1)-1$ und $\log_{\lfloor(d-1)/2\rfloor+1}\left(\dfrac{n+1}{2}\right)$

Aufbau eines Knotens/Seite:
\begin{tabular}{|c|c|c|c|c|}
\hline
\textit{Adresse} & Element & \textit{Adresse} & $\dots$ & \textit{Adresse}\\
\hline
\end{tabular}

Es gilt binärbaum Bedingung jedes Element + rechte/linke Adresse

Bsp: d = 4
\begin{tikzpicture}
\tikzstyle{bplus}=[rectangle split, rectangle split horizontal,rectangle split ignore empty parts,draw]
\tikzstyle{every node}=[bplus]
\tikzstyle{level 1}=[sibling distance=60mm]
\tikzstyle{level 2}=[sibling distance=20mm]
\node {15 \nodepart{two} - \nodepart{three} - } [->]
  child {node {3 \nodepart{two} 7 \nodepart{three} - }
    child {node {1 \nodepart{two} 2 \nodepart{three} -}}
    child {node {4 \nodepart{two} 6 \nodepart{three} -}}
    child {node {8 \nodepart{two} 9 \nodepart{three} -}    }
  } 
  child {node {21 \nodepart{two} 28 \nodepart{three} -}
   child {node {17 \nodepart{two} 20 \nodepart{three} -}}
    child {node {22 \nodepart{two} 25 \nodepart{three} -}}
    child {node {29 \nodepart{two} 30 \nodepart{three} -} }   
    }
;\end{tikzpicture}


\minmeth{Einfügen:} Suche; Wenn Blatt nicht voll: füge ein (sortierung beachten) \\
Wenn Blatt voll: (ggf. rekursiv)

\halfpage{
\begin{enumerate}
\item  Suche das mittlere Element $M_{itte}$, die elemente rechts davon kommen in ein neues Blatt;
\item  füge $M_{itte}$ in den Vater Knoten ein, der rechte Verweis zeigt auf das neue Blatt; 
\item füge das neue Element ein
\end{enumerate}   }

\minmeth{Löschen} 

\halfpage{
\begin{enumerate}
\item Element nicht in einem Blatt => tausche es mit dem Nachfolger in der Sortierreihenfolge; Anschließend löschen; 
\item ist die Seite danach zu Klein ($<\mincontent$) versuche Ausgleich mit direkten Nachbarblatt:
das dazwischenliegende Element x wird aus Vaterknoten in den zu kleinen Knoten verschoben, der Nachfolger/Vorgänger aus dem anderen Knoten an seine Stelle eingefügt, sein Verweis kommt an die leere Adresse vor/nach x.
\item Ist das nicht möglich: Füge 2 benachbarte Knoten + x aus Vaterknoten zu einem Zusammen. Wiederhole ggf. rekursiv.
\end{enumerate}}

\minmeth{B*-Baum} 
Beim Einfügen in volle Seite wird wie beim löschen zuerst versucht mit direkten Nachbarn auszugleichen => bessere Speicherausnutzung, geringere Höhe

\minmeth{B+-Baum}
Bei Datenbank-Indexen: nur die Schlüssel in B-Baum, Blätter zeigen auf die Datenblöcke, diese sind doppelt verkettet. Der Baum selbst enthält nur Schlüsselwerte