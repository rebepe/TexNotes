\minpurp{Hashtabellen}
\textbf{m = Tabellenplätze, ~~~ n=Anzahl der Datensätze, ~~~B=$\frac{n}{m}$} 

\minpurp{Hashfunktionen} %(Geburtstagsparadoxon) = $\prod_{i=1}^{k-1}\left(1-\dfrac{i}{m}\right) = e^{-k(k-1)/2m}$
\minmeth{Multiplikation}
$h(s) \lfloor m \lbrace sc \rbrace \rfloor$ mit $ \lbrace sc \rbrace  = sc - \lfloor sc \rfloor$;
\textcolor{lightgray}{optimal mit c = $0.5(1+\sqrt{5})$ }

Implementieren mit Shift-Operationen für $\mathbf{m = 2^p}$, p$\leq$ wortbreite\\
=> die p vordersten Bits des unteren Worts von $s\cdot c$ \textcolor{lightgray}{(low-Register)}

\fbox{\halfpage{
Bsp: w=8, ~~p=6,~~ c=0.618=0.10011110,~~ s = 4 = 100\\
$h(s) : 10011110*100 = 10\underline{011110}00 => h(4)= 30$}}

\minmeth{Universelle Familien}\\
\textcolor{lightgray}{Zufallsfunktion= Zufällig Wert->Hash Kombination,
Kollisionswahrscheinlichkeit: 1/m}

Funktionsfamilie ist universelle Familie, wenn Kollisionswahrscheinlichkeit = $\frac{1}{m}$

Bsp: Primzahl p, Tabellengröße \\  $m \leq p$; Wähle $a,b \in \mathbb{Z}_p$\\
\framebox{$h(x) = ((ax+b)~mod~ p )~ mod~ m$}

Bsp: Primzahl p, $a=(a_1,\dots,a_r) \in \mathbb{Z}_p^r$, x als p-adische Entwicklung, $x \leq p^r-1$\\
$h_a( x_1, x_2, \dots ,x_r ) = \Sigma_{i_1}^r a_i x_i ~mod ~p$

\minpurp{Verkettung mit Überlaufbereich}
Hashfunktionen liefern Adressen im Primärbereich, Tabelleneinträge speichern zusätzlich Nachfolgeradresse im Überlaufbereich; \textcolor{lightgray}{Freie überlaufzellen zusätzliche verkettung}


Zugriffe bei Erfolg: $<1+\frac{1}{2}B$, ~~~\textcolor{lightgray}{Zugriffe bei Misserfolg: $\leq 1+ B$}

Wahrscheinlichkeit für i Kollisionen: $p_i = \left( \begin{array}{c}n \\ i\end{array} \right) \left( \dfrac{1}{m} \right)^i \left( 1- \dfrac{1}{m} \right)^{n-i} $

=> Überlaufbereich = $n-m(1-p_0)$  =  Kollisionen (m=Primärbereich) 

Freie Plätze im Mittel:$ |(n - m - \text{Überlauf}) |$


\minpurp{Offene Adressierung}
Bei Kollision wird ein anderer Platz in der Tabelle gesucht, anhand der Sondierfolge \\
alle Plätze müssen in Folge $i(s)_j$ enthalten sein).

\minmeth{Einfügen}
Suche erste freie/gelöschte Zelle in Sondierfolge, füge ein;

\minmeth{Suchen}
Durchlaufe Sondierfolge, bis gefunden oder sicher nicht in Tabelle 

\minmeth{Löschen} Suche und markiere als gelöscht

mittlere Länge Sondierfolge beim Suchen : $\dfrac{1}{B}ln\left(\dfrac{1}{1-B}\right)$
beim Einfügen: $1/1-B$

\minmeth{Lineares Sondieren}
betrachte immer den nächsten eintrag bis Erfolg; ($i(s)_j=h(s)+j~mod~m$)\\
Nachteil: Sondierfolgen verketten sich (Cluster)

mittlere Länge Sondierfolge beim Suchen : $\dfrac{1}{2}\left(\dfrac{1}{1+B}\right)$
beim Einfügen:  $\dfrac{1}{2}\left(1+\left(\dfrac{1}{1-B}\right)^2\right)$

\minmeth{Quadratisches Sondieren}
m = Prim; $\mathbf{m \equiv 3 ~mod 4}$ \\
$i(s)_j = h(s) \pm j^2~mod~m$ (also 0,+1,-1,+4,-4,...) 

\minmeth{Doppelhashing}
$i(s)_j = h(s) + jh^*(s)~ mod~ m$,  h $\neq$ h* , m = prim

