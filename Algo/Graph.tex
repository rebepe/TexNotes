\minpurp{Graphen}
im folgenden \textbf{k = Anzahl Knoten, e = Anzahl Kanten}
a adjazent zu b: es existiert die kante a->b, also $(a,b) \in E$\\
Umgebung: alle zu einem knoten adjazenten knoten

q $\leq \left( \begin{array}{c}p \\ 2\end{array} \right)$ (ungerichtet) bzw. $\leq p(p-1)$(gerichtet)

%vollständig: alle Kanten im Graph

%dicht/dünn besetzt: Anzahl der Kanten groß/klein im Vergleich zu möglichen kanten

%Teilgraph: Knoten und Kanten sind Teilmenge des Originalgraphen\\
%aufspannender Teilgraph: alle Knoten und Teilmenge der Kanten des Originalgraphen\\
%erzeugter Teilgraph: Teilmenge der Knoten und alle Kanten zwischen diesen

Pfad: Folge von Knoten $v_0,v_1,\dots,v_n$, mit Kanten von $x_i -> x_{i+1}$\\
%einfacher Pfad: jeder Knoten nur einmal im pfad\\
%geschlossen: Anfang = Ende des Pfads\\
Zyklus: geschlossener Pfad mit länge $\geq$ 3 (ungerichtet) bzw. $\geq$ 2 (gerichtet)

Zusammenhangskomponente: alle gegenseitig erreichbaren knoten bilden Komponente\\
Baum: zusammenhängend, azyklisch und e = k-1\\
Bipartiter Graph: zwei Mengen von Knoten $V_1, V_2$, alle Kanten gehen von $V_1$ nach $V_2$\\
perfekte Zuordnung: Bipartiter Graph, bijektive Abbildung von $V_1$ nach $V_2 \Leftrightarrow$ det(Adjazenzmatrix) = 0

Adjazenzmatrix: $\left(
\begin{matrix}
0 & 1& 0\\
0 & 0 & 0 \\
0 & 0 & 0 
\end{matrix}
 \right)$ => Graph mit 3 Knoten, kante von 0->1\\
Adjazenzliste: Für jeden Knoten eine Liste der adjazenten  Knoten.

\minpurp{Weitensuche}
Besuche die Nachbarn des Startknotens, dann die Nachbarn des ersten Nachbarn usw.\\
$\Leftrightarrow$ gehe den entstehenden Baum Ebenenweise durch.

$V_T: besuchte Knoten$, $V_{ad}$: zum besuch Vorgemerkte Knoten als Queue, $V_R$: Rest\\

Ablauf: int[k] where; <0: in der Queue, 0: $V_R$, >0: besucht (nummer gibt die Zusammenhangskomponente an);\\
Visit(Node k): füge k in die Queue;\\
durchlaufe die Queue, für jeden Knoten füge alle Nachbarn aus $V_R$ in die Queue

Laufzeit: bei Liste: O(e+k), bei Matrix: O($k^2$)

Erweiterungen: Test auf Zyklen $\Leftrightarrow$ Test ob Nachbar schon im Baum (und nicht parent)\\
Ermittlung des Abstands von der Wurzel des erzeugten Baums

\minpurp{Tiefensuche}
Besuche den 1.Nachbarn des Startknotens, dann den 1.Nachbarn des 1.Nachbarn usw.\\
$\Leftrightarrow$ durchlaufe die Pfade bis zum ende.

Visit: durchlaufe die Adjazenzliste, für jeden nicht besuchten nachbarn rufe Visit auf\\
Laufzeit: O(e+k)

Trägt man die Start/Ende-Zeitpunkte von Visit ein, dann ist a vorfahr von b wenn [$Start_b,End_b$] in [$Start_a,End_a$] liegt. 

Erweiterung: Test auf Azyklität $\Leftrightarrow$ Kanten auf einen Vorgänger\\
Topologisches Sortieren: Array mit Länge=|Knoten|, fülle das Array von hinten in Terminierungsreihenfolge\\
Starke Zusammenhangskomponente: 1. Nummeriere in Terminierunsreihenfolge, 2. Konstruiere den reversen Graph (Drehe alle Kanten um) 3. starte Tiefensuche bei Knoten mit höchster Terminierungsnummer, alle Erreichbaren sind starke Zusammenhangskomponente (wiederhole letzten Schritt bei bedarf)

\minpurp{Priority Queue}
Speichert Elemente mit Prioritäten, abgreifen der Elemente in aufsteigender Reihenfolge der Elemente; methoden PQUpdate, PQRemove;
Implementierung durch Heap und Positionsarray für die Elemente;\\
Wird durch UpHeap/DownHeap die Position verändert muss das Positionsarray angepasst werden;\\
\minmeth{Einfügen} füge das Element am HeapEnde ein, UpHeap \\
\minmeth{Löschen} Nimm das erste element aus dem Heap, tausche das letzte an die 1. Stelle und DownHeap
Laufzeit: O(log(n))

\minpurp{Union-Find}
dynamische Partitionierung einer Menge: speicherung in parent array; zusätzliches Array rank für Erweiterungen;\\
jede Partition wird ein Wurzelbaum => Elemente in einer Menge wenn gleiche Wurzel\\
Repräsentant ist Wurzel; Wurzeln haben parent[w] =0

\minmeth{Find(e)}
liefert den Repräsentanten (Wurzel) der Partition in der e liegt zurück\\
durchlaufe parent-Beziehung bis zur Wurzel\\
\textbf{Pfadkomprimierung:} anschließend: setze parent aller Knoten auf dem Pfad von e -> wurzel auf die gefundene Wurzel

\minmeth{Union(x,y)}
true wenn in selber Partition, sonst false + vereinigen diese Partitionen\\
Zusammenfassen von i = Find(i) und j = Find(j) durch; parent[i] = Find(j)\\
\textbf{Höhen-Balancierung:}  bei Union wird die kleinere Wurzel Kind der größeren; bei gleichheit steigt der rang der neuen Wurzel

worst-case Laufzeit für n-1 unions und m finds: O((m+n)$\alpha(n));~~~ \alpha(n) \leq 4$

\minpurp{Dijkstra/Prim (minimaler aufspannender Baum)}
Start: (\{Startknoten\}, $\emptyset$)\\
Schritt: füge die kleinste vom konstruierten Baum ausgehende Kante in den baum ein;\\
Priorität: bei Prim: Kantengewicht, bei Dijkstra Pfad zur Wurzel\\
Implementierung durch Priority Queue bei Adjazenzliste

Laufzeit: O($n^2$) bei Matrix, O((p+q)log(p)) bei liste

\minpurp{Kruskal (minimaler aufspannender Baum)}
Start T= (V,$\emptyset$) // alle Knoten, keine Kanten\\
Schritt: füge die kleinste Kante ein, die keinen Zyklus erzeugt\\
Implementierung: Knoten in Union-Find; Sortiere Kanten nach gewicht, durchlaufe die Kanten und führe für jede Kante Union($v_{Start},v_{End}$) aus; wenn false füge die Kante ein;\\
Laufzeit: O(p+qlog(q))

\minpurp{Boruvka (minimaler aufspannender Baum)}
Min-Max-Ordnung: kante ist Kleiner falls gewicht kleiner bzw. kleinster Knoten kleiner bzw. größter Knoten kleiner\\
minimale indizente Kante: kleinste Kante an einem Knoten\\
Start: Tree(V,$\emptyset$) Graph(V,E)\\
Schritt: füge alle Minimal indizenten Kanten im Baum ein, Kontrahiere sie im Graph; wiederholen
Laufzeit: O((p+q)log(p))

\minpurp{Warshall (transitiver Abschluss)}
Transitiver Abschluss von G: Graph, bei dem Kante (a,b), wenn Pfad von a nach b in G
(bei ungerichteten: Zusammenhangskomponente)


Start: $a_0$ = Adjazenzmatrix;
für k=1 bis p: Schritt: $a_k[i,j] = a_{k-1} or( a_{k-1}[i,k] and a_{k-1}[k,j] )$
für implementierung: es wird nur speicher für eine Matrix benötigt, diese wird angepasst. 3 forschleifen (k,i,j)
\textbf{O($p^3$)}

\minpurp{Floyd}
Adjazenzmatrix gewichteter Graph, gewicht= $\infty$ wenn keine Kante, 0 in Hauptdiagonale;
Start: $a_0$ = Adjazenzmatrix;
für k=1 bis p: Schritt: Schritt: $a_k[i,j] = min{a_{k-1} , a_{k-1}[i,k] + a_{k-1}[k,j] }$
funktioniert auch mit negativen gewichten, wenn keine negativen Zyklen
implementierung analog zu Warshall
