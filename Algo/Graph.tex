\minpurp{Graphen}
\textbf{k = Anzahl Knoten, e = Anzahl Kanten}

a adjazent zu b: es existiert die kante a->b, also $(a,b) \in E$\\
\submeth{Umgebung:} alle zu einem knoten adjazenten knoten

$e \leq \left( \! \begin{array}{c}k \\ 2\end{array} \! \right)$ (ungerichtet) bzw. $\leq k(k-1)$(gerichtet)

%vollständig: alle Kanten im Graph

%dicht/dünn besetzt: Anzahl der Kanten groß/klein im Vergleich zu möglichen kanten

\submeth{Teilgraph:} Knoten und Kanten sind Teilmenge des Originalgraphen\\
\submeth{aufspannender Teilgraph:} alle Knoten und Teilmenge der Kanten des Originalgraphen\\
\submeth{erzeugender Teilgraph:} Teilmenge der Knoten und alle Kanten zwischen diesen

\submeth{Pfad:} Folge von Knoten $v_0,v_1,\dots,v_n$, mit Kanten von $v_i -> v_{i+1}$\\
%einfacher Pfad: jeder Knoten nur einmal im pfad\\
%geschlossen: Anfang = Ende des Pfads\\
\submeth{Zyklus:} geschlossener Pfad mit länge $\geq$ 3 (ungerichtet) bzw. $\geq$ 2 (gerichtet)

\submeth{Zusammenhangskomponente:} alle gegenseitig erreichbaren knoten bilden Komponente\\
\submeth{Baum:} zusammenhängend, azyklisch und e = k-1\\
\submeth{Bipartiter Graph:} zwei Mengen von Knoten $V_1, V_2$, alle Kanten gehen von $V_1$ nach $V_2$
%\textcolor{lightgray}{perfekte Zuordnung: Bipartiter Graph, bijektive Abbildung von $V_1$ nach $V_2 \Leftrightarrow$ det(Adjazenzmatrix) = 0}

Adjazenzmatrix: $\left(
\begin{matrix}
0 & 1& 0\\
0 & 0 & 0 \\
1 & 0 & 0 
\end{matrix}
 \right)$ => 
\tikzset{ LabelStyle/.style = { rectangle, rounded corners, draw, minimum width = 2em, font =fseries }, VertexStyle/.append style = { inner sep=5pt, font = \Large\bfseries}, EdgeStyle/.append style = {->,bend left =20}}
\begin{minipage}{0.1\textwidth}
\begin{tikzpicture}
\tikzset{EdgeStyle/.append style = {->,bend left =20,black}}
\node(1){1}; \node(2)[above right of=1]{2}; \node(3)[below right of=2]{3};
\draw[->] (1) edge (2) (3) edge (1);
\end{tikzpicture}
\end{minipage}
Adjazenzliste: 
\begin{minipage}[c]{0.1\textwidth}
1: 2\\
2:\\
3: 1 \\
\end{minipage}.

\minpurp{Weitensuche}
Besuche die Nachbarn des Startknotens, dann die Nachbarn des ersten Nachbarn usw.\\
$\Leftrightarrow$ gehe den entstehenden Baum Ebenenweise durch.

$V_T:$ besuchte Knoten, $V_{ad}$: zum besuch Vorgemerkte Knoten als Queue, $V_R$: Rest

\textcolor{lightgray}{Ablauf: int[k] where; // <0: in der Queue, 0: $V_R$, >0: besucht }%(nummer gibt die Zusammenhangskomponente an);

Visit(Node k): füge k in die Queue;\\
durchlaufe die Queue, für jeden Knoten füge alle Nachbarn aus $V_R$ in die Queue

Laufzeit: bei Liste:\textbf{ O(e+k)}, bei Matrix: $\mathbf{O(k^2)}$

\minmeth{Erweiterung:} Test auf Zyklen $\Leftrightarrow$ Test ob Nachbar schon im Baum (und nicht parent)\\
Ermittlung des Abstands von der Wurzel des erzeugten Baums

\minpurp{Tiefensuche}
Besuche den 1.Nachbarn des Startknotens, dann den 1.Nachbarn des 1.Nachbarn usw.\\
$\Leftrightarrow$ durchlaufe einen Pfad nach dem Anderen

Visit: durchlaufe die Adjazenzliste, für jeden nicht besuchten nachbarn rufe Visit auf\\
Laufzeit: \textbf{O(e+k)}


\minmeth{Erweiterung:} 
\submeth{Test auf Azyklität} Kanten auf einen Vorgänger\\
bei Visit Start/Ende: A vorgänger von B wenn $[Start_B,End_B] \subset  [Start_A,End_A]$

\submeth{Topologisches Sortieren} Array der Länge k, fülle von hinten bei Visitende\\
\submeth{Starke Zusammenhangskomponente:}

\halfpage{
\begin{enumerate}
\item Nummeriere in Terminierunsreihenfolge
\item Drehe alle Kanten um ( Konstruiere den reversen Graph)  
\item Tiefensuche von höchster Terminierungsnummer aus, alle Erreichbaren sind starke Zusammenhangskomp. 
\item(wiederhole letzten Schritt bei bedarf)
\end{enumerate}
}

\minpurp{Priority Queue}
Speichert Elemente mit Prioritäten, entnahme des Elements mit kleinster Priorität;
 
Implementierung durch Heap und Positionsarray für die Elemente;

Wird durch UpHeap/DownHeap die Position verändert: anpassen des Positionsarrays

\minmeth{Einfügen} füge das Element am HeapEnde ein, Umgekehrtes DownHeap (UpHeap) \\
\minmeth{Löschen} Entnahme der Heapwurzel, ersetzen durch letztes Element; DownHeap

\minpurp{Union-Find}
dynamische Partitionierung; parent array; Für Erweiterungen: array rank\\
Partition als Wurzelbaum => Elemente in einer Menge wenn gleiche Wurzel\\
Repräsentant ist Wurzel; Wurzeln haben parent[w] =0

\minmeth{Find(e)}
return Wurzel der Partition von e : durchlaufe parent-Beziehung bis Wurzel\\
\submeth{Pfadkomprimierung:} setze gefundene Wurzel als parent aller Knoten auf diesem Pfad 

\minmeth{Union(x,y)}
return true wenn in selber Partition, sonst false + vereinige diese Partitionen\\
i = Find(i); j = Find(j) if(i!=j) \{ parent[i] = Find(j)\}\\
\submeth{Höhen-Balancierung} rank[i] speichert rang von i; hänge bei Union die Kleinere Wurzel unter die größere, bei gleichheit steigt der rang der neuen Wurzel

worst-case Laufzeit für n-1 unions und m finds: O((m+n)$\alpha(n));~~~ \alpha(n) \leq 4$

\minpurp{Dijkstra/Prim (minimaler aufspannender Baum)}
\submeth{Start:} (\{Startknoten\}, $\emptyset$) ~~~~// Startknoten, keine Kanten\\
\submeth{Schritt:} füge die kleinste vom konstruierten Baum ausgehende Kante in den baum ein;\\
Priorität: bei Prim: Kantengewicht, bei Dijkstra Pfad zur Wurzel\\
Implementierung durch Priority Queue bei Adjazenzliste

Laufzeit: \textbf{O($n^2$)} bei Matrix, \textbf{O((p+q)log(p))} bei liste

\minpurp{Kruskal (minimaler aufspannender Baum)}
\submeth{Start} T= (V,$\emptyset$) ~~~~// alle Knoten, keine Kanten\\
\submeth{Schritt} füge die kleinste Kante ein, die keinen Zyklus erzeugt\\
Implementierung: Knoten in Union-Find; Sortiere Kanten nach gewicht, durchlaufe die Kanten und führe für jede Kante Union($v_{Start},v_{End}$) aus; wenn false füge die Kante ein;\\
Laufzeit: O(p+qlog(q))

\minpurp{Boruvka (minimaler aufspannender Baum)}
Min-Max-Ordnung: kante ist Kleiner falls gewicht kleiner bzw. kleinerer Knoten kleiner bzw. größerer Knoten kleiner\\
minimale indizente Kante: kleinste Kante an einem Knoten\\
\submeth{Start:} Tree(V,$\emptyset$) Graph(V,E)\\
\submeth{Schritt:} füge alle Minimal indizenten Kanten im Baum ein, Kontrahiere sie im Graph; wiederholen
Laufzeit: \textbf{O((p+q)log(p))}

\minpurp{Warshall (transitiver Abschluss)}
Erzeugt Graph, bei dem jede Kante einem Pfad in der Eingabe darstellt


\submeth{Start:} $a_0$ = Adjazenzmatrix;

für k=1 bis p: 

\submeth{Schritt:} $a_k[i,j] = a_{k-1} or( a_{k-1}[i,k] and a_{k-1}[k,j] )$
für implementierung: es wird nur speicher für eine Matrix benötigt, diese wird angepasst. 3 forschleifen (k,i,j) => 
\textbf{O($p^3$)}

\minpurp{Floyd (minimaler aufspannender Baum)}
Adjazenzmatrix gewichteter Graph, gewicht= $\infty$ wenn keine Kante, 0 in Hauptdiagonale;

\submeth{Start:} $a_0$ = Adjazenzmatrix;

für k=1 bis p: 

\submeth{Schritt:} Schritt: $a_k[i,j] = min{a_{k-1} , a_{k-1}[i,k] + a_{k-1}[k,j] }$

funktioniert auch mit negativen gewichten, wenn keine negativen Zyklen, implementierung analog zu Warshall
