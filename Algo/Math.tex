\newcommand{\IN}{\sum_{i=1}^n}
\newcommand{\INzwei}{\sum_{i=2}^n}
\newcommand{\mathitem}[1]{\item$ #1 $}

\newcommand{\gn}{\textcolor{blue}{g(n)}}
\newcommand{\fn}{\textcolor{magenta}{f(n)}}

\minpurp{O-Notation}
$\gn = O(\fn) \Leftrightarrow \left(\gn\frac{1}{\fn}\right)$ ist beschraenkt (z.B. Konvergent)\\
Fester Wert => g =O(f(n)) UND f = O(g(n);

\minmeth{Logarithmen}

\halfpage{
\begin{itemize}
\item $log(xy) = log(x) + log(y)$
\item $log(x^c) = clog(x)$
\item $e^{log(x)} = x$
\end{itemize}}\\

\minpurp{Differenzengleichungen}
\quarterpage{\begin{enumerate}
\item Aus Angabe lesen:\\   \textcolor{blue}{b}: Für Eingabe=1,  ~~~~~
$ x_n = \textcolor{green}{a_n}x_{n-1} +\textcolor{red}{b_n}  $
\mathitem{\pi_n = \prod_{i=2}^n\textcolor{green}{a_i}}
\mathitem{ x_n =\pi_n\left(\textcolor{blue}{b}+ \sum_{i=2}^n\dfrac{\textcolor{red}{b_{\textbf{i}}} }{\pi_{\textbf{i}}} \right)          }
\end{enumerate}
}
\quarterpage{
Einfache Sonderfalle:          
\begin{itemize}
\mathitem{x_n= a_n x_{n-1} =~~~ b \prod_{i=2}^n a_i          }
\mathitem{x_n= x_{n-1} + b_n  = ~~ b + \sum_{i=2}^n b_i     }
\end{itemize}
}
\minmeth{Nützliche Zahlen}\\
\halfpage{
\begin{itemize}
\mathitem{ \sum_{i=1}^i \frac{1}{i} = H_n~~~~\textcolor{lightgray}{Abschatzung: ln(n+1) \leq H_n \leq ln(n)+1         } \\
\sum_{i=1}^nH_i = (n+1) H_n -n          }

\mathitem{ x_n = \textcolor{red}{\sum_{i=1}^{n-1} x_i} + sth_n \textcolor{lightgray}{\Rightarrow x_{n-1} = \sum_{i=1}^{n-2} x_i + sth_{n-1} \Rightarrow \sum_{i=1}^{n-2} x_i = x_{n-1} - sth_{n-1}} \\ \Rightarrow x_n = 2x_{n-1} +sth_n - sth_{n-1} } 

\item  $\IN c= nc $ \\
\quarterpage{ $\IN i = \frac{n(n+1)}{2}          \\
\IN (2i -1) = n^2$          }
\quarterpage{$ \IN i^2 = \frac{n(n+1)(n+2)}{6}          \\
\IN i^3 = \left( \frac{n(n+1)}{2} \right) ^2          $}

\mathitem{\IN \lfloor log_2(i) \rfloor  = (n+1) \lfloor log_2(n) \rfloor - 2\left( 2^{\lfloor log_2(n) \rfloor} -1 \right)        }

\mathitem{\sum_{i=0}^n c^i = \frac{c^{n+1}-1}{c-1}\\
\sum_{i=0}^n i c^{i-1} = \frac{(n+1)c^n(c-1)-(c^{n+1}-1)}{(c-1)^2}}
\end{itemize}
}

\minmeth{rationale Summen}\\
\halfpage{
\begin{enumerate}
\mathitem{\IN \frac{sth}{polynom}}
\mathitem{Partialbruchzerlegung: \frac{x_i}{polynom} = \frac{a}{1.NST} + \frac{b}{2.NST} \dots \\ }
Bei Mehrfachen NST $(k.NST)^{Vielfachheit}$ statt k. NST 
\item Koeffizientenvergleich;
\end{enumerate}}



