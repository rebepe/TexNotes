\subsection{Box-Model}
Margin, \fcolorbox{black}{lightgray}{\textbf{Border}\fcolorbox{black}{white}{ Padding, \textcolor{lightgray}{\fbox{\textcolor{black}{Content(height,width)}}}}}
\section{CSS}
\begin{verbatim}
Selector [, Selector2, ..., SelectorN]
{
  Property1: Value1; /* Comment */
  ...
  border: 5px solid red;
  visibility: hidden;
}
\end{verbatim}
\subsection{Selectors}
\halfpage{
\begin{itemize}
\item *: alle elemente
\item p: alle <p> elemente
\item h1, p: alle <h1> und alle <p> elemente
\item div p: alle <p> elemente in <div> elementen
\item .example: alle elemente mit class=$"$ example$"$
\item \#id: alle elemente mit id=$"$id$"$
\item div > p: alle <p> elemente mit <div> als parent
\item div + p: <p> elemente, die direkt auf <div> elemente folgen
\item h1\textasciitilde p: alle <p> elemente die auf <h1> elemente folgen 
\item $\left[ \text{href} \right]$ : elemente mit href-Attribut
\item $\left[ \text{ href=/url.tst } \right]$ : elemente mit href="/url.tst"
\end{itemize}}
\subsection{pseudo-tags}
\halfpage{
\begin{itemize}
\item :active geklickt
\item :hover  mausschwebend
\item :visited besuchter link
\end{itemize}

browser-präfixe: -ms-, -moz-, -webkit-
}
\subsection{Media-Query}

\begin{verbatim}
@media not|only mediatype and (expressions) {  CSS-Code;}

z.B.
@media screen and (min-width: 480px) {
  body {
    background-color: lightgreen;
  }
}
\end{verbatim}


