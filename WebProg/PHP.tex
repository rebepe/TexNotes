\newpage
\subsection{PHP}
Wird auf Server ausgeführt, ausgabe wird versendet => Browser weiss davon nichts.

Einbetten in HTML: <?php ... ?>
\subsubsection{Variablen:}
\$myInt = 10;

isset(\$myInt); //True wenn definiert

unset(\$myInt); //Löschen

\textbf{Superglobale variablen}
Immer direkt aufrufbar 

\halfpage{ \begin{itemize}
\item \$GLOBALS - alle Variablen im Globalen scope
\item \$\_SERVER - Informationen über Server und ausführungsumgebung
\item \$\_GET - HTTP GET variablen
\item \$\_POST - HTTP POST variablen
\item \$\_FILES - HTTP datei upload variablen
\item \$\_COOKIE - HTTP cookies
\item \$\_SESSION - Session variablen
\item \$\_REQUEST - HTTP request variablen
\item \$\_ENV - Umgebungs variablen
\end{itemize}} \subsubsectionnb{Referenzen}
\begin{verbatim}
$a =& $b; // $a and $b zeigen auf gleiches objekt
function foo(&$var) {  //übergabe einer Referenz
   $var++;
}
class Foo {
   public $value = 42;
   public function &getValue() {
      return $this->value;
   }
}
\end{verbatim}
\subsubsection{Strings}
Kein Unicode! verketten mit .

\$helloStr = 'Hello Nr. \$myInt';  // Nr. \$myInt\\
\$helloStr2 = "Hello Nr. \$myInt";  // Nr. 10\\
\$helloStr3 = "Hello Nr. \{\$myInt\}eger // Nr. 10eger \newpage
\subsubsection{Arrays}
key-value!\begin{verbatim}
$a1 = array("a", "b", "c");
$a2 = array( "name" => "hans",
   "alter" => 5,
   10 => 25,
   100 => 78)

echo $a1[1];
echo $a2["foo"];
echo $a2[100];
\end{verbatim}

count(\$a2); // höchster integer arraykey (in diesem Fall 100)
\subsubsectionnb{objekte}
\begin{verbatim}
class Foo {
  public $var = 'default value';
  protected $bar;
  public function __construct() { // Konstruktor
    $this->bar = 42;
  }
  public function displayVar() {  echo $this->var;  }
}
$f = new Foo();
$f->displayVar();  // methodenaufruf
//----vererbung
class Bar extends Foo {
  public function __construct() {
    parent::__construct();    // Konstruktor vaterklasse
    $this->bar = 24;
  }
}
$b = new Bar();
$b->displayVar();  //Memberaufruf
\end{verbatim}\newpage

\subsubsection{Kontrollstrukturen}
Einbinden: include, require, include\_once, require\_once (require löst fehler aus)\\
if, elseif, else\\
while, do-while,for analog zu C
\begin{verbatim}
foreach ($a as $value) {
	//bzw.
foreach ($a as $key => $value) {
   echo $key . PHP_EOL;   
   echo $value . " == " . $a[$key] . PHP_EOL;
\end{verbatim}
\subsubsectionnb{funktionen}
\begin{verbatim}
$gvar = 'sth';      // globale variable
function foo($arg1, $arg2, ..., $argN) {
   global $gvar;    // aufruf globale Variable
   return $gvar;
}
foo();           // aufruf
$bar = 'foo';    // dynamischer aufruf
$bar();
$bar = function () { ... }; // anonyme funktion
\end{verbatim} \newpage 
.
\newpage
\section{CGI}
Aufruf von Kommandozeilenbefehlen durch Web-Server: 

server setzt umgebungsvariablen (z.B. QUERY\_ STRING, REQUEST\_ URI, REQUEST\_ METHOD, HTTP\_ ACCEPT, ...) \\ und ruft ein Skript auf( mit PUT/POST als stdin), stdout wird als antwort versendet

\textbf{Nachteile: }Nach Angriff u.U. beliebiger Code direkt auf Betriebssystem einschleusbar, Erzeugt Prozess für jeden Aufruf
\subsection{Verbesserungen}
Compilierte Anwendungen statt Skripte

FastCGI: Länger Laufender Interpreter führt mehrere Anfragen nacheinenader aus \\=> weniger Prozesse

HTTP-Server modules: Interpreter als Modul des Webservers
\section{http}
generisch, plain text, zustandslos

\fbox{
\halfpage{
request-line oder status-line

*(message-header + Leerzeile) //optional

Leerzeile (CRLF)

Nachrichtenteil

request-line : METHOD URL HTTP/1.1\\
response line : HTTP/1.1 STATUS-CODE STATUS-MESSAGE
}}
\textbf{Status-code:}

\halfpage{
\begin{itemize}
\item 1XX - Informational (request received, continues processing)
\item 2XX - Success
\item 3XX - Redirection (further action required in order to fulfill the request)
\item 4XX - Client error (request contains bad syntax or cannot be fulfilled)
\item 5XX - Server error (server failed to fulfill an otherwise valid request)
\end{itemize}}
\subsection{Sessions}
Http ist zustandslos, Sitzungen durch austausch von Session-Identifiern bei Login. \\
SessionID muss eindeutig sein. \\
Client schickt SessionID bei jeder Anfrage, z.B. per Cookie, URL

Cookie: key-value-paare als string, JavaScript-zugriff möglich
\subsection{https}
TLS = Transport Layer Security zwischen TCP und http => schutz vor Man-in-the-middle, authentifikation des Servers und Verschlüsselung der Verbindung

Attacken: wenn URI erratbar: Known-plaintext attacken,  Traffic-analysen \newpage
\section{Web-Service}
Interaktion von Rechnern über das Netzwerk per Remote Procedure Call

\textbf{RESTful:} Verwendet Standart-Http Methoden(GET, PUT, POST, DELETE, ...) \& URI
\section{Sicherheit}
\halfpage{
\begin{itemize}
\item Injection: interpretation von Daten durch Compiler => Einschleusen von Befehlen
\item Session Management: Abgreifen der Session ID => Session Hijacking
\item Cross Site Scripting: Ausführen von Angriffscode durch Browser, \\
z.B. aus Datenbank, reflexion aus Input\\
(Cross Site: Angriff zwischen 2 Aufrufen der Seite)
\item insecure direct object references/missing function level access control: privilegierte Objekte/Funktionen werden nur verborgen (presentation layer access control), nicht über authentifizierung geprüft
\item (server) misconfiguration 
\item Sensitive Data Exposure: Leak z.B. in Logs
\item Cross Site Request Forgery: Angreifer bringt Browser dazu, Anfrage an Funktion zu stellen (z.B. Bank);\\ wenn User eingeloggt wird Session ID mit übertragen => Profit\\
Gegenmaßnahme: Erzeugen und mitschicken von starken Tokens bei kritischen anfragen
\item Vulnerable Components: fehler in libraries/modulen
\item Unvalidated Redirects: z.B auf Phising-Sites oder um Überprüfungen herum
\end{itemize}}