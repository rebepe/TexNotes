\section{JavaScript}
es gibt nur referenz-Typen \\
undefined, null, infinity,NaN\\
$\left[1,2,3\right]$ // Array initializer 



\textbf{Vergleiche} mit =~\!=~\!= oder !~\!=~\!= \\
Objektvergleiche vergleichen die Referenz,\\ 
Primitive Typen(z.B. Strings) werden Wertverglichen

\textbf{Strikter Modus:} $"$use strict$"$; // z.B. für Vergleiche ohne Konvertierung 


\textbf{Foreach schleife:} for (var p in o)\{ ... \}
\subsection{funktionen}
\textbf{hoisting:} variablendeklarationen werden an Funktionsanfang gezogen

\textbf{scope}: global scope and function scope, kein Block-level-scope\\ 
Zugriff auf aufrufkontext: this  \\
innere Funktionen können auf umgebende Properties Zugreifen (außer this); \\
=> closure: funktionen merken sich umgebungsvariablen in einer scope chain
z.B. 
\begin{verbatim}
function outer () {
   var x = 2;
   return function (){ return 2*x;};
}
outer()() // gibt 4 zurück;
\end{verbatim}

\textbf{indirekte funktionsaufrufe:} apply(context, [args]) bzw. call(context, arg1, arg2, ...)
\subsection{ Objektorientierung (Prototypen)}
\textbf{Objekte} sind sammlungen von Properties (name / value pairs) \\
\{x:1, y:2\} // Object initializer 

Konstruktoraufruf: var x = new MyClass();

MyVec.prototype.addTo = function (v) \{ //sth; \} 
\begin{verbatim}
--Vererben:
function Kind(x) {
   Eltern.apply(this, x);   }// Elternkonstruktor
--Kapselung: 
function klasse(x) {		// Constructor
   var _x = x;
   this.getX = function() { return _x; };    }
\end{verbatim}


\newpage

\subsubsection{DOM}
Erlaubt Manipulation des Dokumentes zur laufzeit;\\
Darstellung der Elemente durch Nodes (Inhalt in TextNode)\\
Nodes sind in Baumstruktur gegliedert, incl. Geschwisterliste

\textbf{zugriff}:
getElementById (): liefert HTML-Element\\
getElementsByClassName(), getElementsByName(): liefern liste von HTML-Elementen

\textbf{Attribute auf HTML-Elementen}:\\
innerHTML, setAttribute(name, value), style.property;\\
nodeName (Tag), nodeValue (Text in TextNodes)

\textbf{Events}: \\
Im HTML: <h1 onclick="this.innerHTML='Foo!'">Click me!</h1>\\
Im JS: element.addEventListener(event, function);    (event ist ohne  $"$on$"$)

Event-weitergabe (Defaultverhalten) stoppen:  
event.preventDefault(); event.stopPropagation();

\textbf{Baum-navigation: }
parentNode, childNodes[idx], firstChild,lastChild,
nextSibling, previousSibling

\textbf{Elemente Einfügen }(Bsp)
\begin{verbatim}
var para = document.createElement("p");
var node = document.createTextNode("This is new.");
para.appendChild(node);
var element = document.getElementById("div1");
element.appendChild(para); // analog dazu insertBefore(), ...
                           // ... replaceChild(), removeChild()
\end{verbatim}

\textbf{URL codieren:} encodeURIComponent(\textit{url});
\section{AJAX}
Serverkommunikation ohne neuladen der Seite
\begin{verbatim}
var req = new XMLHttpRequest();
var response = null;
req.open("GET", URL, true);
req.onreadystatechange = function () {
  if (req.readyState == 4) {
    response = req.responseText; // responseXML when transferring XML
    // do something with the response
  }
};
req.send(null)
\end{verbatim}

\newpage
\subsubsection{BOM}
Interaktion mit dem Browser,kein standard aber in allen Browsern ähnlich 
window ist der Tab, bei browsern = globales Objekt, jeder Tab eigener Interpreter

\textbf{Objekte}

\halfpage{
\begin{itemize}
\item document das HTML document als DOM object
\item screen Bildschirminformationen (width, height, ...)
\item location aktuelle URL, ermöglicht redirects
Properties: 
\begin{itemize}
\item hash: teil nach \#
\item hostname
\item href: ganze URL
\item pathname:
\item port
\end{itemize}
\item history eingeschränkter zugriff auf Browserverlauf
\item navigator Browserinformationen  \\
nicht standardisiert, weit unterstützt (appName, geolocation ...);
\end{itemize}}

\textbf{Methoden}

\halfpage{\begin{itemize}
\item alert(msg): Popup mit Text und OK-Button
\item atob()/btoa():Base-64 encoding / decoding von strings
\item open(url)  Öffnet URL in neuem Fenster (Popup) 
\item close() Schließt das Fenster
\item setInterval(function,milliseconds,parameters) / clearInterval()\\
Ruft funktion wiederholt in Intervall auf.
\item setTimeout() / clearTimeout() Ruft funktion nach Ablauf von zeit auf
\end{itemize}}
\subsubsectionnb{Local Storage}
\begin{verbatim}
if (typeof(Storage) !== "undefined") {
  // Store
  localStorage.setItem("name", "Smith"); 
  // Retrieve
  document.getElementById("result").innerHTML 
                   = localStorage.getItem("name");
} 
\end{verbatim}\newpage
\subsubsectionnb{Drag\& Drop}
\begin{verbatim}
<script>
function allowDrop(ev) {
  ev.preventDefault();
}
function drag(ev) {
  ev.dataTransfer.setData("text", ev.target.id);
}
function drop(ev) {
  ev.preventDefault();
  var data = ev.dataTransfer.getData("text");
  ev.target.appendChild(document.getElementById(data));
}
</script>
...
<div id="div1" ondrop="drop(event)" ondragover="allowDrop(event)"/>
<p id="drag1" draggable="true" ondragstart="drag(event)"/>
\end{verbatim}

