\section*{JavaScript}
[1,2,3] // Array initializer \\
\{x:1, y:2\} // Object initializer \\
undefined, null, infinity,NaN sind besondere Zustände

vergleiche mit =~\!=~\!= oder !~\!=~\!= \\
\underline{hoisting:} variablendeklarationen werden an Funktionsanfang gezogen

" use strict"; // Aktiviert Striken Modus, z.B. für Vergleiche ohne Konvertierung \\
es gibt nur referenz-Typen \\
Objektvergleiche vergleichen die Referenz, Primitive Typen(z.B. Strings) werden Wertverglichen

Objekte sind sammlungen von Properties (name / value pairs) [x:1, y:2]

Foreach schleife: for (var p in o){ //sth. }

global scope and function scope, kein Block-scope\\ 
Zugriff auf aufrufkontext: this  \\
innere Funktionen können auf umgebende Properties Zugreifen (außer this); \\
closure: funktionen merken sich umgebungsvariablen in einer scope chain
z.B. 
\begin{verbatim}
function outer () {
   var x = 2;
   return function (){ return 2*x;};
}
outer()() // gibt 4 zurück;
\end{verbatim}

indirekte funktionsaufrufe: apply(context, [args]) bzw. call(context, arg1, arg2, ...)
 
 Objektorientierung: durch Prototypen \\
 MyVec.prototype.addTo = function (v) \{ //sth; \}\\
 Konstruktor: var x = new MyClass();
 
\begin{verbatim}
Vererben:
 	function Kind(x) {
 		Eltern.apply(this, x);
 }
 
Kapselung: 
function klasse(x) {		// Constructor
   var _x = x;
   this.getX = function() { return _x; };
}

\end{verbatim}


\subsection*{APIs}
APIs: Javascript core(Object,Array...), Browser Object Model(location,XMLHttpRequest,...), DOM(document,...)

\subsubsection*{BOM}
Interaktion mit dem Browser,kein standard aber in allen Browsern ähnlich 

window ist der Tab, bei browsern = globales Objekt, jeder Tab eigener Interpreter

Objekte
\begin{itemize}
\item document Represents the HTML document as DOM object
\item screen Contains information about the users screen 
(width, height, ...)
\item location Contains the current pages address (URL) and can be used to redirect the browser to a new page
\item history Very restricted access to the browser history
\item navigator Contains information about the browser
\item innerWidth, innerHeight Width / height of the content area
\item outerWidth, outerHeight Width / height including scrollbars, etc.

\end{itemize}

Methoden
\begin{itemize}
\item alert(msg): Displays alert-box with a message and an OK button
\item atob()/btoa():Base-64 encoding / decoding of strings
\item open(url)  Opens URL in a new window (Popup!) 
\item close() Closes the current window
\item setInterval(function,milliseconds,parameters) / clearInterval()
Calls a function (or evaluates an expression) at specific time intervals 
\item setTimeout() / clearTimeout() Calls a function (or evaluates an expression)after a specified number of milliseconds
\end{itemize}

Navigator: nicht standardisiert, weit unterstützt (appName, geolocation ...);

location: Bearbeiten der URL; 
Properties: 
\begin{itemize}
\item hash: teil nach \#
\item hostname
\item href: ganze URL
\item pathname:
\item port
\end{itemize}

\subsubsection*{DOM}
Erlaubt Manipulation des Dokumentes zur laufzeit;\\
Darstellung der Elemente durch Nodes (Inhalt in TextNode)\\
Nodes sind in Baumstruktur gegliedert, incl. Geschwisterliste

zugriff:
getElementById (): liefert HTML-Element\\
getElementsByClassName(), getElementsByName(): liefern liste von HTML-Elementen

Attribute auf HTML-Elementen:\\
innerHTML, setAttribute(name, value), style.property;\\
nodeName /*Tag*/, nodeValue /*Text in TextNodes*/

Events: \\
Im HTML: <h1 onclick="this.innerHTML='Foo!'">Click me!</h1>\\
Im JS: element.addEventListener(event, function);    (event ist ohne " on")

Event-weitergabe (Defaultverhalten) stoppen:  
event.preventDefault(); event.stopPropagation();

Baum-navigation: parentNode, childNodes[idx], firstChild,lastChild,
nextSibling, previousSibling

Elemente Einfügen
\begin{verbatim}
var para = document.createElement("p");
var node = document.createTextNode("This is new.");
para.appendChild(node);
var element = document.getElementById("div1");
element.appendChild(para); // analog dazu insertBefore(), replaceChild(), removeChild()
\end{verbatim}


\subsubsection*{Examples}
Drag\& Drop
\begin{verbatim}
<script>
function allowDrop(ev) {
	ev.preventDefault();
}
function drag(ev) {
	ev.dataTransfer.setData("text", ev.target.id);
}
function drop(ev) {
	ev.preventDefault();
	var data = ev.dataTransfer.getData("text");
	ev.target.appendChild(document.getElementById(data));
}
</script>
...
<div id="div1" ondrop="drop(event)" ondragover="allowDrop(event)"></div>
<p id="drag1" draggable="true" ondragstart="drag(event)"/>
\end{verbatim}
Local Storage
\begin{verbatim}
if (typeof(Storage) !== "undefined") {
// Store
localStorage.setItem("lastname", "Smith");
// Retrieve
document.getElementById("result").innerHTML = localStorage.getItem("lastname");
} else {
document.getElementById("result").innerHTML = "Sorry, your browser does not support Web Storage...";
}
\end{verbatim}