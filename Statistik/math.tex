absolute Häufigkeit: $h_i$

relative Häufigkeit: $f_i= \frac{h_i}{n}$

kumulative Häufigkeitsverteilung: H(x) =$\sum_{i:a_i < x} h_i$

empirische Verteilfunktion: $F(x) =  \frac{1}{n} H(x) = \sum_{i:a_i < x} f_i$

arithmetisches Mittel: $\overline{x} = \frac{1}{n} \sum_{i=1}^n x_i = \frac{1}{n} \sum_{i=1}^n h_i a_i $

geometrisches Mittel $\overline{x}_{geom} = \sqrt[n]{\prod_{i=1}^n x_i}$

$\overline{x}_{geom} \leq \overline{x}$

median:  $\tilde{x} = 50\% der Werte \geq \tilde{x} $


Modus = häufigster Wert

empirische Varianz: $s^2 = \frac{1}{n-1}\sum_{i=1}^n(x_i-\overline{x})^2 = \frac{1}{n-1}\left(\sum_{i=1}^n x_i^2-n\overline{x}^2\right)$

Standardabweichung:$s = \sqrt{varianz} $

Kovarianz: $s_{xy} = \frac{1}{n-1}\sum_{i=1}^n(x_i-\overline{x})(y_i-\overline{y})$

Korrelationskoeffizient: $r_xy = \frac{s_{xy}}{s_x\cdot s_y}$

\section*{Bedingte Wahrscheinlichkeiten}

prob(A) = $\frac{\text{|günstige Fälle|}}{\text{| alle Fälle|}}$

$prob(B|A) = \frac{prob(B \cap A)}{prob(A)}$

Unabhängig $\Leftrightarrow prob(A \cap B)=prob(A)\cdot prob(B)$

Satz von Bayes: $prob(B|A) = prob(A|B)\frac{prob(B)}{prob(A)}$

$prob(B|A) = prob(A|B)\dfrac{prob(B)}{prob(B)\cdot prob(A|B) + prob(\overline{B} \cdot prob(A|\overline{B}) }$


\section*{Erwartungswert}
diskret: $ E(X) = \sum_i prob(X=x_i)\cdot x_i$\\
stetig: $E(X) = \int_{-\infty}^{\infty}xf(x)dx$\\

E(X+Y) = E(X) + E(Y); E(aX) = aE(X) \\
wenn f symmetrisch um c dann E(X) = c\\
wenn X,Y unabhängig: E(XY) = E(X) $\cdot$ E(Y)

\section*{Kovarianz}
$Cov(X,Y) = E((X-E(X))\cdot(Y-E(Y)) = E(X \cdot Y) -E(X)E(Y)$

\section*{varianz}
diskret: $ \sigma^2 = \sum_i prob(X=x_i)\cdot (x_i -E(X))^2$\\
stetig: $ \sigma^2 = \int_{-\infty}^{\infty} (x_i -E(X))^2 f(x) dx$ \\
$\sigma^2 = E(X^2) - E(X)^2$

Var(X+Y) = Var(X) + Var(Y) + 2Cov(X,Y)\\
Var(aX+b) = $a^2$Var(X)


\section*{Standardisierung}
standartisierte Zufallsvariable $Z:=\dfrac{X-\mu}{\sigma}$ => E(X)=0, Var(X)=1


\section*{Verteilunsfunktionen}
Verteilungsfunktion F(x) $= prob(X \leq x)$\\
steigt monoton von 0 nach 1

Dichte: f(x);

$f(x) = F'(x); \\
 F(x) = \int_{-\infty}^xf(t)dt$

\subsection*{Gleichverteilung}
$f(x) = \frac{1}{b-a} falls a < x < b, sonst 0$

$F(x)  = \frac{x-a}{b-a} falls a < x < b, sonst 0 bzw. 1$

E(X) = (b+a)/2

\subsection*{Exponentialverteilung}
$F(x) = 1 - e^{-\lambda x};  \\ f(x) = \lambda e^{-\lambda x}$ für x > 0 \\
E(X)=$1/\lambda$:= Durchschnittliche Lebensdauer\\
Var(X) = $1/\lambda^2$
\subsubsection*{Weibull-Verteilung}
$F(x) = 1 - e^{\lambda x^\beta};  \\f(x) = \lambda \beta x^{\beta -1} e^{-\lambda x^\beta}$ für x > 0 


\subsection*{Hypergeometrische Verteilung}
Bei $ 20 n \leq N $ \textbf{Näherung} durch Binomialverteilung

Stichprobe ohne Zurücklegen: N Elemente, M Treffermöglichkeiten, Stichprobe mit n

prob(X=x)=$ \dfrac{\left( \! \begin{array}{c}M \\ x \end{array} \! \right) \cdot \left( \begin{array}{c}N-M \\ n-x \end{array}  \right) }{\left( \begin{array}{c}N \\ n \end{array}  \right) }$

$E(X)=n\dfrac{M}{N}
Var(X) = n\dfrac{M}{N} \left( 1- \dfrac{M}{N} \right) \dfrac{N-n}{N-1}
$

\subsection*{binomialverteilung}
n= Stichprobengröße, wahrscheinlichkeit p 

$prob(X=x) = \left( \begin{array}{c} n \\ x \end{array} \right) p^x (1-p)^{n-x}$

E(X) = np; Var(X) = np(1-p)

Wenn X = Bi(n;p) und Y = Bi(m;p) unabhängig, dann X + Y = Bi(m+n;p)

Bei $n \geq 50, p \leq 0.1 $ \textbf{Näherung} durch Poisson-Verteilung mit $\lambda = np$
Bei $np(1-p)\geq 9$: $F_B(x) $\textbf{Näherung}d$ F_N(x+0.5) = \Phi \left( \dfrac{x+0.5 -np}{\sqrt{np(1-p)}}\right)$


\subsection*{Poisson-Verteilung}
Auftreten von Ereignis in Zeitinterval:

$prob(X=x) = \dfrac{\lambda^x}{x!}e^{-\lambda}$

$E(X) = Var(X) = \lambda$

Bei $\lambda \geq 9$ $F_P(x) $ \textbf{Näherung} $ F_N(x+0.5)=\Phi \left( \dfrac{x+0.5-\lambda}{\sqrt{\lambda}}\right)$


\subsection*{Normalverteilung}
$f(x) = \frac{1}{\sqrt{2\pi}\sigma} \cdot e^{-0,5\left(\frac{x-\mu}{\sigma}\right)^2}$

$\mu = erwartungswert,\sigma^2 = Varianz$

$X = N(\mu;\sigma^2), Y = aX + b \Rightarrow Y = N(a\mu + b; a^2\sigma^2)$

\subsubsection*{Standard-Normalverteilung $z_p$}
Dichte: $\phi(x) = \frac{1}{\sqrt{2\pi}} \cdot e^{- 0,5 x^2}$

Verteilung: $\Phi$

$\Phi(-x) = 1-\Phi(x)$

E(X) = 0; Var(X) = 1;


Ablesen an Standard-Normalverteilung: $F(X) = \Phi \left(\dfrac{x-\mu}{\sigma} \right)$, $f(x) = \dfrac{\phi}{\sigma} \left(\dfrac{x-\mu}{\sigma} \right)$

\subsection*{Zentraler Grenzwertsatz, schwache Konvergenz}
Die Summe $\Sigma_{i=1}^nX_i$ über identisch Verteilte Zufallsvariablen $X_i mit E(X_i) = \mu, Var(X_i) = \sigma^2$ konvergiert gegen $N(n\mu,n\sigma^2)$






\subsection*{Chi-Quadrat (mit m Freiheitsgraden)}
$ \chi^2(m) = \Sigma_{i=0}^m X_i^2$ für $X_i$ standardnormalverteilt, unabhängig \\
E($\chi^2)$ = m, Var($\chi^2)$ = 2m

z.B. $\dfrac{(n-1)s^2}{\sigma^2} = \Sigma_{i=1}^n \left(\dfrac{X_i - \overline{X}}{\sigma}\right)^2$
mit $X_i$ Stichproben aus normalverteilt ist $\chi^2$ verteilt mit n-1 Freheitsgraden

$f_m(x) = \dfrac{x^{m/2-1}}{2^{m/2}\Gamma(m/2)}$ für x>0, sonst 0

$\Gamma (x) = \int_0 ^\infty t^{x-1} e^{-1}dt \Leftrightarrow$  x! für $x \in \mathbb{N}$ \\
\textbf{Näherung} durch Normalverteilung für m>30: $\chi^2_{m;p} = \sqrt{2m}z_p +m$

\subsection*{t-Verteilung  mit m-Freiheitsgraden}
$t(m) = \dfrac{Z}{\sqrt{X/m}}$ ist t-Verteilt für Z standardnormalverteilt und X Chi-Quadratverteilt

E(T) = 0 für m>1, Var(T) = $\dfrac{m}{m-2}$ für m>2

z.B. $\dfrac{\overline{X} -\mu }{S/\sqrt{n}}$ für Stichprobe mit Größe n aus Normalverteilter Grundgesamtheit

\textbf{Näherung} für m>30: $t_{m;p} = \approx \sqrt{\dfrac{m}{m-2}} \Phi(p)$

$f_m(x) = \dfrac{\Gamma\left(\dfrac{m+1}{2}\right)}{\sqrt{n\pi}\Gamma( m/2} \left( 1+ \dfrac{x^2}{m}\right)^{-\dfrac{m+1}{2}}$

\subsection*{F-Verteilung}
$F(m_1;m_2) = \dfrac{\sqrt{X_1/m_1}}{\sqrt{X_2/m_2}}$ für $X_1 und X_2$ Chi-Quadrat verteilt

$E(F) = \dfrac{m_2}{m_2-2}$ für$ m_2 > 2$, $Var(X) = \dfrac{m_2^2(m_1+m_2-2)}{m_1(m_2-4)(m_2-2)^2} $für$ m_2 >4$

\section*{Zufallsstichproben}
n gewählte Elemente, die Werte sind zufällig verteilt; wenn ausgreichend Große Grundgesamtheit: Werte unabhängig und gleich verteilt

\subsection*{Punktschätzer}
Test einer Verteilung mit zu schätzendem Parameter $\theta$

Eigenschaften:
\begin{enumerate}
\item erwartungstreu falls E(T) = $\theta$;~~~~ bias=$E(T)-\theta$
\item asymptotisch erwartungstreu: $\lim_{n\rightarrow \infty} E(T_n) = \theta$
\item konsistent: konvergiert stochastisch gegen $\theta$ \\
( $\lim_{n\rightarrow \infty} prob (|T_n - \theta| <\epsilon ) = 1$ für alle $\epsilon >0 $
\item konsistent im quadratischen Mittel: $\lim_{n\rightarrow \infty} E((T_n-\theta)^2) = 0$\\
bzw. wenn asymptotisch erwartungstreu und Var(X)$\rightarrow$ 0 \\
=> ist auch konsistent; Bsp: arithmetisches Mittel, empirische Verteilung
\end{enumerate}

\subsubsection*{Maximum-Likelihood}
ist asymptotisch erwartungstreu, asymptotisch normalverteilt mit $\mu = \theta$ und minimaler Varianz, 
\begin{enumerate}
\item $L = \prod_{i=1}^n f(x_i,\theta)  ~~~~~~~~z.B. L(x_1,\dots,x_n,\lambda) = \prod_{i=1}^n \lambda e^ {-\lambda x_i}$
\item berechne log(L) 	~~~~~~~~$	z.B. log(L(\dots)) = n ln(\lambda) -\lambda \sum_i=1^n x_i $
\item berechne maximum von L \textbf{(ABLEITEN)} ~~~~ z.B.  => $0 = n/\lambda - \sum_i=1^n x_i \dots$
\end{enumerate}

Logarithmen:
\begin{itemize}
\item $log(xy) = log(x) + log(y)$
\item $log(x^c) = clog(x)$
\item $e^{log(x)} = x$
\end{itemize}

\subsubsection*{Regressionsrechnung:}
Zu nähernde Funktion f(x) in Parameterform, z.B. Gerade: y= mx+t\\
Stichprobe mit Wertepaaren: $(x_1,y_1)\dots (x_n,y_n)$

\begin{enumerate}
\item Bilde s$\Delta : = \sum_{i=1}^n(y_i - f(x_i))^2$\\
\item Leite nach jedem Parameter ab, setze gleich Null, bestimme Parameter
\end{enumerate}

Alternative für Geraden: $m= r_xy\frac{x_y}{s_x}, d = \overline{y}-k\overline{x}$

\subsection*{Intervallschätzung}
Irrtumswahrscheinlichkeit: $\alpha$ 
Konfidenzniveau: $1-\alpha$\\
einseitiges Konfidenzintervall: $[-\infty; \overline{x} + Abweichung]$\\
zweiseitiges Konfidenzintervall: Intervall zwischen $\overline{x} \pm Abweichung$ ~~~~ //siehe Folgende

\subsubsection*{unbekannte Varianz}
Abweichung(einseitig): $\dfrac{s}{\sqrt{n}} t_{n-1;~ 1-\alpha}$\\
Abweichung(zweiseitig): $\dfrac{s}{\sqrt{n}} t_{n-1;~ 1-\alpha/2}$

\subsubsection*{Normalverteilung mit bekannter Varianz}
Abweichung(einseitig): $\dfrac{\sigma}{\sqrt{n}} z_{1-\alpha}$\\
Abweichung(zweiseitig): $\dfrac{\sigma}{\sqrt{n}} z_{1-\alpha/2}$


\section*{t-Test}
Stichprobe mit mittel $\overline{x}$; Hypothese $H_0: \mu = \mu_0$
Prüfwert: $z= \dfrac{\overline{x} - \mu_0}{s/\sqrt{n}}$

\subsection*{zweiseitig}
Ablehnungsbereich: $|z| > t_{n-1;1-\alpha/2} => H_0 $muss verworfen werden 
\subsection*{einseitig}
Ablehnungsbereich: $|z| > t_{n-1;1-\alpha} => H_0 $muss verworfen werden 


\section*{Chi-Quadrat-Anpassungstest}
Test auf Verteilung = vermutete Verteilung, Vorraussetzung: große Stichprobe ($np_i \geq 5$) für alle i

Teile Werte in Intervalle $I_i$ auf  pi: prob($X \in I_i$) und $h_i:$ Anzahl der Werte in $I_i$

$p_i = prob(A_i)$ : Wahrscheinlichkeit von $A_i$ gemäß vermuteter Verteilung

$y = \sum_{i=1}^k \dfrac{(h_i -np_i)^2}{n p_i} = 1/n \left( \sum_{i_1}^k \dfrac{h_i^2}{p_i} \right) -n $
ist asymptotisch  $\chi^2 (k-1)$ verteilt 

Ablehnungsbereich: y > $\chi^2_{k-1;1-\alpha}$