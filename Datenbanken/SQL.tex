\section{SQL}
\subsection{Tabellen}
CREATE TABLE <name>  \\
( <spaltenname> <typ> [PRIMARY KEY|NOT NULL| UNIQUE],    \\
... \\
<tabellenconstraints>   // z.B. PRIMARY KEY ( Name, Ort) \\
\rule{10em}{0em}// CONSTRAINT <constraintname> FOREIGN KEY (<fk>) REFERENCES <tabelle>(<spalte>)
\rule{10em}{0em}// CONSTRAINT <constraintname> CHECK(<bedingung>)
)

ALTER TABLE <tabellenname> <änderung>
\begin{itemize}
\item ALTER COLUMN <spaltenname> <typ>
\item ADD COLUMN <spaltenname> <typ> 
\item ADD CONSTRAINT <constraint> // z.B UNIQUE, FOREIGN KEY ...
\end{itemize}

DROP TABLE <name>;

\subsection*{Fremdschlüssel}
CONSTRAINT <name> FOREIGN KEY (<spalte>) REFERENCES <tabelle>(<spalte>)