\newcommand{\com}[1]{\textcolor{lightgray}{#1}}
\newcommand{\bspbox}[1]{\fbox{\halfpage{#1}}}
\newcommand{\myspace}{\rule{2em}{0em}}

\section{SQL}
\subsection{Typen}
Ganzzahlen: INT \textcolor{lightgray}{, SMALLINT, BIGINT}

Gleitkomma: FlOAT, DOUBLE  //nicht exakt

Festkomma: DECIMAL, NUMERIC // exakt

Zeichenketten: VARCHAR(n), TEXT

Zeit: DATE, TIME, DATETIME

\subsection{Data Dictionary}
Metadaten über den Aufbau der Datenbank;

z.B. 	SELECT * FROM information\_ schema.tables;

\subsection{Tabellen}
CREATE TABLE <tabelle>  \\
(\! \myspace  <spalte> <typ> \com{PRIMARY KEY|NOT NULL| UNIQUE]},    \\
\myspace ... \\
\myspace <tabellenconstraints>  \com{ // z.B. PRIMARY KEY ( Name, Ort) \\
)}

\bspbox{
\textbf{Bsp:} CREATE TABLE student  \\
( \myspace id INT PRIMARY KEY,\\
\myspace name VARCHAR(10) NOT NULL,\\
\myspace  lehrer INT,\\
\myspace  CONSTRAINT fk\_klassleiter FOREIGN KEY (lehrer) REFERENCES angestellte(pid) on delete cascade\\
)}


ALTER TABLE <tabelle> <änderung>
\begin{itemize}
\item ALTER COLUMN <spalte> <typ>
\item ADD COLUMN <spalte> <typ> 
\item ADD CONSTRAINT <constraint> // z.B UNIQUE, FOREIGN KEY ...
\end{itemize}

DROP TABLE <tabelle>;

\subsection{Fremdschlüssel}
CONSTRAINT <name> FOREIGN KEY (<spalte>) REFERENCES <tabelle>(<spalte>)\\
ON UPDATE/DELETE    CASCADE/SET NULL



\subsection{Anfrage}
SELECT [DISTINCT] <spalte/Aggregation>      

FROM <tabelle> JOIN <tabelle> ON <bedingung> 
\com{Varianten NATURAL, LEFT, RIGHT}

WHERE <bedingung>

GROUP BY <spalten>   \textcolor{orange}{ Alle spalten im SELECT auch im GROUP BY}

HAVING <bedingung>

ORDER BY <spalten> ASC/DESC

\bspbox{
\textbf{Bsp:}
SELECT \textcolor{orange}{sg.Name, sg.Fak}, SUM(s.Bafoeg), COUNT(DISTINCT s.Ort)\\
FROM Studenten s JOIN Studiengang sg ON s.SgNr = sg.SgNr\\
GROUP BY \textcolor{orange}{sg.Name, sg.Fak}\\
HAVING COUNT(*) >= 2
}





\subsection{Funktionen}
COALESCE(<spalte>,<ersetzung>): Null-behandlung z.B. COALESCE(preis,0)

Aggregationen: SUM, AVG, COUNT, MIN, MAX ...
\subsection{Bedingungen}
WHERE <spalte> BETWEEN <a> AND <b>    //Intervall [a,b]

WHERE <spalte> IN $(x_1, x_2,\dots , x_n)$

WHERE <spalte> LIKE <pattern>   // \_ beliebiges zeichen \% beliebige kette
 
WHERE <spalte> IS NULL;

\subsection{Subquery}
FROM <subquery> 
...

WHERE <spalte> [NOT] IN (<subquery>)

WHERE EXISTS (<subquery>)\\

\bspbox{
z.B. SELECT * FROM Studiengang sg\\
WHERE EXISTS (\\
\myspace SELECT * FROM Studenten s \\
\myspace WHERE s.SgNr = sg.SgNr\\
)
}

\subsection{Einfügen}
INSERT INTO <tabelle> (<spalte1>, <spalte2> ...) VALUES (<wert1>, <wert2>,...)

Es können auch mehrere Datensätze eingefügt werden

\subsection{Verändern}
UPDATE <tabelle> SET Ort = 'Berlin', Bafoeg = 0 

WHERE <bedingung>

\subsection{Löschen}
DELETE FROM <tabelle> 

WHERE <bedingung>

\subsection{Mengenoperationen}
Spalten müssen gleiche Typen haben
 
Vereinigung: <abfrage1> UNION <abfrage2>  

Schnitt: <abfrage1> INTERSECT <abfrage2> 

Differenz: <abfrage1> EXCEPT <abfrage2>

\subsection{Transaktionen}
BEGIN, COMMIT, ROLLBACK;

SQL-Isolationssteuerung: \\
SET TRANSACTION ISOLATION LEVEL \{ READ UNCOMMITTED | READ COMMITTED | REPEATABLE READ | SERIALIZABLE \}

\subsection{Index anlegen}
CREATE INDEX <name> ON 	<tabelle>(<spalte(n)>); DROP INDEX <name>