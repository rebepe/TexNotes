\section{SQL}
\subsection{Anlegen}
CREATE TABLE <tabelle>  \\
( <spalte> <typ> [PRIMARY KEY|NOT NULL| UNIQUE],    \\
... \\
<tabellenconstraints>   // z.B. PRIMARY KEY ( Name, Ort) \\
\rule{10em}{0em}// CONSTRAINT <constraintname> FOREIGN KEY (<fk>) REFERENCES <tabelle>(<spalte>)\\

\rule{10em}{0em}// CONSTRAINT <constraintname> CHECK(<bedingung>)
)

ALTER TABLE <tabelle> <änderung>
\begin{itemize}
\item ALTER COLUMN <spalte> <typ>
\item ADD COLUMN <spalte> <typ> 
\item ADD CONSTRAINT <constraint> // z.B UNIQUE, FOREIGN KEY ...
\end{itemize}

DROP TABLE <tabelle>;

\subsection{Fremdschlüssel}
CONSTRAINT <name> FOREIGN KEY (<spalte>) REFERENCES <tabelle>(<spalte>)

\subsection{Einfügen}
INSERT INTO <tabelle> (<spalte1>, <spalte2> ...) VALUES (<wert1>, <wert2>,...)

\subsection{Verändern}
UPDATE <tabelle> SET Ort = 'Berlin', Bafoeg = 0 

WHERE <bedingung>

\subsection{Löschen}
DELETE FROM <tabelle> WHERE <bedingung>



\subsection{Anfrage}
SELECT [DISTINCT] <spalte/Aggregation>      

FROM <tabelle> JOIN <tabelle> ON <bedingung> // Varianten NATURAL, LEFT, RIGHT

WHERE <bedingung>

GROUP BY <spalte>   // Alle spalten im SELECT auch im GROUP BY

HAVING <bedingung>

ORDER BY ASC/DESC




SELECT \textcolor{orange}{sg.Name, sg.Fak}, SUM(s.Bafoeg), COUNT(DISTINCT s.Ort)\\
FROM Studenten s JOIN Studiengang sg ON s.SgNr = sg.SgNr\\
GROUP BY \textcolor{orange}{sg.Name, sg.Fak}\\
HAVING COUNT(*) >= 2






\subsection{Funktionen}
COALESCE(<spalte>,<ersetzung>): Null-behandlung z.B. COALESCE(preis,0)

SUM, AVG, COUNT, ...
\subsection{Bedingungen}
WHERE <spalte> BETWEEN <a> AND <b>    //Intervall [a,b]

WHERE <spalte> IN $(x_1, x_2,\dots , x_n)$

WHERE <spalte> LIKE <regex>

WHERE <spalte> IS NULL;

\subsection{Subquery}
FROM <subquery> 
...

WHERE <spalte> [NOT] IN (<subquery>)

WHERE EXISTS (<subquery>)\\

z.B. SELECT * FROM Studiengang sg\\
WHERE EXISTS (SELECT * FROM Studenten s WHERE s.SgNr = sg.SgNr)


\subsection{Mengenoperationen}
Spalten müssen gleiche Typen haben
 
Vereinigung: <abfrage1> UNION <abfrage2>  

Schnitt: <abfrage1> INTERSECT <abfrage2> 

Differenz: <abfrage1> EXCEPT <abfrage2>

\subsection{Transaktionen}
BEGIN, COMMIT, ROLLBACK;

SQL-Isolationssteuerung: \\
SET TRANSACTION ISOLATION LEVEL \{ READ UNCOMMITTED | READ COMMITTED | REPEATABLE READ | SERIALIZABLE \}

\subsection{Index anlegen}
CREATE INDEX <name> ON 	<tabelle>(<spalte(n)>); DROP INDEX <name>