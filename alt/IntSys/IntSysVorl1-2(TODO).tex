

\minisec{Eingabegeräte}
\oitem{Natural User Interface:} Eingabegerät nuss nicht Erlernt werden\\
\oitem{Intuitive User Interface:} Erlernen während Bedienung\\
\oitem{Räumliche UI:} Präsentation und Interaktion in 3D => 2d u.U. problematisch\\
\oitem{Haptic User Interface:} Bewegungen werden Erkannt, haptische Rückmeldungen (Vibration, drehung)\\
\oitem{Brain Computer Interface:} Steuerung durch Gehirnströme

\oitem{Gestenbasiert:} abstraktion eines Eingabegerätes ( Touchscreen, Kamera,...); \\
Vorteile: Vielseitig Einsetzbar, komplexe Befehle möglich \\
Nachteile: schlechte Erkennungsrate, Performancelastig\\
Erkennung (Rubine): bilde Eigenschaftsvektor (aus z.B. Diagonale der Begrenzungsbox, Winkel der Diagonale, Dauer der Geste, ... ) und suche passendste Geste

\oitem{Tastatur:} Eingabe von Symbolen oder Kommandos (Hotkeys), Anwendungsspezifisch oder -unabhängig, Taktil oder virtuell, evtl. konfigurierbares Layout\\ 
\oitem{Zeigegeräte:} \\
Direkt: kaum Lernaufwand, Verdeckte Ausgabe, Ermüdung, Präzisionsprobleme 
\begin{minipage}{0.5\textwidth}
\begin{itemize}
\item Touchscreen: evtl. mit Druckstärke, Multitouch
\item Stifteingabe: handschriftgeeignet (OCR), Präziser als Finger
\end{itemize}
\end{minipage}


Indirekt: Lernaufwand, Präzision, billiger, absolut\&relative Positionierung möglich 
\begin{minipage}{0.5\textwidth}
\begin{itemize}
\item Maus: ermüdungsarm, hohe Auflösung
\item Trackball: keine Fläche benötigt, fest verbaut (robust im öffentichen Raum), drag\&drop schwieriger
\item Touchpad: kleine Fläche, fest verbaut (anschluss von Peripherie schwer möglich ) %z.B. Notebook im Zug)
\item Joystick: geeignet für Steuerung und Verfolgung von Objekten
\item Trackpoint: ungenau, oft direkt in Tastatur (kombination, direkter zugriff)
\item 3D-Interaktion: virtuelle Räume, CAD; Tischgebunden oder frei bewegbar (Datenhandschuh)
\end{itemize}
\end{minipage}

\minisec{Klassifikation}
\minisec{Mackinlay}
T= (M,In,S,R,Out,W) \\
M: Operation, z.B. Rotation um X, Verschieben in Y, ...\\
In: Eingabebereich\\
S: Status des Geräts\\
R: Funktion von In nach Out\\
Out: Ausgabebereich\\
W: innere Funktionsbeschreibung (default: {})

z.B. Lautstärkeregler = (Rot z,$[0^\circ,270^\circ],0^\circ,id,[0^\circ,270^\circ],\{\}$)\\
Radioknopf = (Rot z,$[0^\circ,270^\circ],0^\circ,id,\{0^\circ,120^\circ,140^\circ\}$,\{\}) (vgl. oben, Rastet ein)

\minisecno{Krauß} \begin{itemize}
\item Nicht koordinatengebend: z.B. Taste, Geste ...
\item Koordinatengebend
\begin{enumerate}
\item Dimensionalität: 1-dimensional (Schieberegler, Drehknopf), 2-dimensional (Maus), mehrdimensional (Datenhandschuh)
\item direkt ( Touchscreen) VS indirekt(Maus)
\end{enumerate}
\end{itemize}


\oitem{Absolute Positionierung:} 1:1 Umrechnung der aktuellen Position\\
\oitem{Relative Positionierung:} Umrechnung der Veränderungen in Zeigerbewegungen 

\minisecno{Control-Display-Gain:} durch Bewegung des Zeigegerätes erzeugter Effekt\\
Gain klein: Fein-, Gain hoch: Grobpositionierung $\Rightarrow$ Lösung: Dynamische Anpassung

\minisecno{Konflikte:} Genauigkeit vs Abtastrate, Verzögerung vs Abtastrate      \\
\oitem{Abtastrate:} Erfassung des Zustandes in diskreten Zeitabständen   \\
\oitem{Verzögerung:} Zeit zwischen Abtasten und Änderung Bildschirm      \\
\oitem{Genauigkeit:} Abweichung zwischen gemessener und tatsächlicher Position   



\minisec{Mehrere Eingabegeräte: }
\oitem{Redundant:}  verschiedene (alternative) Eingabegeräte für gleiche Eingabe\\
\oitem{Komplementär:} abhängige Geräte: kopplung der Geräteinputs zu einer Eingabe \\
\tab unabhängige Geräte:  Eingaben steuern verschiedene Anwendungsteile

\minisec{Ausgabegeräte}
\oitem{stereoskopische Displays:} verschiedene Bilder für rechtes/linkes Auge;\\
 u.U. Hilfsmittel(Brille) nötig, sonst Autostereoskopie, Nicht für alle geeignet

\oitem{Holographische Displays:} echtes 3d Display, z.B. Laser

\oitem{Taktile Displays:} Fühlbares Interface, z.B. Reibungsbeeinflussung bei Touch

