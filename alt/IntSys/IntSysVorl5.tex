\minisec{Webbasierte Systeme:}
http zustandslos, v.a. Transaktionsbasiert; \\
Mehrnutzerszenarien, Benutzer/Anwendung evtl. Anonym\\
Architektur: Thickclient(kann lokal laufen) vs Thinclient(nur View)\\

\minisec{Böswillige Benutzer: (Menschen/Bots)}
Vandalismus, Trolling, Spam, Fehlinformation,\\ 
DoS, SQL-Injection, Session-Hijacking  \\ 
vs  Moderation, Automatische Prüfung, Turing Test(Captcha)\\

\minisec{Böswillige UIs}
tut mehr/anders als vom Benutzer gewollt
\begin{multicols}{2}
Zwang\\ Verwirrung\\ Ablenkung \\ Ausnutzen von Fehlern des Benutzers\\ Zusätzlicher Interaktionsaufwand\\ Unterbrechung\\ Navigation manipulieren\\ Verschleierung\\ Einschränken der Funktion\\ Erschrecken\\ Irreführen
\end{multicols}


\minisec{Frameworks:}
Struktur zur Unterstützung (z.B. Erstellung von Code-Skeletten)\\
Vorteile: Wiederverwendung von Expertenwissen, Langfristige Zeitersparnis, Systematisch Testbar, Standardisierung, Fehlervermeidung\\
Nachteile:Entwicklung zeitaufwändig, Einarbeitungsaufwand, Dokumentation und Wartung aufwändig, Fehlersuche wird erschwert, kombinierung von Frameworks schwierig\\

\textbf{Kategorien:} \\
\oitem{Blackbox} (Komponenten schon definiert, werden vom Programmierer zusammengefügt) \\
\oitem{Whitebox} (Ableitung eigener Komponenten von Basiselementen) \\
\oitem{Graybox} (Mischung aus Black- und Whitebox)\\

\oitem{Support:} Systemdienste(DirectX,OpenGL)\\
\oitem{Domain:} Funktionalität für Speziellen Bereich (z.B. Semantic Web)\\
\oitem{Application framework:} Funktionalität für komplette Anwendung (WPF)\\
\\ \\




\minisec{Entwicklung interaktive Systeme im Software-Entwicklungsprozess}
Hinzufügen von Elementen der Benutzerschnittstellen\\
Detaillieren von Dialoginhalten, -verhalten und -kontext sowie das Verknüpfen mit ihren Anforderungen\\
Anpassung an Änderungswünsche des Benutzers\\
Vervollständigen von Benutzerschnittstellen\\
Klassifizieren der Bestandteile der Benutzerschnittstelle\\
Protokollierung der Benutzerschnittstellenstruktur und der Benutzerinteraktionen\\
Kommunikation zwischen den Stakeholdern, d.h. Entwickler, Benutzer und Anwender\\
Evaluation der bisherigen Benutzerschnittstelle

\minisec{Probleme:}
\oitem{Abstraktheit} (Modelle für Benutzer schwer deutbar), \\
\oitem{Mehrdeutigkeit} (Kommunikation Benutzer $\rightarrow$ Entwickler),\\
\oitem{Fehlende Erlebbarkeit}(Konsequenzen der Änderungen Benutzer nicht klar), \\
\oitem{Benutzerziele} (unterschiedliche Benutzergruppen, Interaktionsstrategien),\\
\oitem{Komplexität interaktiver Systeme} (viele mögliche Wechselwirkungen)

\minisec{Entwicklungsprinzipien: }
\textbf{ISO 9241} Grundsätze: Aufgabenangemessenheit Selbstbeschreibungsfähigkeit Erwartungskonformität Fehlertoleranz Steuerbarkeit Individualisierbarkeit Lernförderlichkeit\\
\textbf{Prinzipien nach Gould:}\\
\oitem{Konzentration auf Benutzer }(Beobachtung, Befragung von Repräsentativen Benutzern) \\
\oitem{Randbedingungen prüfen} (Umfeldanalyse; Tätigkeitsanalyse,  willig/unwillig) \\
\oitem{Frühes und Kontinuierliches Testen}\\
\oitem{Iteratives Design:} in jeder iteration Ziele Identifizieren, Gestaltungsvorschlag, Testen; Mehrere Iterationen einplanen, änderungsfreundliche Systemarchitektur, Mockups und Papier-Prototypen\\
\oitem{Integriertes Design}( Ganzheitliche Betrachtung des Systems schon in der Entwicklungsphase)
