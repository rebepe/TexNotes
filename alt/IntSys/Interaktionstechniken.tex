\minisec{Interaktionstechniken}
\minisec{Sprachinteraktion}
Varianten: Kommandosprachen, Textuelle Suche, Natürlich-sprachlich\\
Audiosignal, Tastatureingabe

\oitem{Kommandosprachen}\\
synthetische Sprachen: effizient Implementier-/Benutzbar\\
Kriterien: eindeutige kurze Schlüsselwörter, konsistent, nahe Natürlicher Sprache\\
Autovervollständigung, Hilfssysteme, History-Funktion sinnvoll

\oitem{Natürlich-sprachlich}\\
Vorteil: kein Lernaufwand(intuitiv)\\
Nachteil: effiziente Implementierung schwer, Mehrdeutigkeiten, Dialekte, Unterscheidung Eingabe/keine Eingabe

\minisec{Menüs}
hierarchische Gruppierung atomarer Kommandos

Strukturierung: kurze einheitliche Kommandos, häufigkeit, Gruppen, Alphabet ... 

\oitem{Pulldown}
Pfad bleibt sichtbar, Untermenüs\\
Vorteil: Übersichtlich, Einheitliche Auswahl, Kein Schreibaufwand, Leicht Erlernbar\\
Nachteil: Unübersichtlich bei Vielen Einträgen, Suchproblem(Anfänger), Geschwindigkeit(fortgeschritten), Sprachabhängig

\oitem{Popup}
Lokale Kontextmenüs => Nah am Cursor

\oitem{Pie}
Kreisförmige Anordung => gleiche Nähe zum Cursor, Platzbedarf (v.A. Icons), \\
Mobile:Half-pie, Ähnlich: Marking Menüs (Joystickprinzip)

\oitem{(Semi-)Transparente Menüs}
geringe Verdeckung vs geringe Lesbarkeit

\minisec{Befehlstasten/Hotkeys}
Direkter Aufruf schneller als Menü; Kürzel sichtbar im Menü, Redundant zum Menü, Guidelines, kritische Kommandos(löschen) mit Nachfrage, 1-Hand vs 2-Hand

\minisec{Direkt Manipulative Benutzerschnittstellen}
Direktes bearbeiten eines Objektes (z.B. Drag \& Drop)\\
Vorteil: sprachunabhängig, ohne schreibkenntnis möglich (Kinder)\\
Nachteil: Realisierungsaufwand, u.U. ungenaue Handhabung

\minisec{Agentenbasierte Interaktion}
Delegieren von Aufgaben ans System => Benutzermodell, Aufgabenmodell(z.B. aus Historie) notwendig, Kontrolle und Undomechanismen\\
aktiv oder vom Benutzer aktiviert, Vorschläge oder direkt Ausführen, 

\minisec{Wizardinteraktion}
schrittweises Führen durch komplexe Aufgabe, Freiheiten in den Einzelschritten

