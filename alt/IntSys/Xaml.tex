\minisec{Namespaces}
\texttt{
xmlns=\dq http://schemas.microsoft.com/winfx/2006/xaml/presentation\dq \\
xmlns:x=\dq http://schemas.microsoft.com/winfx/2006/xaml\dq
}

Einbinden eines Namespaces z.B. System.Collections:\\
\texttt{xmlns:Collections=\dq  clr-namespace:System.Collections\dq   
}

\minisec{Data Validation}
\texttt{%
\{Binding Path=\dq  whatever\dq ~ ValidatesOnExceptions=\dq  True\dq   \\
\tab ValidatesOnDataErrors=\dq  True\dq  \} // Default: Rot Umranden}

\minisec{ErrorTemplate:}
Bsp: Fehlertext neben der Box\\
\texttt{%
<ControlTemplate x:Key=\dq  TBErrorTemplate\dq   >\\
\tab <DockPanel>\\
\tab \tab <AdornedElementPlaceholder x:Name=\dq  adornedElement\dq   />\\
\tab \tab <TextBlock Text=\dq  Hier ein Fehlertext! \dq   FontSize=\dq  12\dq   \\
\tab \tab \tab Foreground=\dq  Red\dq ~  Margin=\dq  5 0 0 0\dq   />\\
\tab </Dockpanel>\\
</ControlTemplate>\\
<TextBox Validation.ErrorTemplate=\dq  \{StaticResource TBErrorTemplate\}\dq   \\
\tab Text=\dq  \{Binding$\dots$ ValidatesOnExceptions=\dq  True\dq   \\ 
\tab ValidatesOnDataErrors=\dq  True\dq    \}\dq   />
}

\minisec{Eigene Validation Rules}
\texttt{%
public class MyValidationRule : ValidationRule\{		\\
public int maxLength \{get; set;\}					\\
public override ValidationResult Validate(object val, CultureInfo CI)\{ \\
\tab if(val.ToString().Length > maxLength)   \\
\tab \tab return new ValidationResult(false,\dq  \dq  );\\
\tab return ValidationResult.ValidResult;
\}		\\
\}
}

=> XAML:
\texttt{ %
<TextBox Validation.ErrorTemplate=\dq  $\dots$\dq    >		\\
\tab <TextBox.Text>		\\
\tab \tab <Binding Path=\dq  Property\dq    >		\\
\tab \tab \tab <Binding.ValidationRules>		\\
\tab \tab \tab \tab <RichtigerNamespace:MyValidationRule maxLength=\dq  10\dq  />		\\
\tab \tab \tab </Binding.ValidationRules>		\\
\tab \tab </Binding>		\\
\tab </TextBox.Text>		\\
</TextBox>
}\\

\minisec{Bindings:}
\oitem{!Nur public Propertys bindbar!\\}
\textbf{Modes:} TwoWay, OneWay(Kein UpdateSourceTrigger), OneWayToSource(Schreibend), OneTime(InitialiseComponent)\\
\textbf{UpdateSourceTrigger:} PropertyChanged, LostFocus(default bei Textbox)  

\texttt{%
\fbox{XAML: z.B. <Textbox Text=\dq  \{Binding ElementName=box1, Path=Text\}\dq   />		}	\\
\fbox{
\begin{minipage}{0.4\textwidth}
C\#: Binding b = new Binding(\dq  Text\dq  );      // QuellProperty					\\
b.Source = box1;       //Objekt																		\\
box2.SetBinding(TextBox.TextProperty, b)	
\end{minipage}
} }



Binding auf eigene Klassenobjekte mit automatischem aktualisieren: \\
z.B. \texttt{ using System.ComponentModel;\\
class TestData: INotifyPropertyChanged \{			\\
\tab public event PropertyChangedEventHandler PropertyChanged;		\\
\tab string myProperty;													\\ \\ 
\tab public string MyProperty \{ \\
\tab \tab  get\{return myProperty;\} \\
\tab \tab set\{ myProperty = value; \\
\tab \tab \tab OnPropertyChanged(\dq  MyProperty\dq  );\}\\
\tab \} \\ \\
\tab protected void OnPropertyChanged(string pname) \{ \\
\tab \tab PropertyChangedEventHandler handler = PropertyChanged;\\
\tab \tab if (handler != null)											\\
\tab \tab \tab handler(this, new PropertyChangedEventArgs(pname));		\\
\tab \tab \}\\
\tab \}
}



\minisec{Datenkontext}
\texttt{DataContext=\dq  \{Binding RelativeSource=\{x:Static \\ \tab RelativeSource.Self\}\}\dq   \tab \tab //Kontext ist Code-Behind}



\minisec{Routed Events}
Reihenfolge: zuerst Tunneling, dann Bubbling;\\
\textbf{Direkte Events:} nicht weitergereicht (z.B. Click)\\
\textbf{Bubbling} Weiterreichen nach Oben (zum Window) z.B. MouseLeftButtonDown \\
\textbf{Tunneling} Weiterreichen nach Unten (zum Button) z.B. PreviewMouseLeftButtonDown

\clearpage

\minisec{Styles}
\texttt{%
<Window.Resources>		\\
\tab <Style x:Key=\dq  MyStyle\dq ~  TargetType=\dq  \{x:Type Button\}\dq     >		\\
\tab \tab <Setter Property=\dq  Background\dq ~   Value=\dq  Green\dq    />				\\
\tab </Style>		\\
</Window.Resources>		\\
<Button Style=\dq  \{StaticResource MyStyle\}\dq   ...		
}

