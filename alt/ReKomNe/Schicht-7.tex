% --- Schicht 7
\Minisec{Anwendungskommunikation}
Request-Response-Paradigma: Anfrage->Antwort\\
Textbasierte-Protokolle: (z.B. smtp,http), Vorteile: debuggen, leicht zu Dokumentieren(RFC)

\Minisec{http}
Form: Kommandozeile, Headerzeilen, Leerzeile, [Bodyzeilen] (Bodyende: Headerfeld oder End-Tags)\\
Kommandozeile: Methode(GET,$\dots$) Anfrage-URI Version

Persistente Verbindung: Nutzen einer TCP-Verbindung für mehrere HTTP-REQUESTS (Flusskontrolle, Handshake)\\
Pipelining: Client stellt mehrere Anfragen über eine Verbindung bevor erste Antwort kommt\\
Long Polling: Server antwortet nicht sofort, sondern nur bei Ereignis oder Timeout (asynchronität)\\
zustandslos => keine Sitzungen (Benutzeridentifikation) => SitzungsID in Cookie oder URL

\Minisec{Web-Sockets}
Bidirektionale Kommunikation über einen durch http etablierten Kanal => Zuordnung zu Sitzung leicht möglich\\
Nach dem Aufbau verwendung eigenes Format (kein Http): Längenangabe einzelner Nachrichten, zufallsmaskierung gegen Proxie-Caching

\Minisec{Remote Procedure Call}
Aufruf von Funktionen/Methoden auf entferntem Rechner z.B. RPC, CORBA(Sprachenunabhängig), RMI(Java)\\
Nötig: Stubs auf Client/Server, die die Kommunikation übernehmen, plattformunabhängige Kodierung der Daten und Serialisierung(abrollen von Objekten) 

\Minisec{Webservices}
HTTP als Quasi-Transportprotokoll (wird nicht blockiert,...) \\
v.A. für Kommunikation zwischen verschiedenen Verantwortungsbereichen z.B. Amazon $\longleftrightarrow$ DHL

\minisec{SOAP}
Definition von Operationen (mit Parametern, Rückgabe ...)\\
Schnittstellenbeschreibung in WSDL, wird über HTTP übertragen => Prüfen auf Aktuelle Version, Stubgenerierung möglich

\minisec{REST}
Zugriff auf Ressourcen über HTTP-Kommandos mit URIs, Argumente im Body, Rückgabe in Content-Feld\\
z.B. GET http://abookstore/cart/12345\\
Format z.B. in JSON / XML; Kommandos GET, PUT, POST, DELETE

