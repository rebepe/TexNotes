\Minisec{TCP/IP-Prüfsumme: }
\begin{minipage}{0.5\textwidth}
\begin{enumerate}
\item Aufsummieren aller 16-Bit-Blöcken in Einerkomplementdarstellung \\
\textbf{Einerrücklauf}(Wert / 65536)\\
Tatsächlicher Wert: mod 65536;\rule{2cm}{0pt} 65535 = 0; 
\item invertieren aller Prüfsummenbits (Erg = $65535 - $Wert)
\end{enumerate}
\end{minipage}
=> auf Empfängerseite muss $00 \dots 0 $ herauskommen\\
 

\newcommand{\Rx}[1]{\textcolor{blue}{#1}} 
\newcommand{\xx}[1]{\textcolor{cyan}{#1}}
\newcommand{\Ix}[1]{\textcolor{red}{#1}}
\newcommand{\Gx}[1]{\textcolor{brown}{#1}}
 
\Minisec{CRC (Cyclic Redundancy Check)}
\begin{minipage}{0.25\textwidth}
Bitkette $\Leftrightarrow$ Polynom, \\z.B. 1011 $\Leftrightarrow$ $x^3 + x ^1 + 1$\\
$+=-= $Xor\\\\
\Ix{I(x)} = Nutzdaten  \subtil{Informationspolynom}\\
\Gx{G(x)} = vorgegeben \subtil{Generatorpolynom }\\
\xx{k} = grad von G(x)\\
C(x) = Frame \subtil{Codepolynom}  \\\\
Berechnen: \Rx{R(x)} = \Ix{I(x)} $\cdot$ $\xx{x^k}$ mod \Gx{G(x)}; \rule{1cm}{0cm} \\
C(x) = \Ix{I(x)} $ \cdot \xx{x^k}$ +  \Rx{R(x)};\\
=> Empfänger prüft, ob C(x)/G(x) = 0;
\end{minipage}
\texttt{
\begin{minipage}{0.25\textwidth}
\newcommand{\divisor}{\underline{1011}\\}
\Ix{I= 100010,} \Gx{G(x)=$x^3 + x ^1 + 1 \widehat{=}1011$}\\
\textcolor{white}{|}\Ix{100010}\xx{000}:\Gx{1011}=101100\\
\textcolor{white}{|}\divisor
\textcolor{white}{|}001110\\
\textcolor{white}{|}~~\divisor
\textcolor{white}{|}~~01010\\
\textcolor{white}{|}~~~\divisor
\textcolor{white}{|}~~~000\textcolor{blue}{100 <= R(x)}\\
=> C(x) = \Ix{100010}\Rx{100}
\end{minipage}
}
