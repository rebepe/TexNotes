\begin{multicols}{2}
\minpurp{RSA}
\minisec{Schlüsselerzeugung}
Wähle 2 große Primzahlen p und q; n= $p \cdot q$\\
Wähle e und berechne d = $e^{-1}$ mod $\varphi(n)$\\
Schlüssel: öffentlich (n,e), geheim (n,d)

\minisec{Verschlüsselung}
c = $m^e (mod n)$
\minisec{Entschlüsselung}
m = $c^d (mod n)$ Tipp: Berechne in mod p(Exponent mod p-1) und mod q(Exponent mod q-1) ,anschließend Chin. Restesatz)
\end{multicols}

\begin{multicols}{2}
\minisec{Signatur}
$s = m^d$ mod n; 	Signatur:(m,$s$)\\

\minisec{Verifikation}
(m,$s$): prüfe ob $m = s^e$\\
\end{multicols}




\minisec{RSA- Attacken}
\begin{itemize}
\item für $\lvert p-q \rvert$ klein: teste x>$\sqrt{n}$ bis $x^2 -n$ ein quadrat ist; p,q = $\Rightarrow x \pm \sqrt{x^2 -n}$
\item "Common Modulus": gleiche Nachricht m an zwei Empfänger mit gleichem n und unterschiedlichen e $\rightarrow$ dritter kann m ausrechnen;\\
\Euklid ($e_1^{-1} mod~e_2$): $ 1= a \cdot e_1 + b \cdot e_2 \rightarrow m = c_1^a \cdot c_2^b$\\
zudem kann der eine Empfänger den geheimen Schlüssel des anderen berechnen. 
\item "Low-Encryption-Exponent": wird gleiche Nachricht m an 3 Empfänger mit e=3 gesendet kann $m^3~(mod~n_1n_2n_3)$ berechnet werden (Chin. Restesatz);\\
Da $m^3 < n_1n_2n_3$ ist $m = \sqrt[3]{m^3}$
\item Small-Message-Space Attack: nur wenige, bekannte Nachrichten (z.B. j,n) können deren c's berechnet werden $\Rightarrow$ Entschlüsseln durch Vergleich 
\item Chosen-Ciphertext: um c zu entschlüsseln: wähle beliebiges r, lasse $r^ec$ mod n entschlüsseln :m'; m= m'$r^{-1}$\\ 
Angreifer fängt c ab und berechnet $x=r^ec$ und lässt Empfänger x signieren; \\
(Signatur =$x^d$) => m = $r^{-1}x^d$ 
\item Existentiell fälschbar: ($f^e$,f) mit f beliebig wird als signierte Nachricht anerkannt.
\item Bekannte Signatur $(m,s) $ => z.B. $(m^2,s^2)$ wird als signierte Nachricht anerkannt.
\end{itemize}
\minisec{ElGamal-Signatur-Attacken}
\begin{itemize}
\item wenn k wiederverwendet: $k \equiv (s_1 - s_2)^{-1}(m_1 - m_2)$, weiter im nächsten Punkt 
\item wenn k bekannt dann ist $x \equiv r^{-1} (m -sk) (mod p-1)$
\item Existentiell fälschbar: wähle b,c => $r= g^by^c$, $s= -rc^{-1} (mod p-1)$, $m = -rbc^{-1}$
\item Bleichenbacher: wenn gültige signatur (m,r,s) bekannt kann beliebiges $m_{neu}$ signiert werden: 
$u = m_{neu}m^{-1} (mod p-1)$,$s_{neu} = su (mod p-1)  $ \\
$r_{neu}$ aus Chin. Restesatz: $r_{neu} \equiv r mod p$ und $r_{neu} \equiv ru (mod p-1)$
\end{itemize}

%\textcolor{white}{.\\.\\.\\}

\minpurp{OAEP}
n=k+l:
Falltürfunktion f (z.B. RSA): n-Bit $\to$ n-Bit\\
Pseudozufallsgenerator G: k-Bit $\to$ l-Bit\\
Hashfunktion h: l-Bit $\to$ k-Bit
\begin{multicols}{2}
\minisec{Verschlüsseln von m (l-Bit)}
\begin{enumerate}
	\item Wähle r:Zufallszahl mit k-Bit
	\item x = (m $\oplus$ G(r)) || (r $\oplus$ h(m $\oplus$ G(r))
	\item c = f(x)
\end{enumerate}
\minisec{Entschlüsseln von c (n-Bits)}
\begin{enumerate}
	\item Zerlege c in a b,  a=l-Bits, b=k-Bits
	\item r= b$\oplus$h(a)
	\item m = a $\oplus$ G(r)
\end{enumerate}%
\end{multicols}%
\minpurp{ElGamal}
\minisec{Schlüsselerzeugung}
Wähle Primzahl p, (ideal: p-1 mit großem Primfaktor); suche primitive Wurzel g mod p\\
wähle zufallszahl $x \in [0,p-1]$, berechne y=$g^x mod p$\\
privat: (p,g,x) öffentlich: (p,g,y)

\begin{multicols}{2}
\minisec{Verschlüsselung}
wähle $k \in [1,p-2]$ \\
 c= ($c_1,c_2$) = ($g^k, y^km(modp)$)
\minisec{Entschlüsselung}
$\textcolor{gray}{\underbrace{-x= p-1-x}}$\\
m= $c_1^{-x}c_2$
\end{multicols}

\begin{multicols}{2}
\minisec{Signatur}
wähle $\textcolor{magenta}{k \in [1,p-2]}$, gcd(k,p-1)=1\\
berechne r = $g^k, s=k^{-1}(m-rx)$ mod (p-1) => Signatur(m,r,s)\\
\textcolor{magenta}{Wichtig: k muss immer neu gewählt werden, k muss gute Zufallszahl sein}
\minisec{Verifizieren}
Prüfe $r \in [1,p-1]$;~~~Prüfe $g^m = y^rr^s$
\end{multicols}

\minpurp{DSA (Digital Signature Algorithm)}
\minisec{Schlüsselerzeugung}
Wähle Primzahl p, so dass p-1 mit Primfaktor q(160Bit) \\ 
suche g mit ord(g)=q mod p, (Also: $g^{\frac{p-1}{q}} \neq 1$)\\
wähle zufallszahl $x \in [0,q-1]$, berechne y=$g^x mod p$\\
privat: (p,q,g,x) öffentlich: (p,q,g,y)

\begin{multicols}{2}
\minisec{Signatur}
Wähle $k \in [1,q-1]$\\
Berechne r=$g^k mod p mod q$, \\
s=$k^{-1}(m+rx)mod q$ => Signatur (m,r,s)

\minisec{Verifizieren}
Prüfe $r,s \in [1,q-1]$\\
Berechne t= $s^{-1}$mod q\\
Prüfe ($(g^my^r)^t$mod p) mod q = r
\end{multicols}

\clearpage