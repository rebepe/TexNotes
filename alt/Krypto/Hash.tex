\minpurp{Hash}
\{0,1\}* -> $\{0,1\}^n$ // Beliebige Bitketten -> Bitketten fester Länge
Eigenschaften: \begin{enumerate}
\item One-way function: Nicht umkehrbar
\item second pre-image resistance: Nicht möglich auf gegebene Nachricht\& Hashwert andere Nachricht zu finden.
\item collision resistance: Nicht möglich zwei Nachrichten mit gleichem Hash zu finden => stärkste Bedingung
\end{enumerate}

\minisec{Merkle-Damgard}
Kollisionsresistente Kompressionsfunktion: f:$\{0,1\}^{n+r} \rightarrow \{0,1\}^n$	\\

Padding: hänge 10$\dots$0 an, so dass |m|= vielfaches von r;

Hashen: %
\begin{enumerate}
\item teile m in Blöcke der Länge r
\item füge Block der Länge r in dem die ursprüngliche Länge von m steht
\item $v_0$= (zufälliger) Startwert
\item $v_i$= f($v_{i-i}||m_i$)
\item Hashwert ist letztes v
\end{enumerate}

\minisec{Birthday-Attack}
Angriff gegen Kollisionsresitenz: Cachen von Hashwerten und testen auf Kollision

\minisec{Kompression aus Blockchiffren}
$f_1: (x||y) \rightarrow E(y,x)$ // Spalte Wert in Key und Message und Verschlüssle => Komprimierter Wert 

\minpurp{MAC}
\minisec{HMAC}
HMAC(k,m) = h( (k$\oplus$opad)||h((k$\oplus$ipad)||m) )\\
h: Hash nach Merkle-Damguard mit Kompressionsrate r (in Bytes);\\
k: Schlüssel mit Padding auf Länge r, ipad: r mal 0x36, opad: r mal 0x5C

\minisec{CBC-MAC}
Aus Blockchiffre:$c_0=IV$, $c_i= E(k,m_i\oplus c_{i-1}$\\
=> letztes c ist MAC

\minisec{Pseudozufallsgenerator PRF}
PRF(secret,seed)= HMAC(secret,A(1)||seed) || HMAC(secret,A(2)||seed) ...\\
A(0)= seed, A(i) = HMAC(secret,A(i-1));