\begin{multicols}{2}
\minisec{DES:}
m = 64 Bits; 	k = 56 Bits\\
Aus k werden 16 Rundenschlüssel $k_i$ generiert;

\minisec{Verschlüsseln:}
Führe Anfangspermutation durch;\\
Ablauf einer Runde i:
\begin{enumerate}
\item Halbiere die nachricht m in x und y: $m \to (x,y)$
\item  Aus (x,y)berechne $(x \oplus f(k_i,y),y)$

 mit $f(k_i,x) = P(S(E(x)\oplus k_i))$\\
	S= Substitution (ersetzten des Blocks)\\
	P= Permutation (Umsortieren der Bits)
\end{enumerate}

\minisec{Entschlüsseln:}
Führe Verschlüsseln in Umgekehrter Reihenfolge durch
\\ \\ \\
\minisec{AES:}
m= 128 Bits;	k= 128, 192 oder 256 Bits		\\
Rundenanzahl ($N_r$): Für k= 128: 10 k= 192: 12 k= 256: 14

Zustandsmatrix:  4x4 Matrix

\minisec{Ablauf:}
\begin{enumerate}
\item initialisiere Zustandsmatrix mit m
\item bilde $Zustandsmatrix \oplus k_0$
\item for i = 1 bis i= $N_R-1$
\begin{enumerate}
\item 	S-Box: $x = x^{-1} \in F_{2^8}$
\item	Verschiebe die Zeilen der Matrix
\item 	Durchmische die Spalten (MixColumns)
\item 	bilde $Zustandsmatrix \oplus k_0$
\end{enumerate}
\item S-Box
\item Verschiebe die Zeilen der Matrix
\item Durchmische die Spalten (MixColumns)
\end{enumerate}


\end{multicols}

\minisec{Modes of Operation:}		
Ist |m|>r: 	Teile m in Blöcke und für jeden Block: (r= Blocklänge)\\
\begin{itemize}
\item ECB: Electronic Codebook Mode \\
			$c_i = E(m_i)$		( Blöcke einzeln Verschlüsseln)\\
			Fehler: nur betroffener Block
\item CBC: Cipher-Block Chaining Mode	\\
			Wähle zufälliges Kryptogramm $c_0$		(Startwert, wird mitgeschickt)\\
			$c_i = E(m_i \oplus c_{i-1})$			(Verschlüssle Nachricht $\oplus$ voriges Kryptogramm\\
			Fehler: betroffener Block und folgender
\item CFB: Cipher Feedback Mode: \\
			$c_i = m_i \oplus msb_r(E(x_i))$	(nimm die vordersten r Bits aus $E(x)$ und XORe)\\
			$x_{x_i+1} = lsb_{|x|-r}(x_i) || c_i$      (shifte $c_i$ in x ein)\\
			Fehler: Fehler in den Folgenden Blocklänge/r c's; zus. bei Bitfehlern => Bitfehler an gleicher Stelle
\item OFB: Output Feedback Mode:\\
			$c_i = m_i \oplus msb_r(E(x_i))$ 	(siehe CFB)\\
			$x_{i+1} = E(x_i)		$		(x als Pseudozufallsgenerator)\\
			Fehler: verloren = alle Folgenden, Bitfehler = Bitfehler an gleicher Stelle
\end{itemize}

\rule{\textwidth}{1pt}