\begin{multicols}{2}
\minisec{Ziele der Kryptographie:}
\begin{itemize}
	\item Vertraulichkeit (kein Abhören möglich)
	\item Integrität der Daten (keine Veränderung von außen)
	\item Authentizität des Partners
\end{itemize}

\minisec{Verfahren: }
\begin{itemize}
	\item Symmetrische Verschlüsselung (z.B. AES): Beide Partner kennen den geheimen Schlüssel
	\item Asymmetrische Verschlüsselung (z.B. RSA): Der Empfänger kennt den geheimen Schlüssel, der Verschlüsselungsschlüssel ist allgemein bekannt
\end{itemize}

\minisec{Chiffren:}
\begin{itemize}
	\item Blockchiffre: Verschlüsselung von Blöcken (feste Länge) 
	\item Stromchiffre: Verschlüsselung von beliebigen Bit-Ketten (z.B. One-Time-Pad)
\end{itemize}

\minisec{Attacken:}
\begin{itemize}
	\item Ciphertext-only
	\item Known-plaintext
	\item Chosen-plaintext
	\item Adaptively-chosen-plaintext
	\item Chosen-, Adaptively-chosen-ciphertext
\end{itemize}

\minisec{Challenge Response (User Autentification): }
\begin{enumerate}
	\item B wählt Challenge c  
	\item A signiert c 
	\item B überprüft die Signatur
\end{enumerate}
\end{multicols}\rule{\textwidth}{1pt}