\begin{minipage}{0.7\textwidth}

KB = $2^{10}$, MB=$2^{20}$

$Taktfrequenz 	= 	\frac{1}{Zyklusdauer};  	Zyklusdauer  	= 	\frac{1}{Taktfrequenz}$ z.B.2ns = 0.5GHz 

\minisec{Amdahl}
$	Dauer_{neu}  	= 	Dauer_{alt} \cdot \left( Anteil_{unbenutzt} 	+ \frac{Anteil_{benutzt}}{Beschleunigung_{benutzt}} \right)   $\\
$	Beschleunigung  = 	\dfrac{1}{Anteil_{unbenutzt} + \dfrac{Anteil_{benutzt}}{Beschleunigung_{benutzt}}}	$

\end{minipage}
\begin{minipage}{0.2\textwidth}
\begin{tabular}{c|c|c|c}
\hline
 &Hz & s \\
\hline
$10^3$ & k & m & $10^{-3}$ \\
\hline
$10^6$ &M & $\mu$ & $10^{-6}$ \\
\hline
$10^9$ &G  & n & $10^{-9}$ \\
\hline
\end{tabular}
\end{minipage}

\begin{minipage}{0.7\textwidth}
\minisec{Fehlertoleranz}
Wahrscheinlichkeit (nicht parallel):	$A_{sytem}$ 	= Produkt aller Einzelwahrscheinlichkeiten \\ 
Wahrscheinlichkeit TMR: $A_{sytem}	= (3p^2 - 2p^3) \cdot p_{voter}$

\newcommand{\note}[2]{\textcolor{gray}{\underbrace{\textcolor{black}{#2}}_{\text{#1}}}}

Allg: mind. x-1 aus x fehlerfrei: $A_{sytem}	= \note{x aus x fehlerfrei}{p^x} + \note{x-1 aus x fehlerfrei}{\left( \! \begin{array}{c} x \\ x-1 \end{array} \! \right) p^{x-1}(1-p)}$\\
\end{minipage}
\begin{minipage}{1pt} \rule{1pt}{5\baselineskip} \end{minipage} 
\begin{minipage}{0.3\textwidth}
Quadratische Gleichungen:\\
 $ax^2+bx+c=0$\\
\textbf{$x_{1/2} = \dfrac{-b \pm \sqrt{b^2-4ac}}{2a} $}
\end{minipage}

\minisec{Codes:}
gerade Parität: gerade Anzahl an 1\\
Hamming-Abstand h: Anzahl der Bits die sich unterscheiden\\
Hamming-Abstand bei Code: kleinster Hamming-Abstand der Codewörter\\
Erkennung von d Bitfehlern: h=d+1; 	Korrektur von d Bitfehlern: h=2d+1\\



\newcommand{\rub}[1]{\textcolor{red}{\overline{\textcolor{black}{#1}}}}
\newcommand{\bub}[1]{\textcolor{blue}{\overline{\textcolor{black}{#1}}}}
\newcommand{\gub}[1]{\textcolor{green}{\overline{\textcolor{black}{#1}}}}
\newcommand{\brub}[1]{\textcolor{brown}{\overline{\textcolor{black}{#1}}}}


\minisec{Erstellen von Hamming-Code, Einzelbitkorrektur:}
m= Datenbits, r=Prüfbits \\
Suche r mit  $(m+r+1) \leq 2^r$\\
Nummeriere Bits von links mit 1 beginnend durch\\
Bits mit 2er-Potenznummern (1,2,4,8,...) sind Paritätsbits; aufteilung nach binär (5 ist 1 + 4) \\
\fbox{  \begin{minipage}{0.3\textwidth}
\textbf{Bsp:} Gerade Parität mit $ m=10 $ \\
$ 10 + r + 1 \leq 2^r \Rightarrow r=4$
\end{minipage} } \rule{1cm}{0pt}
\fbox{  \begin{minipage}{0.3\textwidth}
\begin{tabbing}
Beispielwort: \=1~\=0~1~\=1~0~1~0~\=1~0~1~0~1~0~1~0   \kill
\> \textcolor{gray}{1} \> \textcolor{gray}{2} \> \textcolor{gray}{4} \> \textcolor{gray}{8}  \\
Beispielwort: \rub{\textcolor{red}{  1}}~\bub{\textcolor{blue}{  0}~\rub{1}}~\gub{\textcolor{green}{  1}~\rub{0}~\bub{1~\rub{0}}}~\brub{\textcolor{brown}{0}~\rub{0}~\bub{0~\rub{0}}~\gub{0~\rub{0}~\bub{0~\rub{0}}}}
\end{tabbing}
\end{minipage} }

\textbf{SECDED} zum Erkennen von 2-Bit Fehlern: Hamming Code mit zus. Paritätsbit (Stelle 0) über das Ganze Wort\\


\minisec{Festplatte}
Kapazität = $\textbf{2} \cdot Scheiben \cdot Spuren \cdot Sektoren \cdot Sektorengr$öß$e $\\
$Geschwindigkeit_{\text{außen}} \textbf{>} Geschwindigkeit_{innen}$\\
$mittlere Zugriffszeit = Sektoren \cdot (t_{Spurwechsel} + t_{Halbe Umdrehung})$\\
$Verschnitt = Sektorgr$öß$e - Rest(letzter Sektor) $\\


\minisec{SSD}
Lebensdauer=$\frac{\text{mögliche Schreibzyklen}}{\text{Schreibvorgänge pro Zeit}}$\\
mögliche Schreibzyklen: verfügbare Zellen * Schreibzyklen pro Zelle (= TBW)\\

Bei Wear-Leveling: 
\begin{enumerate}
\item ohne: Lebensdauer = $\frac{\text{Schreibzyklen pro Zelle}}{\text{Schreibvorgänge pro Zeit}}$ (verfügbare Zellen = 1)
\item dynamisch: verfügbare Zellen: Gesamtkapazität - statische Daten
\item statisch: mögliche Schreibzyklen: gesamte SSD
\end{enumerate}


\minisec{Bus}
Steuer-, Adress-, Datenpins, (+Interruptpins)

Bus-Protokoll: Wer legt wann welches Signal an;	\\
Bus-Master: kann Transfer einleiten, Bus-Slave: passiv (Rolle von Kommunikation abhängig)\\
Bus-Breite: Anzahl der Leitungen \\

Synchroner Bus: zentraler Takt $\Rightarrow$ Zeiträume aus Graph ablesen

Asynchroner Bus: Buszyklen variabler Länge $\Rightarrow$ Signaländerungen lösen Reaktionen aus\\

Bus-Arbitrierung: zentral: Bauteil (Arbiter) teilt Buszugriffe zu. (nach Prioritäten)\\
dezentral: Bauteile prüfen selbst ob sie die höchste angeforderte Priorität haben

Blocktransfer: holen von x Worten ab Adresse
