\minisec{ISA}
\minisec{Instruktionsformate:}
Opcode +[Operand/en], evtl. gleiche Länge

\minisecclose{Addressiermodi}
\begin{itemize}
\item Umittelbar: Konstanter Wert (z.B. 4)
\item Direkt: Speicherstelle (z.B. 0x4568AF)
\item Indirekt: Adresse der Speicheradresse (Zeiger)
\item Register: Registerinhalt
\item Indirekte Register: Adresse steht im Register (z.B: Stackpointer)
\item Indiziert: Offset von fester Adresse (z.B. Arrays)
\item Basis-Indiziert: Offset von Registerinhalt
\item Stapel: vgl. Postfix-Notation
\end{itemize}

Für Sprünge:
\begin{itemize}
\item Direkt
\item Indirekte Register
\item Indiziert 
\item PC-relative: Offset zum Program Counter
\end{itemize}



\minisec{Postfix-Notation:} Operator hinter Operanden \\
z.B. A+BxC => ABCx+\\
(A+B)/(C-D) => AB+CD-/


