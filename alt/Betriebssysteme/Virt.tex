\section{Virtualisierung}
Abstraktion von Ressourcen durch Softwareschicht

BS-Virtualisierung: Softwareschicht: Hypervisor oder VMM

\textbf{Vorteile} von VMs: unterschiedliche Betriebssysteme gleichzeitig auf einem Rechner, Hardwareauslatung, Test nicht vertrauenswürdiger Programme, Einfaches Sichern / Wiederaufsetzen /Klonen (Snapshot)

\textbf{Nachteile}: VMs müssen sich reale Ressourcen teilen, Rechnerausfall =>  Ausfall aller VMs, probleme mit spezialhardware, ca. 10\% Leistungsminderung


\textbf{Emulation}: Interpreter der Maschinenbefehle, Optimierung: JiT-Compilation

\textbf{Virtualisierung}: native Ausführung von Nichtprivilegierten Befehlen
Voraussetzung: Ausreichend ähnliche Prozessoren

Technik: Hypervisor fängt kritische Befehle ab, und ersetzt sie durch geeignete Systemaufrufe 
(Implementiert durch Interrupt-Service-Routinen, Hypervisor im Kernelmodus, Gast-BS im Usermodus)


\textbf{Mindestvorraussetzungen}: Unterscheidung zwischen Kernel- und Anwendungsmodus durch 
Prozessor, MMU ,Kriterien nach Popek und Goldberg:

\textbf{Popek und Goldberg: }

\halfpage{
\begin{itemize}
\item  Es gibt privilegierte Befehle, \item diese Lösen Trap aus wenn nicht im Kernelmodus. \item alle sensitiven Befehle: (zustandsverändernd z.B. Zugriff auf I/O oder die MMU,etc.) müssen privilegiert sein.
\end{itemize}
}

Lösung falls nicht alle sensitiven Befehle priviligiert sind: Ersetzung zur Laufzeit oder JiT-Compilation

\subsection{Typen}
\textbf{Typ-1:} Hypervisor als mini-Betriebssystem, ohne Wirt, Treiber ggf. problematisch

\textbf{Typ-2:} Hypervisor als Prozess des Wirtsbetriebssystem, nutzt Wirtstreiber mit

\textbf{Paravirtualisierung}: ersetzung kritischer Befehle im Gast Betriebssystem durch Hypercalls 
=> schneller, vorraussetzung: Quellcode des Gast-BS offen

\subsubsection{Speicher}
Virtuelle Adresse Gast -> Virtuelle Adresse Host -> Reale Adresse Host

Schattentabelle: tabelle für umsetzung $Virt_{Host} -> Real_{Host}$

alternativ Hardwarelösung EPT (Extended Page Table) in MMU

\subsubsection{Treiber}
\textbf{Geräte-Emulation}: Nutzen des Treibers des Hypervisors oder Host-BS, Standardgerät für den Gast; 

schlechter Durchsatz

\textbf{Direktzuweisung eines Geräts}: Gast nutzt Gerät exklusiv mit eigenem Treiber

Effizienteste Möglichkeit, Erfordert spezielle Hardware


\subsubsection{Verschieben einer VM:} kopieren von Dateisystem + RAM von Quelle zum Ziel;
Wiederholt die geänderten Teile umkopieren; anhalten Quelle, kopieren des Rests, Umleiten Netzwerk, Ziel-VM starten, Quell-VM löschen

\section{Cloud}
IaaS = Infrastructure as a Service: Cloud-Provider bietet Zugang zu virtualisierten Rechnern (inkl. BS)
und Speichersystemen

PaaS = Platform aaS: Cloud-Provider bietet Zugang zu Programmierungsumgebungen

SaaS = Software aaS: Cloud-Provider bietet Zugang zu Anwendungsprogrammen
