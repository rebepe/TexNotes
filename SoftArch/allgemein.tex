\textbf{\textcolor{purple}{ Prinzipien}}~
\begin{itemize}
\item abstraktion (modellbildung, vernachlässigung von details); 3 Merkmale:
	\begin{itemize}
		 \item Abbildungsmerkmal: Bildet reelles/fiktives ab => Realitätsbezug
		 \item Verkürzungsmerkmal: hebt Wesentliches hervor, lässt Unwesentliches weg
		 \item Pragmatisches Merkmal: Kann unter bestimmten Bedingungen original ersetzen
	\end{itemize}
\item Strukturierung: komplex => reduzierte Darstellung mit gleichem character und spezifischen Merkmalen
\item Dekomposition (Stepwise Refinement, Zerteilung eines problems)
\item Hierarchisierung (Erstellen einer Rangordnung => Hierarchieebenen; Spezialfall Strukturierung)
\item Wiederverwendung (technische,organisatorische Vorraussetzungen; \~60\% Mehraufwand)
\item Standardisierung (Festlegung/Einhaltung von Richtlinien, Normen)
\item Verbalisierung (Ideen geeignet in Worten ausdrücken, z.B. Namen)
\item Modularisierung (Zusammenbau des Systems aus Bausteinen s.U.)
\end{itemize}
\textbf{\textcolor{purple}{Modularität}}
\begin{itemize}
\item hohe Kohäsion (Abhängigkeiten im Modul,eine Verantwortung)
\item lose Kopplung (Abhängigkeiten zwischen modulen)
\item Information Hiding (Geheimnisprinzip)
\item Schnittstellenspezifikation
\item Kontextunabhängigkeit (Analog zu lose Kopplung zur Umgebung
\item Lokalitätsprinzip (Alle Bestandteile zu Problemlösung an einer Stelle)
\end{itemize}
=> Änderungsfreundlichkeit, Wartbarkeit, Standardisierung, Arbeits- Organisation und Planung, Überprüfbarkeit

\minmeth{Modulare Operatoren}
\begin{itemize}
\item Splitting: Aufteilen eines Moduls
\item Substituting: Austausch von Modul durch Modul gleicher Funktion
\item Augmenting: Hinzufügen von Modul für Funktionserweiterung (leicht)
\item Excluding: Entfernen von Modul für Funktionsverringerung (schwer)
\item Inverting: Herausziehen von redundanter Funktionalität aus Modulen in höheres Modul
\item Porting: Verwenden eines Translators für ein Modul in nicht vorhergesehenem Kontext
\end{itemize}

\submeth{Architectural Erosion}
Verletzungen der Architektur. Änderungen => inkonsistenter Code (entfernen tragender Säulen).

\submeth{Architectural Drift}
Erweiterung ohne Beachtung der bestehenden Architektur (Anbauten).

\submeth{Gegenmaßnahmen} Dokumentation, Open-Closed-Prinzip, Kapselung, Modularisierung

\minpurp{Strukturparadigmen}
Orhogonal zu Programmierparadigmen
\begin{itemize}
\item Unstrukturiert (Spagetti-code, goto)
\item Strukturiert (Blöcke (z.B. if))
\item Modular (sammlung von Typen/Prozeduren, explizite schnittstellen)
\item Objektorientierung (Kapselung von Daten/Methoden in Objekten)
\end{itemize}

\minpurp{CleanCode Prinzipien}
Fail-Fast(Exceptions statt Rückgabe), 

Command-Query-Separation(nicht Zustandänderung u. -abfrage in ener Methode), 

Least Astonishment (tu das was signatur beschreibt ) , 

unnötiger Boilerplate-code (Boilerplate = zus. Aktionen vor/nach Funktionsaufruf ) , 

defensive Kopien (Kopien bei get/set-Aktionen, incl. Konstruktor)


