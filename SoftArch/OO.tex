\minpurp{Objektorientierung}
\minmeth{Prinzipien eines wiederverwendbaren OO Designs}
\begin{itemize}
\item Program to an Interface, not an Implementation
\item Favor delegation over class inheritance. \\(Delegation: aufruf der methode in anderer Klasse z.B Strategy)
\end{itemize}

\minmeth{Law of Demeter/Feature Envy}
Empfehlung: Klassen sollten nur mit direkt gekoppelten Klassen arbeiten, nicht mit indirekt verknüpften

A $\rightarrow$ B $\rightarrow$ C => A ruft methoden von B auf, nicht von C

\minmeth{high Fan-In, lowFan-Out}
Fan-in: Anzahl der Module, die das Modul verwenden (tolerabel)
Fan-out: Anzahl der verwendeten Module (möglichst klein)

\minmeth{Open-Close Prinzip}
Offen für Erweiterung, geschlossen für Veränderungen


\minmeth{Vererbung}

\textbf{Arten:}

Realisierung: Interfacevererbung => nur Subtyping

Generalisierung/Spezialisierung: Klassenvererbung => Subtyping und Subclassing

\textbf{Facetten:}

\submeth{Subtyping} Vererbung von ähnlichem Verhalten/Contractfragmenten (Signaturen) 

\submeth{Subclassing} Vererbung von Implementierungen 

\minmeth{Liskovsches Substitutionsprinzip}

S1: Subtypen haben gleiche Operationen mit kompatibler Signatur (Compilergeprüft)

S2: Subtypen zeigen das gleiche Verhalten (Programmierergeprüft)

=> Design-By-Contract: Subtypen dürfen Precondition (Anfang der Methode/Parameter) schwächen (AND-Terme weglassen), und/oder Postconditions (Ende der Methode/Rückgaben) stärken (AND Verknüpfung) => Obermenge der gültigen Eingaben, Teilmenge der gültigen Ausgaben.

\minmeth{Ko- Kontravariant}
Szenario: Unterklasse überschreibt Methode in Oberklasse mit anderen Typen in Ein- Ausgabewerten.

\begin{tabular}{|c|c|c|c|}
\hline	
& Parametertyp in Unterklasse & Parametertyp & Rückgabetyp \\
\hline
kovariant & ist \textcolor{red}{Subtyp} des Parameters in Oberklasse & unsicher & sicher\\
\hline
kontravariant & ist \textcolor{red}{Obertyp} des Parameters in Oberklasse & sicher & unsicher \\
\hline
\end{tabular}
