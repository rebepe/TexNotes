\textbf{\textcolor{darkblue}{ OWASP Top 10}}~
OWASP: Open Web Application Security Project
\subsection*{\textcolor{darkgreen}{A1 Injection}}

Hier geht es um einschleusen von Code über Eingabefelder. Meist wird ein zusätzliches Kommando
dazu genutzt, um Daten vom Server zu lesen, schreiben oder zu verändern ohne das dies von
der Anwendung kontrolliert wird.
Unterarten von Injection
	\begin{itemize}
	\item SQL
	\item XML
	\item shell
	\end{itemize}
Kann mit Escapen/ prepared Statements bekämpft werden


\begin{itemize}
\item "Not so blind": Sprechend, d.h. mit Fehlermeldungen der DB
\item Blind-Injectinon: Allgemeine, leicht unterschiedliche Fehlermeldungen
\item Totally-Blind-Injection: Immer identische Meldung => Messung von Laufzeitunterschieden, erzwingen von Fehlern (division-by-zero)

\item Advanced: alles komplexere, hier umgehung Boundary Filtering, z.B. durch encoding der Daten;
\end{itemize}



\subsection*{\textcolor{darkgreen}{A2 Fehler in Authentifizierungs und Session-Management}}

	\begin{itemize}
	\item Session-Management und ID sind falsch implementiert,
	\item Session hijacking,
	\item ID berechenbar,
	\item Passwörter nicht gehasht,
	\item SessionID läuft nicht ab,
	\item keine Transportverschlüsselung
	\end{itemize}

\subsection*{\textcolor{darkgreen}{A3 Cross-Site Scripting}}
	Cookie stehlen: document.cookie
	\begin{itemize}
	\item JavaScript Injection: von Benutzer/Angreifer eingegebener JS-Code wird unescaped an Browser weitergeleitet
	\item Die vom User in den Browser eingegebenen Daten werde nicht validiert bzw. die Daten die an den Server geschickt werden. 
	\item Reflection( ist nur einmalig, bleibt nicht in der DB) 
	\item Persistent ( wird gespeichert und jeder der die Seite aufruft wird injected)
\item Hier hilft meist escapen und testen der Anwendung ( manuelle pentest, reviews usw.) und indirekte Objektreferenz: z.B. Index auf Liste der Konten des Kunden
	\end{itemize}
\subsection*{\textcolor{darkgreen}{A4 Unsichere direkte Objektreferenzen}}

	\begin{itemize}
	\item ID 1 im Browser = ID 1 in der DB => erratbar
	\item Zugriff muss auch auf Ressourcen Ebene vom Server überprüft werden ( id 1 darf nur daten von id 1 sehen)
	\end{itemize}



\subsection*{\textcolor{darkgreen}{A5 Sicherheitsrelevante Fehlkonfiguration}}

	\begin{itemize}
	\item veraltete Softwarekomponenten
	\item nicht benötigte Komponenten aktiv oder installiert
	\item Standardkonten mit initial PW\'{ '}s aktiv 
	\item Fehlermeldungen, Stack Traces geben zuviel Informationen über das System raus
	\item Framework Einstellung sind nicht sicher,
	\end{itemize}

\subsection*{\textcolor{darkgreen}{A6 Verlust der Vertraulichkeit sensibler Daten}}
Data in Motion( Daten im Arbeitsspeicher), Data at Rest ( im Backup) 
	\begin{itemize}
	\item Daten werden in Klartext gespeichert
	\item Daten in Klartext übertragen
	\item schwache/alte Krypto Verfahren
	\item schwache Schlüssel oder falsches Verwalten der Schlüssel
	\item Sicherheitsdirektiven und Header werden nicht genutzt
	\end{itemize}

\subsection*{\textcolor{darkgreen}{A7 Fehlerhafte Autorisierung auf Anwendungsebene}}
\begin{itemize}
	\item Links zu Funktionen werden nur ausgeblendet und dann werden die Rechte nicht vom Server überprüft (Security by obscurity)
	\item serverseitige Prüfung von Authentisierung und Autorisierung wird nicht durchgeführt
	\item serverseitige Prüfung nur mit Daten vom Anwender
\end{itemize}

\subsection*{\textcolor{darkgreen}{A8 Cross-Site Request Forgery}}

\begin{itemize}
	\item geheimer Token bei jeder Anfrage/Link/Formular wird nicht mitgeschickt
	\item Dem User wird meistens ein Request untergeschoben womit ohne Benutzereingabe was gemacht wird
	\item Bsp.: Request wird in einem HTML-Objekt versteckt ( z.B. IMG) , User geht auf die Seite, während er noch die Seite offen hat die den Request entgegen nimmt => Request wird abgeschickt ohne das der Nutzer es weiß
\end{itemize}

\subsection*{\textcolor{darkgreen}{A9 Nutzung von Komponenten mit bekannten Schwachstellen}}

 Ein oder mehrere kleine oder auch große Lücken ( auch hintereinander in unterschiedlichen Programmen) können ausgenutzt werden um an die Server/Daten zu kommen
 
 Verzeichnisse: cve, nvd
 
\subsection*{\textcolor{darkgreen}{A 10 Ungeprüfte Um- und Weiterleitungen}}

\begin{itemize}
	\item Umleiten sollte vermieden werden
	\item Benutzer kann auf Angreiferwebseite weiter geleitet werden ( Phising)
	\item Benutzer informieren wenn er umgeleitet wird
\end{itemize}
