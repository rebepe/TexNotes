	\textbf{\textcolor{darkblue}{ Allgemein}}~
	\begin{itemize}
	
\item	Regel 1: Dont underestimate the power of the dark Side
\item	Regel 2: Benutze Post anfragen wenn Seiteneffekte auftreten.
\item	Regel 3: in einem serverseitigen Kontext gibt es keine clientseitige Sicherheit!
\item	Regel 4: Benutze nie den Referer Header zur Authentifizierung oder Autorisierung
\item	Regel 5: Generiere immer eine neue Session ID, wenn sich der Benutzer anmeldet.
\item	Regel 6: Gebe nie ausführliche Fehlermeldungen an den Client weiter.
\item	Regel 7: Identifiziere jedes Zeichen, das in einem Subsystem als Metazeichen gilt.
\item	Regel 8: Behandel jedes Mal die Metazeichen, wenn Daten an Subsysteme weitergegeben werden.
\item	Regel 8: Übergebe, soweit möglich, Daten getrennt von Steuerinformationen.
\item	Regel 10: Achte auf mehrschichtige Interpretation.
\item	Regel 11: Strebe gestaffelte Abwehr an.
\item	Regel 12: Vertraue nie blind einer API Dokumentation .
\item	Regel 13: Identifiziere alle Quellen, aus denen Eingaben in die Anwendung gelangen.
\item	Regel 14: Achte auf die unsichtbare Sicherheitsbarriere.
\item	Regel 15: Wihtelisten anstatt Blacklisting
\item	Regel 23: Erfinde keine eigenen kryptographische Algorithmen, sondern halte dich an die existierenden.
\item	Regel 24: Speichere Passwörter nie als Klartext.
\item	Regel 25: 	
\item   Regel (Trommler): Implementiere keine kryptographische Algorithmen, benutze existierende Bibliotheken.

	\end{itemize}