\textbf{\textcolor{darkblue}{ Vorlesungsstoff}}~

	\section*{Internet Architektur und E-Commerce}
	
	\subsection*{Sicherheit nicht im Design des Internet}
		\begin{itemize}
		\item 	Fehlen der Theorie
		\item  Geschlossenes Militärnetz
		\item Schwerpunkt: Ausfallsicherheit
		\end{itemize}
	
	\subsection*{Architektur}
	\begin{itemize}
		\item 	Austausch von Forschungsergebnissen
		\item  Vernetzung von Wissen
		\item Webserver liefert statische Dokumente, die vom Browser dargestellt werden
	\end{itemize}

\subsection*{Angriffspunkte}
\begin{itemize}
	\item 	Server hacken
	\item  Nachrichten fälschen
	\item Device vom User hackenn
\end{itemize}

\subsection*{Ziele der Security für E-Commerce}
\begin{itemize}
	\item 	Daten schützen
	\item  Fehler vermeiden
	\item Fehler erkennen und beseitigen
	\item 	Daten schützen
	\item  Fehler vermeiden
	\item Auswirkung der Fehler begrenzen
\end{itemize}

\subsection*{Funktionen für E-Commerce}
\begin{itemize}
	\item 	Dynamische Inhalte
	\item  Zustand ( mehere Anfragen pro kaufvorgang, suchen, auswählen, bezahlen)
	\item Erweiterung des Web Servers (CGI)
	\item nach Appliationserver ( Java Servlets, Java Server Pages, Active Server Pages)
\end{itemize}


	\section*{Programmierfehler}
	
	\subsection*{Software Qualität}
	\begin{itemize}
		\item Definition "Sicher"?
		\item Hoher Zeitaufwand System zu brechen
		\item Erfüllt die Sicherheitsspezifikation
		\item Flaw ( Unerwartetes Verhalten), Vulnerability ( Klasse von Flaws, zb. Buffer Overflow)
	\end{itemize}

\subsection*{ Fehlerklassen}
\begin{itemize}
	\item Pufferüberlauf
	\item Unvollständige Entkopplung ( 30.2 eingeben können)
	\item Serialisierungsfehler ( time-of-check to time-of-use)
\end{itemize}

\subsection*{ Ort des Fehlers}
\subsubsection*{Standardsoftware}
\begin{itemize}
	\item Gefundene Fehler publiziert
	\item Anbieter bietet Lösung an
	\item Aber evtl. auch Exploit verfügbar
\end{itemize}
\subsubsection*{Eigene Webanwendung}
\begin{itemize}
	\item Legitime Benutzer finden selten Fehler
	\item Anzahl Hacker geringer
	\item Exploits unwahrscheinlich
\end{itemize}


\subsection*{ Pufferüberlauf}
\begin{itemize}
	\item 2L Wasser in 1L-Eimer
	\item Example in C : 	Char sample[3]; Sample[3] = 'a';
	\item Nicht entscheidbar
\end{itemize}
\subsubsection*{Folgen des Überlaufs, Überschreiben von:}
\begin{itemize}
	\item Programmdaten
	\item Betriebssytemdaten
	\item Betriebssytemcode
	\item Programmcode
	\item Rechnen mit falschen Daten
	\item Ausführen unerwünschter Instruktionen
\end{itemize}

\subsubsection*{Identifikation eines Überlaufs}
\begin{itemize}
	\item Vermutung der Existenz eines Puffers ( Konsistenzprüfung, Signaturen)
	\item Vermutung über die Länge des Puffers ( Telefonnummer, Postleitzahl)
	\item Zufall ( Absturz bei bestimmten Eingabedaten)
\end{itemize}

\subsubsection*{Überlauf und Sicherheit }
\begin{itemize}
	\item Absturz ( System oder Programm)
	\item Code mit erweiterten Rechten ( Priviligen im OS oder Server Anwendung)
\end{itemize}

\subsection*{Unvollständige Entkopplung}
\begin{itemize}
	\item Benutzereingaben nicht validiert ( 30.2, Auswahlmenüs, Format und Wertebereichsprüfung)
\end{itemize}
	=> Folgen
	\begin{itemize}
		\item Absturz ( Ausfall eines Handelsystems)
		\item Manipulation an Systemen ( SQL oder Shell Injection )
	\end{itemize}

\subsection*{ Serialisierung}
\subsubsection*{Analogon Auto Kauf}
\begin{itemize}
	\item 100 Euro Noten vor Verkäufer zählen
	\item In Umschlag stecken
	\item Verkäufer ablenken
	\item Umschlag mit 1 Euro Noten überreichen
\end{itemize}

\subsubsection*{Serialisierung in E-Commerce }
\begin{itemize}
	\item Links auf Bereiche der Anwendung:
	\item Beim Einloggen geprüft
	\item Nacher nicht mehr
	\item Versteckte Felder
	\item Links mit vorbelegten Parametern
	\item Fehlannahmen ( HTML-Seite nicht manipulierbar ( loool), Versteckte Felder unsichtbar ( firebug usw)
\end{itemize}
=> Folgen : 
\begin{itemize}
	\item Manipulation an Transaktionen ( Preis in Url)
	\item Unerwünschter Zugriff auf beschränkte Funktionen/ Daten anderer Kunden 
\end{itemize}



\section*{Einschleusen von Kommandos}
\subsection*{ SQL Injection}
\begin{itemize}
	\item Rain Forest Puppy 1998
	\item Datenbank Abfragen manipulieren ( SQL Metazeichen)
\end{itemize}
\subsubsection*{Schwachstelle }


\subsubsection*{ Gegenmaßnahmen - Daten säubern }
\begin{itemize}
	\item Anfürhungszeichen verbieten oder Anführungszeichen verdoppeln
	\item PostgeSQL und MySQL: Backslash \
	\item Zahlen durch Parser der Programmiersprache jagen, wieder in Text umwandeln
	\item Nur Stored Procedures ( SP) erlauben, schützt nicht vor aufruf andere SPs
\end{itemize}

\subsubsection*{Gegenmaßnahmen - Prepared Statements }
\begin{itemize}
	\item  Anweisung ist bereits teilwese verarbeitet
	\item Daten haben keine Metazeichen mehr
	\item Typ-Prüfung durch Programmiersprache umwandeln
\end{itemize}