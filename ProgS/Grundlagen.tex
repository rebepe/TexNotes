\minpurp{Grundlagen}
\begin{itemize}
\item Universalität (muss Turing mächtig sein)
\item Implementierbarkeit: Korrekte Programme müssen ausgeführt werden können.
\item Syntax: Form (Anordnung von Zeichen, Ausdrücken...)
 \item Semantik: Bedeutung (Verhalten)
 \item Pragmatik: Zweck (wie, von wem, wozu wird Sprache verwendet)
\end{itemize}
\minpurp{Kategorien}
\submeth{imperativ}
\begin{itemize}
\item prozedural z.B. C
\item objektorientiert z.B. Java, C\#
\item Skriptsprachen (interpretierbar, dyn. typisiert) z.B. JavaScript, Python
\end{itemize}
\submeth{deklarativ}
\begin{itemize}
\item funktional z.B. LISP, Scala
\item logisch z.B. Prolog
\item Domain Specific Language (DSL) z.B.  SQL, XAML
\end{itemize}


