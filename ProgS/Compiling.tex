%\minpurp{Kompilierzeitpunkt}
%\minmeth{Früh}Vorteil: Leistungsfähigere Geräte, 1x Übersetzen spart Ressourcen\\
%aber: Information über Zielplattform nötig
%
%\submeth{AOT (Ahead-of-Time)} Entwicklung, Server/Store, Installation
%
%\minmeth{Spät}Vorteil: geht immer, einfacherer Compiler/Interpreter, besserer Code möglich da vollständige Information über Hardware, OS, Bibliotheken etc.
%
%\submeth{JIT (Just-in-Time)} Direkt vor/während Ablauf\\
%Ablauf: AOT in Zwischencode für hypothetische Maschine (Aufwändige Schritte wie Syntaxanalyse, Typprüfung)\\
%JIT von Zwischencode in optimierten Maschinencode => Performanter als Interpreter, Leichtere Validierung, i.d.R. verzögerter Start
%
\newcommand{\process}[1]{$\rightarrow$ \textcolor{orange}{\textit{#1}}$\rightarrow$\\}
\newcommand{\optprocess}[1]{$\rightarrow$ \textcolor{lightgray}{\textit{#1}}$\rightarrow$\\}
\newcommand{\object}[1]{\textcolor{darkblue}{\textit{#1}~}}
\newcommand{\optobject}[1]{ \textcolor{lightgray}{\textit{#1}}}


\textbf{\textcolor{purple}{ Kompiliervorgang}}
\rule{2cm}{0cm}
\fbox{Legende: \optobject{optional} \object{Artefakt} \process{Verarbeitung} }

\object{Zeichenstrom}
\process{Scanner (Lexikalische Analyse)} 
\object{Token-Strom}
\process{Parser}	
$\left. 
\begin{minipage}{0.39\textwidth}
  \optobject{Ableitungsbaum}
  \process{AST Generierung \& Semantische Analyse}
  \object{Abstract Syntax Tree (AST)}
  \optprocess{Zwischencode generieren/optimieren} 
  \optobject{Zwischencode}
  \process{Maschinencodegenerierung}
  \object{Maschinencode}
  $\leftrightarrow$ \textcolor{lightgray}{\textit{Maschinencodeoptimierung}}
  \end{minipage} \right\rbrace$
{Symboltabelle}

\minmeth{Scanner}
Erkennt Token für Parser (Schlüsselwörter, Bezeichner, Zahlen,$\dots$) \\
Whitespaces, Kommentare, Präprozessoranweisungen,.. beeinflussen Tokenerkennung

Implementierung: Angepasster DFA (Neustart nach Token, erkennt längstes Mögliches Token, Schlüsselworttabelle, Fehler wenn weder passende Kante noch Endzustand)
\includegraphics[angle=90,width=0.35\textwidth]{scanner}
\newcommand{\K}{ \textcolor{red}{k} }


\minmeth{Parser}
Rekonstruiert Ableitungsbaum (bzw. den reduzierten AST) gemäß der Grammatik der Programmiersprache.
\begin{itemize}
\item LL(\K) (Links-nach-Rechts Linksableitung); Vorschau von \K Zeichen (v.a. \K = 1)\\ 
		Parsen Top-Down: \\
		Start: Keller enthält Startwort\\
		Ende: Keller und Eingabe sind leer ($\epsilon$)
	\begin{enumerate}
		\item Predict: betrachte die vordersten \K Zeichen und wähle \textbf{die} passende Regel aus der Grammatik.
		\item Match: Entferne übereinstimmende Terminale aus dem Keller und der Eingabe.
	\end{enumerate}
	
	
\item LR (Links-nach-Rechts Rechtsableitung); mächtiger als LL Parser\\
	Parsen Bottom-Up:\\
		Start: Keller ist leer ($\epsilon$)\\
		Ende: Startsymbol im Keller, Eingabe leer
	\begin{enumerate}
		\item Shift: Lade nächstes Zeichen in den Keller
		\item Reduce: wende wenn möglich eine Regel der Grammatik an.
	\end{enumerate}
		

\end{itemize}

\newpage
\minmeth{Semantische Analyse}
Statische Bindungen (Bezeichner-Objekt $\rightarrow$ Typ-Objekt)$\Rightarrow$ Symboltabelle und AST

\submeth{Symboltabelle} Sammelt Definitionen/Deklarationen von Objekten und damit auch:

\underline{Objektarten}: Namensraum, Typ, Methode/Funktion, Parameter, Variable, Konstante

\underline{Bindungen}: Typ, Adresse, Sichtbarkeit, innerer Gültigkeitsbereich

Gültigkeitsbereiche als Baumstruktur, Mehrdeutige Namen als eindeutiges Symbol


\minmeth{AST}
Reduzierter Ableitungsbaum: 
\begin{itemize}
	\item keine Satzzeichen (desugaring)
	\item Operation Elternknoten, Operanden Kinder
	\item Verkettung der Anweisungen
	\item Deklarationen in Symboltabelle
	\item Namen verweisen auf die Symboltabelle
\end{itemize}

\submeth{Typprüfung}
Typen werden im AST propagiert 
\begin{itemize}
\item Typprüfung
\item Typinferenz (fehlende Typen in Symboltabelle eintragen)
\item Auflösen von Überladungen und Literalen Konstanten
\item implizite Konversionen erkennen
\item generische Typen instanziieren
\end{itemize}



\minmeth{Zwischencode}
Ableitungsbaum - Transpiler (Source-to-Source)

AST - "Lowering": Neue Konstrukte durch alte darstellen (z.B. Iteratoren)

Maschinenunabhängige Optimierung z.B. Function Inlining, Simple constant propagation, loop-unroll, ...



\minmeth{Maschinencode}
Symboltabelle um Adressen erweitern (auch Stackpointer relative)

Registerallokation, Auswahl und Anordnung von Befehlen

\submeth{Maschinenabhängige Optimierung}
Architekturabhängige Befehle/Adressierungen

Cache Coherence

Keyhole-Optimierung: Folgen von Befehlen durch schnellere ersetzen (z.B. *4 durch shift-left 2)

\minmeth{Linken}
statisch: Bibliotheken u. Laufzeitsystem => nach Kompilieren in die Binary gepackt

dynamisch: separate Bibliotheken; Linking Loader bindet vor Ausführung im RAM (dll)

\minmeth{Laufzeitsystem}
Zur Ausführung nötiger Code (der Sprache) z.B. für:

Code-Verifikation, JIT, Exceptions, Garbage-Collector, Linken zur Laufzeit

