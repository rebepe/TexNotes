\minpurp{Speicherverwaltung}
\submeth{Zugriff} Symboltabelle (von Funktion/Struct) speichert offset

\minmeth{Lebensdauer}
\submeth{Global} unbegrenzt\\
\submeth{Stack} allokation/freigabe mit funktionsaufruf/rückgabe\\
\submeth{Heap} explizite reservierung/freigabe (Bei fehlern: Memory Leak/dangling reference)


\minmeth{Alignment}
\submeth{primitive Typen} Adresse muss durch Größe teilbar sein (z.B. Pointer: x8-Adressen)\\
\submeth{Zusammengesetzte Typen (Struct)} Alignment vom größten Element; \\padding auf nächste Alignte größe, auch am ende \\
(z.B. {int, bool} => x4) 

Struct/Class: Reihenfolge wird nicht verändert.

Frame: ggf. durch umordnen optimiert

\minmeth{Größen in Byte}
\begin{tabular}{|r|l|l|l|l|l|l|}
\hline
Sprache & bool & char & int & float & double & Pointer(x64) \\ 
\hline
C++ & 1&1 & 4 &4&8&8\\
\hline
C\# & 1&2 & 4 &4&8&8\\
\hline
Javascript & 1 &&&&8&8\\
\hline
\end{tabular}


\minmeth{Stack}
Aufbau Stackframe(x64 Windows): geregelt im ABI (Application Binary Interface):

Aufruf erzeugt neuen Stackframe (groß genung für alle parameter) und lädt parameter. Aufbau:

\fbox{
\quarterpage{
\newcommand{\shadowspace}{\hline
Shadowspace\\
(8 Byte)\\}
\begin{tabular}{|c|}
\multicolumn{1}{l}{(16bit aligned):}\\
\shadowspace
\shadowspace
\shadowspace
\shadowspace
\hline
$\cdots$\\
Variablen (aligned)\\
ggf. Leerräume\\
$\cdots$ \\
\hline
Return Adress \\
(8 Byte)\\
\hline
\multicolumn{1}{l}{Nächster Frame,} \\
\multicolumn{1}{l}{\textbf{Achtung x16 Alignment}}
\end{tabular}
}}
\quarterpage{
Parameterübergaberegister: 

RCX, RDX, R8, R9\\

Rückgaberegister: RAX \\

Caller saved: 

RAX, RCX, RDX, R8, R9, R10, R11 \\
(Unterprogramm kann überschreiben)\\

Callee preserved:

 RBX, RBP, RDI, RSI, RSP, R12...R15 \\
(Unterprogram  muss sichern(Prolog) \\und restaurieren (Epilog))
}
 



\minmeth{Funktionen}
\submeth{Statische Kette} nichtlokale Variablen in darunterliegenden Stackframes, implementierung durch Pointer

\submeth{Closure} wird bei übergabe/speichern von Funktion gebildet. (Lambda)\\ Besteht aus Funktionszeiger und zeiger auf Heap-Objekt mit den gefangenen Variablen\\
Löst upward Funarg problem(Verweis auf nicht mehr existierende stackframes)\\
Ggf. teil der statische kette



\submeth{Parameterübergabe}
Call by Value: Wert wird kopiert

"Call by Sharing": Call by Value einer Pointervariable ( => Swap nicht möglich)

Call by Reference: Zeiger auf übergebene Variable

Call by const Reference (c++): analog Call by reference, verbot schreibzugriff

Call by name: unausgewertete Übergabe durch implizite(explizite) Closures






\minmeth{RTTI}
Dynamischer Typ (u.a. Vererbung) => Zeiger auf Typobjekt im Deskriptor

Typobjekt: Virtual Table (Adressen der \textbf{virtuellen} Methoden)

In C\#/Java: zus.: Verweis auf Obertyp und andere Metadaten


\minpurp{Heapspeicherverwaltung}
\submeth{Heap}
Blöcke mit längenangaben, verkettet, werden nach speicher durchsucht

\submeth{Probleme} Memory Leak, Dangling References, Garbage

\minmeth{manuell}
freigabe durch programmierer (C++: delete)

\minmeth{Reference Counting}
Zähler für daraufzeigende Pointer, löschen wenn zähler=0

Probleme: Zyklen sind selbsterhaltend 

=> schwache pointer: Beeinflussen Counter nicht -> existenzprüfung!

\submeth{manuell} Programmierer muss zähler selbst erhöhen/reduzieren (Bsp: COM)

\submeth{automatisch} Smartpointer (shared\_ptr / weak\_ptr, unabhängig davon unique\_ptr)

Implementierung (C++) Handle mit Zeiger, \#ptr und \#wptr

\minmeth{Garbage Collection}
\submeth{Wurzeln} direkt verwendbare Objekte: Register,Stack,globale Variablen

Lebendige Objekte sind direkt oder indirekt von Wurzeln erreichbar, Rest Garbage

Vorraussetzung: Zeiger müssen durch Metadaten gefunden werden (Funktionen werden anhand von Rücksprungadresse identifiziert, Objekte mit RTTI)

\submeth{Copy GC /Scavenge/ Stop\& Copy}
Halbierung Speicher in from-space (genutzt) und to-space (ungenutzt);
Tausch der Rollen nach jedem Vorgang

1. Alle direkt von Wurzeln erreichbaren Objekte => in den to-space kopieren; Eintragen der neuen Adresse an alter Speicherstelle (wird in allen Zeigern darauf eingetragen)

2. Dasselbe mit allen von den Objekten im to-space erreichbaren Objekte. (Breitensuche)

Vorteil: Laufzeit linear zu Anzahl gültiger Objekte; kompaktifiziert

Nachteil: hälfte des Speichers nicht nutzbar

\submeth{Mark-Sweep-Compact}

1. Tiefensuche lebendiger Objekte (von der Wurzel aus); Markierung im Mark-Word des Objekts oder globaler Tabelle; Forward-zeiger auf zukünftige Speicherstelle;

2. Korrigieren aller Zeiger (durch Forward-Zeiger)

3. Verschieben des Objekts auf den Forward-Zeiger, zurücksetzen Mark-Word 

Aufwand: Mark-Phase linear zu anzahl lebendiger Objekte, Sweep-Compact linear zu Heap-Größe;

\submeth{Mark-Sweep}
Analog zu Mark-Sweep Compact ohne 3. Schritt; 

Vorteil: keine Zeigerkorrekturen nötig

Nachteil: Fragmentierung des Speichers

\submeth{Mehrgenerationen GC}
kurzlebige temporäre Objekte, langlebige Geschäftsobjekte

=> Aufteilen des Heaps in Generationen (0-$\dots$); Nach überleben mehrerer GCs verschieben in höhere Generation; GC höherer Generationen nur wenn GC unterer Generationen nicht ausreichend. Gen 0: Copy GC; darüber Mark-Sweep(-Compact)

\textit{Schreibbarriere:} Protokollieren der Zeiger Gen 1 auf Gen 0 pro Card (8kB Block) in Card Table (1Byte pro Karte). => Traversieren nicht durch Generation 1 sondern nur in den markierten Cards. 

=> Markierung und Zeigerkorrektur schneller, Zeigerupdates langsamer


