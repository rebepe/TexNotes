\minpurp{Bindungen}
%Namensbindung, Typbindung, Wertbindung, Adressbindung
%
%anonym vs Namensbindung
%
\submeth{statisch} zur Kompilezeit in Symboltabelle

z.B. int i = 32;  
Achtung: auch Typinferenz (var in C\#)

\submeth{dynamisch} zur Laufzeit im Speicher z.B. Werte, virtuelle Methoden, Typen

z.B. Javascript, dynamic in C\# 

=> Ducktyping

\submeth{Methoden} siehe Funktionen

%\minpurp{scope}
%lexikalischer Scope: bindung an den umgebenden Block.
%
%freie Variablen: keine lokale bindung (nicht in diesem Block)
%Funktionen sind Closures wenn alle freien Variablen nicht-lokal gebunden sind
%


\minmeth{misc}
"use strict": Wechsel zu ECMA-Script 5: let, const => variable erst nach definition gültig