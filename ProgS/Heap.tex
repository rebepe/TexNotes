\minpurp{Speicherfreigabe/Garbage Collection}
\minmeth{manuell}
freigabe durch programmierer (C++: delete)

\minmeth{Reference Counting}
Zähler für daraufzeigende Pointer, löschen wenn zähler=0

Probleme: Zyklen sind selbsterhaltend 

=> schwache pointer: Beeinflussen Counter nicht -> existenzprüfung!

\submeth{manuell} Programmierer muss zähler selbst erhöhen/reduzieren (Bsp: COM)

\submeth{automatisch} Smartpointer (shared\_ptr / weak\_ptr, unabhängig davon unique\_ptr)

Implementierung (C++) Handle mit Zeiger, \#ptr und \#wptr

\minmeth{Garbage Collection}
\submeth{Wurzeln} direkt verwendbare Objekte: Register,Stack,globale Variablen

Lebendige Objekte sind direkt oder indirekt von Wurzeln erreichbar, Rest Garbage

Vorraussetzung: Zeiger müssen durch Metadaten gefunden werden (Funktionen werden anhand von Rücksprungadresse identifiziert, Objekte mit RTTI)

\submeth{Copy GC /Scavenge/ Stop\& Copy}
Halbierung Speicher in from-space (genutzt) und to-space (ungenutzt);
Tausch der Rollen nach jedem Vorgang

1. Alle direkt von Wurzeln erreichbaren Objekte => in den to-space kopieren; Eintragen der neuen Adresse an alter Speicherstelle (wird in allen Zeigern darauf eingetragen)

2. Dasselbe mit allen von den Objekten im to-space erreichbaren Objekte. (Breitensuche)

Vorteil: Laufzeit linear zu Anzahl gültiger Objekte; kompaktifiziert

Nachteil: hälfte des Speichers nicht nutzbar

\submeth{Mark-Sweep-Compact}

1. Tiefensuche lebendiger Objekte (von der Wurzel aus); Markierung im Mark-Word des Objekts oder globaler Tabelle; Forward-zeiger auf zukünftige Speicherstelle;

2. Korrigieren aller Zeiger (durch Forward-Zeiger)

3. Verschieben des Objekts auf den Forward-Zeiger, zurücksetzen Mark-Word 

Aufwand: Mark-Phase linear zu anzahl lebendiger Objekte, Sweep-Compact linear zu Heap-Größe;

\submeth{Mark-Sweep}
Analog zu Mark-Sweep Compact ohne 3. Schritt; 

Vorteil: keine Zeigerkorrekturen nötig

Nachteil: Fragmentierung des Speichers

\submeth{Mehrgenerationen GC}
kurzlebige temporäre Objekte, langlebige Geschäftsobjekte

=> Aufteilen des Heaps in Generationen (0-$\dots$); Nach überleben mehrerer GCs verschieben in höhere Generation; GC höherer Generationen nur wenn GC unterer Generationen nicht ausreichend. Gen 0: Copy GC; darüber Mark-Sweep(-Compact)

\textit{Schreibbarriere:} Protokollieren der Zeiger Gen 1 auf Gen 0 pro Card (8kB Block) in Card Table (1Byte pro Karte). => Traversieren nicht durch Generation 1 sondern nur in den markierten Cards. 

=> Markierung und Zeigerkorrektur schneller, Zeigerupdates langsamer

