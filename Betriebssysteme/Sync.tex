\section*{Synchronisation}
Race Conditions: gemeinsam genutzte Betriebsmittel, ergebnis abhänging von ausführungsreihenfolge

Kritische Abschnitte: logisch ununterbrechbare Code-bereiche; synchronisation zum gegenseitigen Ausschluss

Kriterien von Dijkstra: 
-Keine zwei Prozesse dürfen gleichzeitig in einem kritischen Abschnitt sein (mutual exclusion)
- Keine Annahmen über die Abarbeitungsgeschwindigkeit und die Anzahl der Prozesse bzw. Prozessoren
- Kein Prozess außerhalb eines kritischen Abschnitts darf einen anderen Prozess blockieren 
- kein ewiges Warten (fairness condition)

Methoden:
\begin{itemize}
\item busy waiting/spinlock: testen einer Variablen bis zutritt erlaubt
\item interrupts maskieren: nur bei Monoprozessoren, sehr ungünstig
\item Hardwareunterstützung durch atomare Befehle
\item Semaphore/Mutex
\begin{verbatim}
Semaphore x = new Semaphore();
x.Down(); // kritischer Abschnitt besetzt?
c=counter.read(); // kritischer Abschnitt
c++;
counter.write(c);
x.Up();  // Verlassen des kritischen Abschnittes
\end{verbatim}

\end{itemize}

Erzeuger-Verbraucher
\begin{multicols}{2}
\begin{verbatim}
Erzeuger
While (true) {
produce(item);
Down(frei);
Down(mutex);
putInBuffer(item);
Up(mutex);
Up(belegt);
}
Verbraucher
While (true) {
Down(belegt);
Down(mutex);
getFromBuffer(item);
Up(mutex);
Up(frei);
}
\end{verbatim}
\end{multicols}

Monitor:
eine Menge von Prozeduren und Datenstrukturen, die als
Betriebsmittel betrachtet werden
und mehreren Prozessen zugänglich sind,
aber nur von einem Prozess/Thread zu einer Zeit benutzt
werden können

\section*{Deadlock}
Darstellung: Belegungsgraph; Prozess -> Ressource

Bedingungen: 
Mutual exclusion: Ressourcensharing nicht möglich (DVD-Brenner)
Hold-and-wait: Prozesse belegen Ressourcen und wollen weitere
No preemption: Entzug nicht möglich
Circular waiting: gegenseitiges Warten

Strategien: 
Ignorieren (wenn selten)

Erkennen und beheben (Erkennen anhand Belegungsgraph) :
Unterbrechung, Rollback Prozessabbruch Transaktionsabbruch

Dynamisches Verhindern: notwendig Vorwissen über Bedarf
z.B. Bankiers-Algorithmus: prüfen, ob es eine Zuteilungsreihefolge gibt, bei der der Bedarf erfüllt werden kann

Vermeiden:
Mutual exclusion: z.B. virtualisieren mit Spooling
Hold-and-wait: anfordern aller benötigten Ressourcen auf einen Schlag, oder freigabe alter Ressourcen bevor weitere Angefordert werden
No preemption: Entzug nicht möglich
Circular waiting: nummerieren der Ressourcen, anforderung nur in aufsteigender Reihenfolge

Echtzeitsysteme: Priority Ceiling Protocol
Ressource hat Ceiling Priorität = maximale Priorität der Tasks, die sie verwenden werden. Der sie nutzende Task hat während der Nutzung diese Priorität

\section*{Kommunikation}
Nachrichten: verbindungsorientiert vs verbindungslos

Speicher: gemeinsamer Adressraum (Threads), Shared Memory, Datei (Prozess)

Synchron(Blockierend) vs Asynchron

\subsection*{Interprozesskommunikation}
- Pipes und FIFOs (Named Pipes) als Nachrichtenkanal
- Nachrichtenwarteschlangen (Message Queues)
- Gemeinsam genutzter Speicher (Shared Memory)
- Sockets (Ip-Loopback)
\subsubsection*{Pipes}
Unidirektional, bidirektional über mehrere Pipes; Standardausgabe zu Standardeingabe

\begin{verbatim}
int fds[2] / Filedescriptoren für Pipe
pipe(fds);
if (fork() == 0) {
// 1. Kindprozess, Standardausgabe auf Pipe-Schreibseite (Pipe-Eingang) legen und Pipe-Leseseite (Pipe-Ausgang) schließen (wird nicht benötigt)
dup2(fds[1], 1); // 1 = Standardausgabe
close(fds[0]);
write (1, text, strlen(text)+1);
}
else{
if (fork() == 0) {
// 2. Kindprozess, Pipe-Leseseite (Pipe-Ausgang) auf
// Standardeingabe umlenken und Pipe-Schreibseite
// (Pipe-Eingang) schließen
dup2(fds[0], 0); // 0 = standardeingabe
close(fds[1]);
while (count = read(0, buffer, 4))
{
// Pipe in einer Schleife auslesen
prozess Pipe
buffer[count] = 0; // String terminieren
printf(“%s“, buffer) // und ausgeben
}
else {
// Im Vaterprozess: Pipe an beiden Seiten schließen und
// auf das Beenden der Kindprozesse warten
close(fds[0]);
close[fds[1]);
wait(&status);
wait(&status);
}
exit(0);
}
}
\end{verbatim}