\input{../MerkzettelPraeambel.tex}

\xpatchcmd{\tabbing}
  {\def\makelabel}
  {	\setlength{\itemsep}{0mm}
 	\setlength{\parskip}{0pt}
  	\setlength{\parsep}{0pt}  
  \def\makelabel}
  {}
  {}
\definecolor{darkgreen}{rgb}{0.1,0.7,0.1}
\newcommand{\com}[1]{  \textcolor{darkgreen}{ \# #1   } }

\usepackage{tikz}
	\usepackage{tkz-graph}
\usetikzlibrary{shapes.multipart}
\usetikzlibrary{matrix}
\usetikzlibrary{positioning}
\usetikzlibrary{shadows}
\usetikzlibrary{calc}
\usetikzlibrary{shapes,snakes}

\begin{document}
 \section*{bash-Kommandos}
exit/<Strg>D: beenden der Shell

expr : Arithmetische Ausdrücke, bei Vergleichen 0 = false

ls: Verzeichnis ausgeben

wc: worte zählen

echo: ausgabe

cd

cat

man /--help

read var1 var2 ...:  lesen von eingabe in variable

-gt größer als
-lt kleiner 

\subsection*{Ströme}
stdin: 0\\
stdout: 1\\
stderr: 2\\
\subsubsection*{Umleitung}
> Datei überschreiben

>> an Datei anhängen

< aus Datei lesen (in stdin)

| Pipe

\& Prozess im Hintergrund starten

cd; ls  Sequenz
\subsection*{Variablen}
Defintion:  var=12

Ausgabe:    echo \$var

alle Ausgeben:  set
\subsubsection*{Parameter}
\$\# Anzahl der Parameter

\$* Alle Parameter (zusammengefasst)

\$- übergebene Schalter (z.B. -a)

\$@ Alle Parameter (einzeln)

\$? Wert des letzten ausgeführten Kommandos

\$\_ letztes Argument des letzten Kommandos

\$\$ PID dieser Shell

\$\! PID des letzten Hintergrundkommandos


\subsection*{Kontrollstrukturen}
\subsubsection*{if}
if Bedingung     \# if [ "\$v1" = "\$v2" ]

then

[elif Bedingung

then]

[else] 

fi
 
wahr: 0, falsch !=0
\subsubsection*{Mehrfachauswahl}
case \$var in

  1) echo wert = 1

  c) echo wert = c

  *) echo Ungueltig

esac 

\subsection*{for}
for x in \$Liste \# for F in `find -name bla*`

do 
   
done 

\subsection*{Zählschleife}
for (( i = 1; i <= \$max; i++ ))

\subsection*{while}
while 

do 
  
done
\subsection*{until}
until 

do 

done

\section*{Allgemein}
\textbf{Kernel:} wesentliche Dienste des Betriebssystems, möglichst immer im RAM

\textbf{Usermodus:} Ablaufmodus für Anwendungsprogramme, kein Zugriff auf Kernel

\textbf{Kernelmodus:} Privilegiert für Kernelausführung, Wechsel über Syscall/Trap

\textbf{Monolithisch:} alle Module des Kernels in einem Adressraum (ein Prozess)\\
Schichtenmodell: Monolithischer Kern, Module durch Schichten strukturiert\\
\underline{Pro:} effizient \underline{Contra:} Modul kann zu Absturz führen (Blue-Screen)


\textbf{Mikrokern:} Module (z.B. Speicherverwaltung) als eigener Serverprozess, Kernel sorgt nur für Kommunikation zwischen Anwendung und Serverprozess; \\
\underline{Pro:} läuft stabiler Contra: Häufiger Wechsel zwischen User/Kernelmodus

\textbf{Teilhaberbetrieb:} Viele User nutzen ein Programm

\textbf{Teilnehmerbetrieb:} Viele User nutzen eigene Programme

\section*{Unterbrechungen}
\subsection*{Polling}
Busy waiting abfrage des Zustandes
\subsection*{Interrupts}
Unterbrechen der Aktuellen Routine, ausführen der Interrupt Service routine, Auslöser Hardware oder Software(Trap)

Interrupt-Service-Routine: Kernel-Code der auf interrupt reagiert, wird durch Index in Interrupt-Vector-Table aufgerufen

\subsection*{Systemcall/Trap}
Software-Interrupt zur Kommunikation mit BS, z.B. fork(), open(), close()...


\section*{Prozess}
Prozesskontext: Zustandsinformation zum Prozess (Stack, Register,...)
=> Process-Control-Block: Programmzähler, Prozesszustand, Priorität ,Verbrauchte Prozessorzeit seit dem Start des Prozesses, Prozessnummer (PID), Elternprozess (PID), Zugeordnete Betriebsmittel z.B. Dateien

zustände:
bereit, aktiv, beendet, blockiert 

fork() klont Prozess, return 0 für kind, return kind-PID für Eltern

\section*{Scheduling}
Zeitscheibe: zuteilung von zeit-Quanten an Prozesse, wechseln nach ablauf/Blockieren

Ziele: Fairness, Effizienz, Antwortzeit, Verweilzeit (Durchlaufzeit), Durchsatz

Non-Preemptive Scheduling vs Preemptive Scheduling (Prozess kann unterbrochen werden)

\subsection*{Strategien}
\begin{itemize}
\item First Come First Served (FCFS): Der Reihe nach
\item Shortest Job First (SJF); Theoretisch Optimal, kürzester Gewinnt
\item Shortest Remaining Time Next (SRTN): kürzeste Restlaufzeit gewinnt, nicht preemtiv

\item Round-Robin-Scheduling (RR) = Rundlauf-Verfahren: Der Reihe nach
\item Priority Scheduling (PS) statisch/dynamisch: höchste Priorität gewinnt

\item Shortest Remaining Time First (SRTF): SRTN preemtiv
\item Lottery Scheduling: Zufällige Vergabe von CPU-Zeit

\end{itemize}

Echtzeit-Betriebssystem: garantierte zeiten; Tasks in endlosschleife

Parameter: Computation time C $\leq$ Deadline D $\leq$ Period T

=> Wiederholung nach kleinstem gemeinsamen Vielfachen der Perioden => Major Cycle / Hyperperiode


Rate Monotonic Scheduling (RMS): kürzeste Periode gewinnt

Earliest-Deadline First (EDF): nächste Deadline zuerst
\section*{Prozess}
\textbf{Prozesskontext:} Zustandsinformation zum Prozess, \\
Speicherung in \underline{Process-Control-Block:} Programmzähler, Prozesszustand, Priorität ,Verbrauchte Prozessorzeit seit dem Start des Prozesses, Prozessnummer (PID), Elternprozess (PID), Zugeordnete Betriebsmittel z.B. Dateien, $\dots$

\textbf{zustände:} bereit, aktiv, blockiert 

\textbf{fork()} klont Prozess, return 0 für kind, return kind-PID für Eltern

\section*{Scheduling}
\textbf{Ziele:} Fairness, Effizienz, Antwortzeit, Verweilzeit (Durchlaufzeit), Durchsatz

\textbf{Non-Preemptive Scheduling} vs \textbf{Preemptive Scheduling} (Prozess kann unterbrochen werden)



\textbf{Zeitscheibe/Quantum:} zuteilung von Zeit, z.B. 3ms, an Prozesse, wechseln nach ablauf der Zeit/Blockieren;


\subsection*{Strategien}
\begin{itemize}
\item First Come First Served (FCFS): In Ankunftsreihenfolge, Non-Preemptive
\item Shortest Job First (SJF); Theoretisch Optimal, kürzester Gewinnt, Non-Preemptive
\item Shortest Remaining Time Next (SRTN): kürzeste Restlaufzeit gewinnt, nicht preemtiv

\item Round-Robin-Scheduling (RR) = Rundlauf-Verfahren: Der Reihe nach, Preemptiv
\item Priority Scheduling (PS) statisch/dynamisch: höchste Priorität gewinnt, preemptiv
\item Shortest Remaining Time First (SRTF): wie SRTN aber preemptiv
\item Lottery Scheduling: Zufällige Vergabe von CPU-Zeit nicht preemptiv

\end{itemize}

\subsection*{Echtzeit-Betriebssystem:} 
garantierte zeiten; Tasks in endlosschleife

Parameter: Computation time C $\leq$ Deadline D $\leq$ Period T

\textbf{ Major Cycle / Hyperperiode}: Wiederholung nach kleinstem gemeinsamen Vielfachen der Perioden

Rate Monotonic Scheduling (RMS): kürzeste Periode gewinnt, preemptiv

Earliest-Deadline First (EDF): nächste Deadline zuerst, preemptiv
\section*{Synchronisation}
\textbf{Race Conditions:} geteilte Ressource, Ergebnis abhänging von Ausführungsreihenfolge

\textbf{Kritische Abschnitte:} logisch ununterbrechbare Code-bereiche;

\textbf{Kriterien von Dijkstra: }
\begin{itemize}
\item  Keine zwei Prozesse gleichzeitig im gleichen kritischen Abschnitt (mutual exclusion)
\item  Keine Annahmen über die Geschwindigkeit, Anzahl der Prozesse bzw. Prozessoren
\item  Kein Blockieren durch Prozesse außerhalb eines kritischen Abschnittes
\item kein ewiges Warten (fairness condition)
\end{itemize}


Methoden:
\begin{itemize}
\item busy waiting/spinlock: testen einer Variablen bis zutritt erlaubt
\item interrupts maskieren: nur bei Einkernern, sehr ungünstig
\item Hardwareunterstützung durch atomare Befehle
\item Semaphore/Mutex
\begin{verbatim}
Semaphore x = new Semaphore();
x.Down(); // kritischer Abschnitt besetzt?
   c=counter.read(); // kritischer Abschnitt
   c++;
   counter.write(c);
x.Up();  // Verlassen des kritischen Abschnittes
\end{verbatim}

\item Erzeuger-Verbraucher
\begin{multicols}{2}
\begin{verbatim}
Erzeuger:
While (true) {
   produce(item);
   Down(frei);
   Down(mutex);
      putInBuffer(item);
   Up(mutex);
   Up(belegt);
}
Verbraucher:
While (true) {
   Down(belegt);
   Down(mutex);
      getFromBuffer(item);
   Up(mutex);
   Up(frei);
}
\end{verbatim}
\end{multicols}

\item Monitor: Ein Betriebsmittel aus Prozeduren und Daten, geshared zwischen Prozessen, aber nur von einem gleichzeitig nutzbar (bsp. synchronized in Java)

Methoden: Enter, Leave, Wait, Pulse

\end{itemize}
\section*{Deadlock}
Darstellung: Belegungsgraph; Prozess -> \fbox{Ressource}


\begin{tikzpicture}
%\node[ellipse](1){Prozess1} -- \node[rectangle](2){Ressource} -- \node[circle](3){Prozess2};
%\Edge(1)(2); \Edge(2)(3);

\end{tikzpicture}

Bedingungen: 
\begin{itemize}
\item Mutual exclusion: Ressourcensharing nicht möglich (DVD-Brenner)
\item Hold-and-wait: Prozesse belegen Ressourcen und wollen weitere
\item No preemption: Entzug nicht möglich
\item Circular waiting: gegenseitiges Warten
\end{itemize}

Strategien: 
\begin{itemize}
\item Ignorieren (wenn selten)
\item Erkennen und beheben (Erkennen: Zyklus im Belegungsgraph) :\\
Unterbrechung, Rollback Prozessabbruch Transaktionsabbruch
\item Dynamisches Verhindern: notwendig Vorwissen über Bedarf \\
z.B. Bankiers-Algorithmus: prüfen, ob es eine Zuteilungsreihefolge gibt, bei der der Bedarf erfüllt werden kann

\item Vermeiden: vermeiden einer der Deadlock-Bedingungen
\begin{itemize}
\item Mutual exclusion: z.B. virtualisieren mit Spooling
\item Hold-and-wait: anfordern aller benötigten Ressourcen auf einen Schlag, oder freigabe alter Ressourcen bevor weitere Angefordert werden
\item Circular waiting: nummerieren der Ressourcen, anforderung nur in aufsteigender Reihenfolge\\
z.B. in Echtzeitsysteme: Priority Ceiling Protocol\\
Ressource hat Ceiling Priorität = maximale Priorität der Tasks, die sie verwenden werden. Der sie nutzende Task hat während der Nutzung diese Priorität
\end{itemize}
\end{itemize}
\section*{Kommunikation}
\textbf{Nachrichten:} verbindungsorientiert vs verbindungslos, Synchron(Blockierend) vs Asynchron

\textbf{Speicher:} gemeinsamer Adressraum (Threads), Shared Memory, Datei (Prozess)

\subsection*{Interprozesskommunikation}
\begin{itemize}
\item Pipes und FIFOs (Named Pipes) als Nachrichtenkanal
\item Nachrichtenwarteschlangen (Message Queues)
\item Gemeinsam genutzter Speicher (Shared Memory)
\item Sockets (Ip-Loopback)
\end{itemize}

\subsubsection*{Pipes}
Unidirektional, bidirektional über mehrere Pipes; Standardausgabe zu Standardeingabe

\begin{verbatim}
int fds[2] / Filedescriptoren für Pipe
pipe(fds);
if (fork() == 0) {
   // 1. Kindprozess, Standardausgabe auf Pipe-Schreibseite (Pipe-Eingang) legen und Pipe-Leseseite (Pipe-   Ausgang) schließen (wird nicht benötigt)
   dup2(fds[1], 1); // 1 = Standardausgabe
   close(fds[0]);
   write (1, text, strlen(text)+1);
}
else{
   if (fork() == 0) {
      // 2. Kindprozess, Pipe-Leseseite (Pipe-Ausgang) auf
      // Standardeingabe umlenken und Pipe-Schreibseite
      // (Pipe-Eingang) schließen
      dup2(fds[0], 0); // 0 = standardeingabe
      close(fds[1]);
      while (count = read(0, buffer, 4))
      {
         // Pipe in einer Schleife auslesen
         prozess Pipe
         buffer[count] = 0; // String terminieren
         printf(“%s“, buffer) // und ausgeben
      }
   else {
      // Im Vaterprozess: Pipe an beiden Seiten schließen und
      // auf das Beenden der Kindprozesse warten
      close(fds[0]);
      close[fds[1]);
      wait(&status);
      wait(&status);
   }
   exit(0);
}
\end{verbatim}
\section*{Speicherverwaltung}
\textbf{Lokalitätsprinzip:} örtlich: nah beieinanderliegende Daten werden oft zusammen benötigt, zeitlich: Daten werden oft sofort wieder benutzt

\textbf{Adressraum:} benutzbare Adressen (z.B. $2^{32}$); Anordung durch Compiler (z.B.  
\fbox{Code}\fbox{Konstanten}\fbox{ Heap}\fbox{Stack)}

\subsection*{Cache}
write through (sofort) vs write back (bei auslagern) vs write on demand (erst durch expliziten Befehl)

\subsubsection*{Ersetzungsstrategien}
\begin{itemize}
\item LRU (Least Recently Used) ältester Zugriffszeitstempel
\item LFU (Least Frequently Used) Zähler, hochzählen pro zugriff, niedrigster wird zuerst ausgelagert;\\
mit altern: setzen auf 0 nach intervall
\item LRL (Least Recently Loaded) analog zu FIFO, ältester Einlagerunszeitstempel wird ausgelagert
\item Zufällige Seite wird ausgelagert; billig umzusetzen
\end{itemize}


\subsection*{Speicherverwaltung}
\begin{itemize}
\item Monoprogramming: Ein Programm hat gesamten RAM (bis auf BS)
\item feste Partitionierung: Aufteilen des RAM in feste Bereiche, Prozesse bekommt einen zugeteilt
\item Swapping: Prozesse werden im Ganzen ein/ausgelagert (Fragmentierung!)
\item Virtueller Speicher: Aufteilung des Adressraums in Seiten, diese werden ein/ausgelagert. => Adressraum kann größer sein als RAM.
\end{itemize}

\begin{tabular}[width=\textwidth]{|l|}
\hline
\textbf{Umrechnung durch MMU: }\\
$Adresse_{Virtuell}$ = Seite + \underline{Offset};\\
$Adresse_{Real}$ = Rahmen + \underline{Offset};\\
Seite $\Leftrightarrow$ Rahmen: Lookup in Seitentabelle \rule{0.25\textwidth}{0em} \\
\hline
\end{tabular}

Ein/Auslagern in \textbf{Paging Area} auf Festplatte;

Ggf. Caching durch \textbf{TLB} (Translation Lookaside Buffer); \\
TLB-Einträge beinhalten PID

\textbf{Page Fault:} Seite nicht im RAM -> Einlagern nötig

\textbf{Mehrstufige Adresstabellen:} die bitgruppen gibt den index in der entsprechenden Tabelle an, diese enthält die Nummer der nächsten Seitentabelle 

Bsp:$ \underbrace{001100}_{1.Level}\underbrace{110101}_{2.Level}\underbrace{1001010001}_{offset}$

\textbf{invertierte Seitentabelle:} eine Tabelle mit Zuordnung Rahmen->Seiten; aufwändigere Suche, weniger Speicherbedarf (Lookup über Hashtabelle)

\stopwashere

\subsection*{Ersetzungsstrategien}
Demand Paging: nach Page-Fault
\begin{itemize}
\item Belady (optimal): ersetzung der Seiten die am spätesten in der Zukunft wieder verwendet wird
\item FIFO: älteste Seite wird ersetzt, einfach zu implementieren (Verkettete Liste)
\item Second Chance: ähnlich FIFO; ist R-Bit gesetzt hänge hinten an und setze R=0, sonst auslagern
\item NRU (Not Recently Used): \textbf{R}ead-Bit \textbf{M}odified-Bit; Auslagerungsreihenfolge: R=0,M=0;~~~R=0,M=1;~~~R=1,M=0;~~~R=1,M=1
\item LRU (Least Recently Used) Am längsten nicht genutzt wird ausgelagert
\item NFU (Not Frequently Used) Zähler für Zugriffe, auslagerung des Eintrags mit kleinstem Zähler\\
Aging:
\begin{itemize}
	\item als Matrix: setze Bei Zugriff alle in Zeile auf 1, alle in spalte auf 0
	Bsp: $\left( \begin{matrix}
	0 &1 &0 \\
	0 & 0& 0\\
	1 & 1 & 0
	\end{matrix} \right), Zugriff auf 2 => \left( \begin{matrix}
	0 &0 &0 \\
	1 & 0& 1\\
	1 & 0 & 0	
	\end{matrix} \right)$
	\item als Register: shifte die R-Bits von links ein
	Bsp: $R \begin{matrix}
	1\\0\\1\\1\\0
	\end{matrix} Register: \begin{matrix}
	1001\\1010\\0011\\0001\\1101
	\end{matrix}   =>  \begin{matrix}
	1100\\0101\\1001\\1000\\0110
	\end{matrix}$
\end{itemize}
\end{itemize}

Prepaging: Working Set: aktuell Bearbeitete Seiten, versuch daraus die benötigten Seiten zu ermitteln; $\tau$ = Zeitraum für ein Workingset
Bsp: Working Set Clock: 
wenn R == 1
\{ R=0, nächste Seite; \}
sonst
\{ wenn Alter > $\tau$ und M == 0 überschreiben 
   wenn Alter > $\tau $ und M == 1 sichere die geänderte Seite, betrachte nächste seite
\}


Ist der Zeiger wieder am Anfang: wenn seite ausgelagert wurde laufe weiter bis zur nächsten sauberen Seite.
Wurde keine Seite ausgelagert: wähle Zufällige Seite (denn alle gehören zum Working Set)

\subsection*{Speicherbelegung}
Suche nach freien Speicherbereichen; 
\begin{itemize}
\item sequentielle Suche: erster Passender Bereich wird vergeben
\item optimale Suche: möglichst genau passender Bereich wird vergeben um Fragmentierung zu vermeiden
\item Buddy-Technik: schrittweises halbieren des Speichers => externe Fragmentierung sinkt, interne Steigt;
\end{itemize}

\subsection*{Cleaning}
\begin{itemize}
\item Demand-Cleaning: Bei Bedarf
\item Precleaning: Präventives Zurückschreiben, wenn Zeit ist
\item Page-Buffering: Verwaltung in Listen (Modified List, Unmodified List)
\end{itemize}

\section*{Dateiverwaltung}
Dateien: abstrahiert perisistente Speicherung

Dateien, Pseudodateien (Freiliste), Verzeichnisse, Gerätedateien ( abstraktion von geräten, z.B. /dev/sda)

Sequentieller vs Wahlfreier Zugriff

Metadaten: informtionen über die Datei (Zugriffsrechte, Erstelldatum, zugriffsdatum...)

Operationen:
create
• delete
• open
• close
• read
• write
• append
• seek
• get attributes
• set attributes
• rename

Memory-Mapping: einblenden der Datei in den Adressraum eines Prozesses


Master Boot Record auf Sektor 0, Partition Table, Bootblock, Superblock enthält Verwaltungsinformationen zum
Dateisystem (Anzahl der Blöcke,...), Free Blocks (z.B. Bitmap) gibt die freien Blöcke des
Dateisystems an, Rootverzeichnis enthält den Inhalt des Dateisystems, I-nodes;

Dateiimplementierung:
\begin{itemize}
\item Zusammenhängend => Fragmentierung
\item Verkettung der Blöcke => Langsam
\item Verkettet durch FAT im Arbeitsspeicher, hoher Platzbedarf
\item I-Node: enthält: Metadaten, direkte Blockverweise, verweise auf ein-, zwei und dreifach-Indirekte Blöcke
\end{itemize}

Finden einer Datei: suche Verzeichnis anhand Pfad, Verzeichnis enthält I-Node-nummer, bzw. Nummer des 1. Blocks

Hard-Link: Verzeichnisse zeigen auf den selben I-Node;
Symbolic Link: Datei vom Typ LINK enthält Pfad

Virtuelles Filesystem: abstraktionsschicht zwischen Systemaufrufen und Dateisystemen (Windows: Laufwerksbuchstabe, Linux: mounten im Verzeichnisbaum

Blockgröße: je größer desto bessere Datenrate, je kleiner desto Speichereffizienter

Freiblockverwaltung: Liste vs Bitmap

\subsection*{Konsistenzsicherung}
Blockebene:

Zähler vorkommen eines blocks in dateien vs Zähler Block in Freiliste
\begin{itemize}
\item Fehlender Block => einfügen in Freibereichsliste
\item Doppelt in Freibereich => Freibereich anpassen
\item 1x Frei 1x Belegt => Aus Freibereich entfernen
\item Belegt in 2 Dateien => Kopiere Block, ordne ihn einer der Dateien zu; ausgabe für den Benutzer
\end{itemize}

Dateiebene: Prüfe den Linkzähler aller Dateien und passe ihn an

sonstiges, z.B. unsinnige Zugriffsrechte, Dateien größe 0, etc.

\subsubsection*{Journaling}
Vor ausführen einer Aktion: Eintrag in Log; => Aktion kann nach absturz wiederholt werden.

\subsection*{Cache}
Puffern der Blöcke, zugriff über hash, auslagerungsstrategie meist LRU, angepasst nach heuristik;

Konsistenzrelevante Änderungen sofort rausschreiben, Write-Through oder regelmäßiges Rausschreiben von Änderungen sinnvoll

\subsection*{Flash}
wichtig: gleichmäßige Abnutzung => zurückschreiben nicht an selber stelle (Copy-on-write) => "wandering Trees"

Flash Translation Layer: Abbildung Dateisystemadresse auf Sektor

Virtual Block Map:  zuordnungstabelle im RAM, mehrstufig; 
Invertierte: jeder Sektor speichert seine Adresse, oder reservierter sektor pro bank enthält tabelle => scannen beim Mounten

Superblock: reservierte Bänke + Zeittempel, oder durchsuche gesamten Datenträger

löschen: informieren des Controllers mit TRIM-Kommando über gelöschte Blöcke, damit sie wiederverwendet werden können

Logging Dateisysteme: 
ähnlich Journaling, aber es werden nur die Änderungen geschrieben + Checkpoints; Garbage Collection\\
Vorteil: schnelleres Schreiben, passt zu Flash
Nachteil: (Fragmentierung), Garbage Collecting


\section*{NTFS}
Alles steht in MFT (Master-File-Table), Blöcke in Serien angegeben (von-anzahlBlöcke)

\section{Virtualisierung}
Abstraktion von Ressourcen durch Softwareschicht

BS-Virtualisierung: Softwareschicht: Hypervisor oder VMM

\textbf{Vorteile} von VMs: unterschiedliche Betriebssysteme gleichzeitig auf einem Rechner, Hardwareauslatung, Test nicht vertrauenswürdiger Programme, Einfaches Sichern / Wiederaufsetzen /Klonen (Snapshot)

\textbf{Nachteile}: VMs müssen sich reale Ressourcen teilen, Rechnerausfall =>  Ausfall aller VMs, probleme mit spezialhardware, ca. 10\% Leistungsminderung


\textbf{Emulation}: Interpreter der Maschinenbefehle, Optimierung: JiT-Compilation

\textbf{Virtualisierung}: native Ausführung von Nichtprivilegierten Befehlen
Voraussetzung: Ausreichend ähnliche Prozessoren

Technik: Hypervisor fängt kritische Befehle ab, und ersetzt sie durch geeignete Systemaufrufe 
(Implementiert durch Interrupt-Service-Routinen, Hypervisor im Kernelmodus, Gast-BS im Usermodus)


\textbf{Mindestvorraussetzungen}: Unterscheidung zwischen Kernel- und Anwendungsmodus durch 
Prozessor, MMU ,Kriterien nach Popek und Goldberg:

\textbf{Popek und Goldberg: }

\halfpage{
\begin{itemize}
\item  Es gibt privilegierte Befehle, \item diese Lösen Trap aus wenn nicht im Kernelmodus. \item alle sensitiven Befehle: (zustandsverändernd z.B. Zugriff auf I/O oder die MMU,etc.) müssen privilegiert sein.
\end{itemize}
}

Lösung falls nicht alle sensitiven Befehle priviligiert sind: Ersetzung zur Laufzeit oder JiT-Compilation

\subsection{Typen}
\textbf{Typ-1:} Hypervisor als mini-Betriebssystem, ohne Wirt, Treiber ggf. problematisch

\textbf{Typ-2:} Hypervisor als Prozess des Wirtsbetriebssystem, nutzt Wirtstreiber mit

\textbf{Paravirtualisierung}: ersetzung kritischer Befehle im Gast Betriebssystem durch Hypercalls 
=> schneller, vorraussetzung: Quellcode des Gast-BS offen

\subsubsection{Speicher}
Virtuelle Adresse Gast -> Virtuelle Adresse Host -> Reale Adresse Host

Schattentabelle: tabelle für umsetzung $Virt_{Host} -> Real_{Host}$

alternativ Hardwarelösung EPT (Extended Page Table) in MMU

\subsubsection{Treiber}
\textbf{Geräte-Emulation}: Nutzen des Treibers des Hypervisors oder Host-BS, Standardgerät für den Gast; 

schlechter Durchsatz

\textbf{Direktzuweisung eines Geräts}: Gast nutzt Gerät exklusiv mit eigenem Treiber

Effizienteste Möglichkeit, Erfordert spezielle Hardware


\subsubsection{Verschieben einer VM:} kopieren von Dateisystem + RAM von Quelle zum Ziel;
Wiederholt die geänderten Teile umkopieren; anhalten Quelle, kopieren des Rests, Umleiten Netzwerk, Ziel-VM starten, Quell-VM löschen

\section{Cloud}
IaaS = Infrastructure as a Service: Cloud-Provider bietet Zugang zu virtualisierten Rechnern (inkl. BS)
und Speichersystemen

PaaS = Platform aaS: Cloud-Provider bietet Zugang zu Programmierungsumgebungen

SaaS = Software aaS: Cloud-Provider bietet Zugang zu Anwendungsprogrammen

\end{document}