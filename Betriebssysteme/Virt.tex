\section*{Virtualisierung}
Abstraktion der Ressourcen durch Softwareschicht

BS-Virtualisierung: Softwareschicht: Hypervisor oder VMM

Vorteile von VMs: unterschiedliche Betriebssysteme gleichzeitig auf einem Rechner, Hardwareauslatung, Test nicht vertrauenswürdiger Programme, Einfaches Sichern / Wiederaufsetzen /Klonen (Snapshot)

Nachteile: VMs müssen sich reale Ressourcen teilen, Rechnerausfall =>  Ausfall aller VMs, probleme mit spezialhardware, ca. 10\% Leistungsminderung


Emulation: interpretation der Maschinenbefehle, mögliche Optimierung: Just-In-Time Compilation

Virtualisierung: native Ausführung von Nichtprivilegierten Befehlen
Voraussetzung: Ausreichend ähnliche Prozessoren

Technik: Hypervisor fängt bestimmte Befehle ab, und ersetzt sie durch geeignete Befehle 
(Implementiert durch Interrupt-Service-Routinen, Hypervisor im Kernelmodus, Gast-BS im Usermodus)


Mindestvorraussetzungen: Unterscheidung zwischen Kernel- und Anwendungsmodus durch 
Prozessor, MMU ,Kriterien nach Popek und Goldberg:

Popek und Goldberg: Es gibt privilegierte Befehle, diese Lösen Trap aus wenn nicht im Kernelmodus. alle sensitiven Befehle: (zustandsverändernd z.B. Zugriff auf I/O oder die MMU,etc.) müssen privilegiert sein.

Lösung falls nicht alle sensitiven Befehle priviligiert sind: Ersetzung zur Laufzeit/JiT-Compilation

\subsection*{Typen}
Typ-1: Hypervisor als mini-Betriebssystem, ohne Wirt, Treiber ggf. problematisch

Typ-2: Hypervisor als Prozess des Wirtsbetriebssystem, nutzt Wirtstreiber mit

Paravirtualisierung: ersetzung kritischer Befehle im Gast Betriebssystem durch Hypercalls 
=> schneller, vorraussetzung: Quellcode Gast-BS

\subsubsection*{Speicher}
Virtuelle Adresse Gast -> Virtuelle Adresse Host -> Reale Adresse Host

Schattentabelle: tabelle für umsetzung $Virt_{Host} -> Real_{Host}$

alternativ Hardwarelösung EPT (Extended Page Table) in MMU

\subsubsection*{Treiber}
Geräte-Emulation: Nutzen des Treibers des Hypervisors oder Host-BS, Standardgerät für den Gast; 

schlechter Durchsatz

Direktzuweisung eines Geräts: Gast nutzt Gerät exklusiv mit eigenem Treiber

Effizienteste Möglichkeit, Erfordert spezielle Hardware


Verschieben einer VM: kopieren von Dateisystem + RAM von Quelle zum Ziel;
Wiederholt die geänderten Teile umkopieren; anhalten Quelle, kopieren des Rests, Umleiten Netzwerk, Ziel-VM starten, Quell-VM löschen

\section*{Cloud}
IaaS = Infrastructure as a Service: Cloud-Provider bietet Zugang zu virtualisierten Rechnern (inkl. BS)
und Speichersystemen

PaaS = Platform as a Service: Cloud-Provider bietet Zugang zu Programmierungsumgebungen

SaaS = Software as a Service: Cloud-Provider bietet Zugang zu Anwendungsprogrammen
