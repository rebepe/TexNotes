\section{Prozess}
\textbf{Prozesskontext:} Zustandsinformation zum Prozess, \\
Speicherung in \underline{Process-Control-Block:} Programmzähler, Prozesszustand, Priorität ,Verbrauchte Prozessorzeit seit dem Start des Prozesses, Prozessnummer (PID), Elternprozess (PID), Zugeordnete Betriebsmittel z.B. Dateien, $\dots$

\textbf{zustände:} bereit, aktiv, blockiert 

\textbf{fork()} klont Prozess, return 0 für kind, return kind-PID für Eltern

\section{Scheduling}
\textbf{Ziele:} Fairness, Effizienz, Antwortzeit, Verweilzeit (Durchlaufzeit), Durchsatz

\textbf{Non-Preemptive Scheduling} vs \textbf{Preemptive Scheduling} (Prozesse unterbrechbar)



\textbf{Zeitscheibe/Quantum:} zuteilung von Zeit, z.B. 3ms, an Prozesse, wechseln nach ablauf der Zeit/Blockieren;


\subsection{Strategien}
\halfpage{
\begin{itemize}
\item First Come First Served (FCFS): In Ankunftsreihenfolge, Non-Preemptive
\item Shortest Job First (SJF); Theoretisch Optimal, kürzester Gewinnt, Non-Preemptive
\item Shortest Remaining Time Next (SRTN): kürzeste Restlaufzeit gewinnt, Non-Preemptive
\item Round-Robin-Scheduling (RR): vgl. Zeitscheibe, Preemptiv
\item Priority Scheduling (PS) statisch/dynamisch: höchste Priorität gewinnt, preemptiv
\item Shortest Remaining Time First (SRTF): wie SRTN aber preemptiv
\item Lottery Scheduling: Zufällige Vergabe von CPU-Zeit nicht preemptiv
\end{itemize}}

\subsection{Echtzeit-Betriebssystem:} 
garantierte zeiten; Tasks in endlosschleife

Parameter: Computation time C $\leq$ Deadline D $\leq$ Period T

\textbf{ Major Cycle / Hyperperiode}: Wiederholung nach kleinstem gemeinsamen Vielfachen der Perioden

Rate Monotonic Scheduling (RMS): kürzeste Periode gewinnt, preemptiv

Earliest-Deadline First (EDF): nächste Deadline zuerst, preemptiv