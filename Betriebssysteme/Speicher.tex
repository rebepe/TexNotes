\section{Speicherverwaltung}
\textbf{Lokalitätsprinzip:} \underline{örtlich}: nah beieinanderliegende Daten werden oft zusammen benötigt, \underline{zeitlich}: Daten werden oft gleich wieder benutzt

\textbf{Adressraum:} benutzbare Adressen (z.B. $2^{32}$);\\
 Anordung durch Compiler (z.B.  \fbox{Code}\fbox{Konstanten}\fbox{ Heap}\fbox{Stack)}

\textbf{externe Fragmentierung:} Speicher zerfällt in kleine Bereiche \\ 
\textbf{interne Fragmentierung:} Prozesse mehr speicher als nötig => Leere Bereiche

\subsection{Cacheersetzungsstrategien}
\halfpage{
\begin{itemize}
\item LRU (Least Recently Used) ältester Zugriffszeitstempel
\item LFU (Least Frequently Used) Zähler, hochzählen pro zugriff, niedrigster wird zuerst ausgelagert;\\
mit altern: setzen auf 0 nach intervall
\item LRL (Least Recently Loaded) analog zu FIFO, ältester Einlagerunszeitstempel wird ausgelagert
\item Zufällige Seite wird ausgelagert; billig umzusetzen
\end{itemize}}


\subsection{Speicherverwaltung}
\halfpage{
\begin{itemize}
\item Monoprogramming: Ein Programm hat gesamten RAM (bis auf BS)
\item feste Partitionierung: Aufteilen des RAM in \textbf{feste} Bereiche
\item Swapping: Prozesse werden im Ganzen ein/ausgelagert (Fragmentierung!)
\item Virtueller Speicher: Aufteilung des Adressraums in Seiten, diese werden ein/ausgelagert. \textbf{(Paging Area)} => Adressraum kann größer sein als RAM. 
\end{itemize}}

\begin{tabular}[width=\textwidth]{|l|}
\hline
\textbf{Umrechnung durch MMU: }\\
$Seiten = \dfrac{Adresse_{virtuell}}{\textit{Seitengröße}}, Frame = \dfrac{Adresse_{real}}{\textit{Seitengröße}} $\\
Offset = Adresse mod Seitengröße\\
$Adresse_{Virtuell}$ = Seite $\cdot$ Seitengröße + \underline{Offset};\\
$Adresse_{Real}$ = Rahmen $\cdot$ Seitengröße + \underline{Offset};\\
Seite $\Leftrightarrow$ Rahmen: Lookup in Seitentabelle \rule{0.25\textwidth}{0em} \\
\hline
\end{tabular}


\textbf{TLB} (Translation Lookaside Buffer):MMU-Cache für Seiten->Frame zuordnung (ggf. mit PID)

\textbf{Page Fault:} Seite nicht im RAM -> Einlagern nötig

\textbf{Mehrstufige Adresstabellen:} die bitgruppen gibt den index in der entsprechenden Tabelle an, diese enthält die Nummer der nächsten Seitentabelle/des Frames

Bsp:$ \underbrace{01\dots00}_{1.Level}\underbrace{11\dots01}_{2.Level}\underbrace{1001\dots0001}_{offset}$

\textbf{invertierte Seitentabelle:} eine Tabelle mit Zuordnung Rahmen->Seiten;\\ aufwändigere Suche, weniger Speicherbedarf (Lookup über Hashtabelle)
\subsection{Ersetzungsstrategien}
\textbf{Demand Paging}: nach Page-Fault

\halfpage{
\begin{itemize}
\item Belady (optimal): ersetze am spätesten wieder verwendete Seite (Zukunft)
\item FIFO: älteste Seite wird ersetzt, einfach zu implementieren (Verkettete Liste)
\item Second Chance: ähnlich FIFO, aber: ist R-Bit gesetzt stelle hinten an und setze R=0
\item NRU (Not Recently Used): \textbf{R}ead-Bit \textbf{M}odified-Bit; Auslagerungsreihenfolge: R=0,M=0;~~vor~~R=0,M=1;~~vor~~R=1,M=0;~~vor~~R=1,M=1
\item LRU (Least Recently Used) Am längsten nicht genutzt wird ausgelagert
\item NFU (Not Frequently Used) Zugriffszähler, auslagern des kleinsten Zählers\\
\textbf{Aging:}   \newcommand{\cc}[1]{\textcolor{purple}{#1}}  \newcommand{\reg}[1]{\textcolor{blue}{#1}}
\begin{minipage}[t]{\textwidth}
\begin{itemize}
	\item Zähler als Matrix: Bei Zugriff: Zeile auf 1, spalte auf 0
\\$ \begin{matrix}
	1: \\ 2: \\ 3:\end{matrix}	    \left( \begin{matrix}
	0 &1 &0 \\
	0 & 0& 0\\
	1 & 1 & 0
	\end{matrix} \right), Zugriff auf 2 => \left( \begin{matrix}
	0 &\textcolor{red}{0} &0 \\
	\textcolor{green}{1} & \textcolor{red}{0}& \textcolor{green}{1}\\
	1 & \textcolor{red}{0} & 0	
	\end{matrix} \right)$
	\item als Register: shifte die R-Bits von links ein\\
	Bsp: $\begin{matrix}
	R:\\\cc{1}\\\cc{0}\\\cc{1}\\\cc{1}
	\end{matrix} ~~~~ \begin{matrix}
	Register:\\	\reg{100}1\\\reg{101}0\\\reg{001}1\\\reg{000}1
	\end{matrix}   =>  \begin{matrix}Register:\\
	\cc{1}\reg{100}\\\cc{0}\reg{101}\\\cc{1}\reg{001}\\\cc{1}\reg{000}\\
	\end{matrix}$
\end{itemize}\end{minipage}
\end{itemize}
}

\textbf{Prepaging}: Working Set: gerade bearbeitete Seiten, $\tau$ = Zeitraum für ein Workingset\\
Working Set Clock: Seiten bilden Ring, lasse Zeiger laufen:\\
if R == 1:\\
\rule{2em}{0em}R=0, betrachte nächste Seite; \\
else\\
\rule{2em}{0em}if Alter > $\tau$ und M == 0 ersetzen und stop \\
\rule{2em}{0em}if Alter > $\tau $ und M == 1 sichere die geänderte Seite, betrachte nächste seite

Wurde jede Seite betrachtet: wenn seite ausgelagert wurde laufe bis zu sauberer Seite.\\
Wurde keine Seite ausgelagert: wähle Zufällige Seite (denn alle sind im Working Set)

\subsection{Speicherbelegung}
Suche nach freien Speicherbereichen; 

\halfpage{
\begin{itemize}
\item sequentielle Suche: erster Passender Bereich wird vergeben
\item optimale Suche: möglichst genau passender Bereich wird vergeben
\item Buddy-Technik: halbieren des Speichers zu passender größe =>  mehr externe, weniger interne Fragmentierung
\end{itemize}}

\subsection{Cleaning}
Demand-Cleaning vs Precleaning\\(z.B. Page-Buffering: Abarbeiten in Listen (Modified List, Unmodified List))
\section{Dateiverwaltung}
Dateien: abstrahierung der perisistenten Datenhaltung

Dateien, Pseudodateien (z.B. Paging Area), Verzeichnisse, Gerätedateien ( abstraktion von geräten, z.B. /dev/sda)

Sequentieller vs Wahlfreier Zugriff

\textbf{Metadaten:} informtionen über die Datei (Zugriffsrechte, Erstelldatum, zugriffsdatum...)

\textbf{Operationen:}
create, delete, open, close, read, write, append, seek, get attributes, set attributes, rename

\textbf{Memory-Mapping:} einblenden der Datei in den Adressraum eines Prozesses

\subsection{Struktur (Unix)}Master Boot Record auf Sektor 0,\\ Partition Table,\\ Bootblock, \\Superblock (enthält Verwaltungsinformationen zum Dateisystem (Anzahl der Blöcke,...)\\Free Blocks (als Bitmap oder Liste) \\ Rootverzeichnis

\subsection{Dateiimplementierung:}

\halfpage{
\begin{itemize}
\item Zusammenhängend => Fragmentierung
\item Verkettung von Blöcken => Langsam
\item Verkettet durch FAT im Arbeitsspeicher, hoher Platzbedarf
\item I-Node: 
\begin{tabular}{|c|}
\hline Metadaten, Attribute\\
\hline Direkte Verweise (siehe aufgabenstellung) \\
\hline Adresse einfach Indirekter Block \\
\hline Adresse doppelt Indirekter Block \\
\hline 
Adresse 3fach Indirekter Block\\
\hline 
\end{tabular}

dreifach Indirekte -> doppelt Indirekte -> einfach Indirekte -> Blöcke\\
Anzahl der Verweise pro Indirektem Block: $\dfrac{Blockgr"osse}{Adressgr"osse}$
\item NTFS: Alles steht in MFT (Master-File-Table), Dateien als Listen von Serien (StartBlock \& anzahlBlöcke)
\end{itemize}
}

\textbf{Hard-Link:} Verzeichnisse zeigen auf den selben I-Node;
\textbf{Symbolic Link:} Datei vom Typ LINK enthält Pfad

\textbf{Virtuelles Filesystem:} abstraktionsschicht zwischen Systemaufrufen und Dateisystemen \\
(Windows: Laufwerksbuchstabe, Linux: mounten im Verzeichnisbaum

Blockgröße: je größer desto bessere Datenrate, je kleiner desto Speichereffizienter

Freiblockverwaltung: Liste(schrumpft) vs Bitmap(feste größe)

\subsection{Konsistenzsicherung}
\textbf{Blockebene}: Zähle vorkommen eines blocks in dateien, und vorkommen Block in Freiliste
\begin{itemize}
\item Fehlender Block => einfügen in Freibereichsliste
\item Doppelt in Freibereich => Freibereich anpassen
\item 1x Frei 1x Belegt => Aus Freibereich entfernen
\item Belegt in 2 Dateien => Kopiere Block, ordne ihn einer der Dateien zu; ausgabe für den Benutzer
\end{itemize}

\textbf{Dateiebene}: Prüfe den Linkzähler aller Dateien und passe ihn an

sonstiges, z.B. unsinnige Zugriffsrechte, Dateien größe 0, etc.

\textbf{Journaling:} Vor ausführen einer Aktion: Eintrag in Logdatei; => Aktion kann nach absturz wiederholt werden.

\textbf{Logging Dateisysteme: }
ähnlich Journaling, aber es werden nur die Änderungen geschrieben + Checkpoints; Garbage Collection\\
Vorteil: schnelleres Schreiben, passt zu Flash\\
Nachteil: (Fragmentierung), Garbage Collecting
\subsection{Dateicache}
Puffern der Blöcke, zugriff über hash, auslagerungsstrategie meist LRU, angepasst nach heuristik;

Konsistenzrelevante Änderungen sofort rausschreiben, Write-Through oder regelmäßiges Rausschreiben von Änderungen sinnvoll

\subsection{Flash}
wichtig: gleichmäßige Abnutzung => zurückschreiben nicht an selber stelle (Copy-on-write) => "wandering Trees"

Flash Translation Layer: Abbildung Dateisystemadresse auf Sektor

Virtual Block Map:  zuordnungstabelle im RAM, ggf. mehrstufig; 

Invertierte Block Map: jeder Sektor speichert seine Adresse, oder reservierter sektor pro bank enthält tabelle => scannen beim Mounten

Superblock: reservierte Bänke + Zeittempel, oder durchsuche gesamten Datenträger

löschen: informieren des Controllers mit TRIM-Kommando über gelöschte Blöcke, damit sie wiederverwendet werden können




