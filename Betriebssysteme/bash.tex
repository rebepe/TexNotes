 \section*{bash-Kommandos}
exit/<Strg>D: beenden der Shell

expr : Arithmetische Ausdrücke, bei Vergleichen 0 = false

ls: Verzeichnis ausgeben

wc: worte zählen

echo: ausgabe

cd

cat

man /--help

read var1 var2 ...:  lesen von eingabe in variable

-gt größer als
-lt kleiner 

\subsection*{Ströme}
stdin: 0\\
stdout: 1\\
stderr: 2\\
\subsubsection*{Umleitung}
> Datei überschreiben

>> an Datei anhängen

< aus Datei lesen (in stdin)

| Pipe

\& Prozess im Hintergrund starten

cd; ls  Sequenz
\subsection*{Variablen}
Defintion:  var=12

Ausgabe:    echo \$var

alle Ausgeben:  set
\subsubsection*{Parameter}
\$\# Anzahl der Parameter

\$* Alle Parameter (zusammengefasst)

\$- übergebene Schalter (z.B. -a)

\$@ Alle Parameter (einzeln)

\$? Wert des letzten ausgeführten Kommandos

\$\_ letztes Argument des letzten Kommandos

\$\$ PID dieser Shell

\$\! PID des letzten Hintergrundkommandos


\subsection*{Kontrollstrukturen}
\subsubsection*{if}
if Bedingung     \# if [ "\$v1" = "\$v2" ]

then

[elif Bedingung

then]

[else] 

fi
 
wahr: 0, falsch !=0
\subsubsection*{Mehrfachauswahl}
case \$var in

  1) echo wert = 1

  c) echo wert = c

  *) echo Ungueltig

esac 

\subsection*{for}
for x in \$Liste \# for F in `find -name bla*`

do 
   
done 

\subsection*{Zählschleife}
for (( i = 1; i <= \$max; i++ ))

\subsection*{while}
while 

do 
  
done
\subsection*{until}
until 

do 

done
