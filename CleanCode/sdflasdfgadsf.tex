\textbf{\textcolor{darkblue}{Grundlegende Konzepte des OOD}}~
\paragraph{Abstraktion}
\begin{itemize}
	\item Ignorieren irrelevanter Details
\end{itemize}

\paragraph{Kapselung}
\begin{itemize}
	\item Geheimnisprinzip ( information hiding)
	\item trennung von Implementation und Schnittstelle
\end{itemize}

\paragraph{Vererbung}
\begin{itemize}
	\item Erweiterung und Spezialisierung
\end{itemize}

\paragraph{Polymorphismus}
\begin{itemize}
	\item Austauschbarkeit
	\item Gleiche Schnittstelle, anderes Verhalten
\end{itemize}

\paragraph{Kohäsion}
\begin{itemize}
	\item Maß für die Zusammengehörigkeit der Bestandteile einer Komponent
	\item Hohe Kohäsion einer Komponente erleichtert Verständnis, Wartung und Anpassung
\end{itemize}

\paragraph{Kopplung}
\begin{itemize}
	\item Maß für die Abhängigkeiten zwischen Komponenten
	\item Geringe Kopplung erleichtert die Wartbarkeit und macht das Systeme stabiler
\end{itemize}

\paragraph{Allgemeines}
\begin{itemize}
	\item Immer gegen Interfaces anstatt Classes programmieren  (zb. list -> arraylist)
	\item Wenn die equals method implementiert wird dann auch hashcode implementieren
\end{itemize}
