\textbf{\textcolor{darkblue}{Test}}~
\paragraph{Whitebox Tests (WBT)}
Testet eher Struktur als Funktion, Entwicklertest

\subparagraph{Vorteile}
\begin{itemize}
	\item Testen von Teilkomponenten und der internen Funktionsweise
	\item Geringerer organisatorischer Aufwand
	\item Automatisierung durch gute Tool-Unterstützung
\end{itemize}

\subparagraph{Nachteile}
\begin{itemize}
	\item Erfüllung der Spezifikation nicht überprüft
	\item Eventuelles Testen "um Fehler herum"
\end{itemize}

\paragraph{Black-Box-Test (BBT)}
 Entwicklung ohne Kenntnis des Codes, Orientierung an Spezifikation bzw. Anforderungsdefinition,
 funktionsorientiert, kein Entwicklertest
 
\subparagraph{Vorteile}
\begin{itemize}
	\item bessere Verifikation des Gesamtsystems
	\item Testen von semantischen Eigenschaften bei geeigneter Spezifikation
	\item Portabilität von systematisch erstellten Testsequenzen auf plattformunabhängige Implementierungen
\end{itemize}

\subparagraph{Nachteile}
\begin{itemize}
	\item größerer organisatorischer Aufwand
	\item zusätzlich eingefügte Funktionen bei der Implementierung werden nur durch Zufall getestet
	\item Testsequenzen einer unzureichenden Spezifikation sind Unbrauchbar
\end{itemize}

\paragraph{Grey-Box-Text (GBT)}
\begin{itemize}
	\item Vereint die Vorteile aus WBT und BBT
	\item Unterstützt eine „testgetriebene Entwicklung“
	\item Ohne Kenntnis der Implementierung entwickelt
\end{itemize}

\paragraph{Fazit Unittests}
Anzahl der Testabdeckung sagt nur aus das die Codezeilen durch laufen wurden, nicht wie gut die Test sind!